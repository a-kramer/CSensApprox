\documentclass[utf8,english,DIV=12]{scrartcl}

\usepackage[osf]{mathpazo}
\usepackage{FiraSans}
\usepackage[T1]{fontenc}
\usepackage[utf8]{inputenc}
\usepackage[names,dvipsnames]{xcolor}
\usepackage{amsmath,amssymb}
\usepackage{booktabs}
\usepackage{microtype}

\newcommand{\ti}{\ensuremath{t_{\text{i}}}}
\newcommand{\tf}{\ensuremath{t_{\text{f}}}}
\newcommand{\I}{\ensuremath{\mathcal{I}}}
\newcommand{\eye}[1]{\ensuremath{I_{#1}}}

\author{AK}
\date{\today}
\subject{Notes on Sensitivity Approximation}
\title{Transition Matrix}

\begin{document}
\maketitle

\section{Ordinary Differential Equation}
\label{sec:ODEs}

ODE setup:
\begin{align}
  \dot x&=f(x,t;\rho,u)\,,&x&\in\mathbb{R}^n\,\label{eq:ODE}\\
  x(t_0)&=x_0\,,&\rho&\in\mathbb{R}^m\,,
\end{align}
where $\rho$ are parameters that may be subject to estimation
procedures and $u$ parameters that are related to an input to the
system and are known. The solution $x(t;\rho,u)$ to this initial value problem has a
parameter sensitivity:
\begin{equation}
  \label{eq:sens}
  S(x,t;\rho,u)_i^{~j}=\frac{dx_i(t;\rho,u)}{d\rho_j}\,.
\end{equation}
We also define the Jacobian and parameter derivative as:
\begin{align}
  \label{eq:AB}
  A(x,t;\rho,u)_i^{~j} &= \frac{df_i}{dx_j}\,,&  B(x,t;\rho,u)_i^{~j} &= \frac{df_i}{d\rho_j}\,.\\
\end{align}
For any given solution, we will abbreviate the list of arguments to
just $A(t)$ and $B(t)$ when appropriate. We will use the symbol $\eye{n}$ to denote the identity matrix: $(\eye{n})_{ij}=[i=j]$, $i=1,\dots,n$ and $j=1,\dots,n$.

\section{Peano-Baker series}

The transition matrix, as a series expansion:
\begin{equation}
  \label{eq:PHI}
  \Phi(\tf;\ti)=\I_0(\tf;\ti) + \I_1(\tf;\ti) + I_2(\tf;\ti) + ...
\end{equation}
where
\begin{equation}
  \label{eq:I}
  \begin{split}
    \I_0(\tf;\ti)&=\eye{n}\,,\\
    \I_1(\tf;\ti)&=\int_{\ti}^{t}A(x,t;\rho,u) \I_0(\tf;\ti) dt\,,\\
    \I_{k+1}(\tf;\ti)&=\int_{\ti}^{t}A(x,t;\rho,u) \I_k(\tf;\ti) dt\,.
  \end{split}
\end{equation}
Let us assume that we obtain the state variable iteratively (as a
series of vectors) from a numerical solver with adaptive intervals
such that each time step has an initial time $\ti$ and a final time
$\tf$. These intervals can be considered small compared to the
system's dynamics. So, we shall approximate the integral using only
it's boundaries, via the trapezoidal integration rule.

We note that
\begin{equation}
  \label{eq:I123}
  \I_{k\ge 1}(t,t)=0\,,
\end{equation}
because of the integration over an interval of measure $0$.

Initially we get\footnote{we ompit all function arguments except $t$ for ease of notation}:
\begin{equation}
  \label{eq:trapz}
  \begin{split}
    \I_0(\tf;\ti)&=\eye{n}\,,\\
    \I_1(\tf;\ti)&=\int_{\ti}^{t}A(t) \I_0(t;\ti) dt=\frac{\tf-\ti}{2} \left(A(\tf) \underbrace{\I_0(\tf;\ti)}_{\eye{n}} + A(\ti) \underbrace{\I_0(\ti;\ti)}_{\eye{n}}\right)\\
    &=\frac{\tf-\ti}{2} \left(A(\tf) + A(\ti)\right)\,,\\
    \I_2(\tf;\ti) &= \int_{\ti}^{\tf}A(t) \I_1(t;\ti)dt = \frac{\tf-\ti}{2}\left(A(\ti) \I_1(\tf;\ti) + A(\ti) \I_1(\ti;\ti) \right)\\
    &=\frac{\tf-\ti}{2}A(\ti) \I_1(\tf;\ti)\,.
  \end{split}
\end{equation}
The last line in \eqref{eq:trapz} establishes a pattern that holds for all subsequent iterations:
\begin{equation}
  \label{eq:pattern}
  \begin{split}
    \I_{k+1}(\tf;\ti) &= \frac{\tf-\ti}{2}A(\ti) \I_k(\tf;\ti)\,.
  \end{split}
\end{equation}
Let $s:=(\tf-\ti)/2$, then we sum up to:
\begin{align}
  \Phi(\tf;\ti)&=\eye{n} + \I_1(\tf;\ti) + s A(\tf)\I_1(\tf;\ti) + s A(\tf)
                 \I_2(\tf;\ti) +s A(\tf)\I_3(\tf;\ti) + \dots   \label{eq:PhiIteration}\\
    \Phi(\tf;\ti)&=\eye{n} + \I_1(\tf;\ti) + s A(\tf)\I_1(\tf;\ti) + s^2 A(\tf)^2 \I_1(\tf;\ti) + s^3 A(\tf)^3 \I_1(\tf;\ti) +\dots   \label{eq:PhiPower}\\
\end{align}
The series can be calculated in two ways:
\begin{enumerate}
\item calculate the series members iteratively, by multiplying the previously claulcated $\I_n$ with $s A(\tf)$,
\item calculate the power series \eqref{eq:PhiPower} using Horner's method.
\end{enumerate}
Truncating at $k=3$ and writing the power series in reverse order, using Horner's method results in:
\begin{equation}
  \label{eq:PhiHorner}
  \Phi(\tf;\ti)=((\underbrace{(sA(\tf) + \eye{n})}_{W}\cdot sA(\tf) + \eye{n})\cdot sA(\tf) + \eye{n}) \cdot\I_1(\tf;\ti) + \eye{n}\,.
\end{equation}
Like this:
\begin{equation}
  \label{eq:HornerLong}
  \begin{split}
    \Phi(\tf;\ti)&=\eye{n} + \I_1(\tf;\ti) + s A(\tf)\I_1(\tf;\ti) + s^2 A(\tf)^2 \I_1(\tf;\ti) + s^3 A(\tf)^3 \I_1(\tf;\ti) + \dots\\
    &=\eye{n} + (\eye{n} + s A(\tf) + s^2 A(\tf)^2 + s^3 A(\tf)^3) \I_1(\tf;\ti) + \dots \\
    &=\eye{n} + (\eye{n} + s A(\tf)\cdot(\eye{n} + s A(\tf) + s^2 A(\tf)^2)) \I_1(\tf;\ti) + \dots \\
    &=\eye{n} + (\eye{n} + s A(\tf)\cdot(\eye{n} + s A(\tf)\cdot(\eye{n} + s A(\tf)))) \I_1(\tf;\ti) + \dots
  \end{split}
\end{equation}



This formulation makes it easy to calculate $\Phi(\tf;\ti)$ iteratively, using an intermediate variable $W$ (starting with $W=\eye{n}$):
\begin{enumerate}
\item $W \leftarrow s W A(\tf) + \eye{n}$
\item repeat step 1 for $(k-1)$ times
\item $\Phi \leftarrow W \I_1(\tf;\ti) + \eye{n}$
\end{enumerate}
The two different methods may have different compute times, so they
may be both worthy of consideration.

\end{document}
