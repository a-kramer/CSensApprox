\documentclass[utf8,english,DIV=12]{scrartcl}
\usepackage[osf]{mathpazo}
\usepackage{FiraSans}
\usepackage[ttdefault=true]{AnonymousPro}
\usepackage[T1]{fontenc}
\usepackage[utf8]{inputenc}
\usepackage[names,dvipsnames]{xcolor}
\usepackage{amsmath,amssymb}
\usepackage{booktabs}
\usepackage{microtype}
\usepackage{graphicx}
\usepackage{listings}
\lstset{
  basicstyle=\ttfamily\footnotesize,%
  captionpos=b,                    % sets the caption-position to bottom
  commentstyle=\color{green!50!black},    % comment style
  keepspaces=true,                 % keeps spaces in text, useful for keeping indentation of code (possibly needs columns=flexible)
  keywordstyle=\color{blue},       % keyword style
  language=R,                 % the language of the code
  numbers=left,                    % where to put the line-numbers; possible values are (none, left, right)
  numbersep=5pt,                   % how far the line-numbers are from the code
  numberstyle=\tiny\color{gray}, % the style that is used for the line-numbers
  showstringspaces=true,          % underline spaces within strings only
  showtabs=true,                  % show tabs within strings adding particular underscores
  stepnumber=3,                    % the step between two line-numbers. If it's 1, each line will be numbered
  stringstyle=\color{Bittersweet},     % string literal style
}
\usepackage{tikz}

\newcommand{\diag}{\text{diag}}
\newcommand{\Sapprox}{\texttt{S}}

\frenchspacing
\setcapindent{0em}
\addtokomafont{captionlabel}{\bfseries}
\addtokomafont{caption}{\sffamily}
\addtokomafont{pagenumber}{\sffamily}
\pagestyle{headings}

\author{Andrei Kramer <andreikr@kth.se>}
\date{\today}
\title{Solution to Harmonic Oscillator}
\subject{Sensitivity Approximation and Verification}

\begin{document}
\maketitle
\begin{abstract}
  \noindent This part of the documentation justifies the expressions used in
  \texttt{verify.R} where we check the results of the sensitivity
  approximation method against the expected analytical result. The
  solution to the initial value problem of the Harmonic Oscillator is
  of course well known. The important part is its
  parameter-sensitivity (parameter derivative). We treat the damping
  parameter $c$ as a system input and the coefficiant $k$ of the main
  \emph{restoring force} as the \emph{interesting} parameter for sensitivity
  calculations. This choice is arbitrary.
\end{abstract}
\section{Model Definition}
\label{sec:model}

We will use a dampened harmonic oscillator model with unit mass: 
\begin{equation}
  \label{eq:ivp}
  \ddot y = -ky -c \dot y \,,
\end{equation}
to compare numerical $k$-sensitivity approximation methods to the
analytical solution sensitivity. This model is well suited for the
purpose of illustration as the sensitivity is a scalar function.

To use ordinary differential equation solvers, we reformulate the
model as a system of equatuions, with order one:
\begin{align}
  \dot v&=-ky -c v\,,&v(0)&=v_0\label{eq:y}\\
  \dot y&=v\,,&y(0)&=y_0\label{eq:v}\\
\end{align}
This system has a known analytical solution:
\begin{equation}
  \label{eq:sol}
  \begin{split}
    \omega&=\sqrt{k}\,,\\
    r&=\frac{c}{2\omega}\,,\\
    \Rightarrow r\omega&=\frac{c}{2}\,,\\
    y(t;k)&=a\exp(-r\omega t)\cos(\sqrt{1-r^2}\omega t+\phi)\,,\\
    \Rightarrow v(t;k)&=-a\exp(-r\omega t)(r\omega)\cos(\sqrt{1-r^2}\omega t+\phi)\\
    &\quad~- a\exp(-r\omega t)\sin(\sqrt{1-r^2}\omega t+\phi) \sqrt{1-r^2}\omega\,,\\
    &=-y(t;k)\left(r\omega + \tan(\sqrt{1-r^2}\omega t+\phi) \sqrt{1-r^2}\omega\right)\,,
  \end{split}
\end{equation}
where $a$ and $\phi$ are to be determined from the initial conditions
$v_0$ and $y_0$:
\begin{equation}
  \label{eq:init}
  \begin{split}
    y(0;k)&=a\cos(\phi)=y_0\,,\\
    v(0;k)&=-y_0r\omega-y_0\tan(\phi)\sqrt{1-r^2}\omega=v_0 \,,
  \end{split}
\end{equation}
In Section~\ref{sec:proof} we show that \eqref{eq:sol} solves the
ordinary differential equation~\eqref{eq:ivp}.

All four constants $a$, $\omega$, $\phi$, and $r$ are functions of $(k,c)$, so we could instead write:
$a(k,c)$, $r(k,c)$, etc.. But, to ease notation and avoid too many
parentheses, we will instead note the dependence on $k$ via
subscripts. Since we are interested in the sensitivity of $y$ only
with repsect to $k$ we will drop the dependence on $c$ in notation.

\subsection{Phase and Amplitude}
\label{sec:aphi}

We solve the initial value equations~\eqref{eq:init} for $\phi$:
\begin{align}
    \frac{v_0 + y_0r_k\omega_k}{y_0\sqrt{1-r_k^2}\omega_k} &= - \tan(\phi_k)\,\,,\label{eq:tanphi}\\
    \arctan\left(-\frac{2v_0 + y_0{c}}{2y_0\sqrt{1-r_k^2}\omega_k}\right) &= \phi_k\,.\label{eq:phi}
\end{align}
Here, we note that this solution for $\phi$ is also a function of $k$ and append this note as a subscript. 
This result makes $a$ immediately available as
\begin{equation}
  \label{eq:a}
  a_k = \frac{y_0}{\cos(\phi_k)}\,,
\end{equation}
where we make the dependence on $k$ noted once again. We disregard the dependence on 

\subsection{Sensitivity}
\label{sec:sens}

Because we are calculating the derivative with respect to $k$ it is
useful to rewrite the solution and make it explicit when a term does
depend on $k$, and show where $k$ cancels:
\begin{equation}
  \label{eq:yk}
  \begin{split}
    \Gamma_k:&=\sqrt{1-r_k^2}\,,\\
    \Gamma_k\omega_k&=\sqrt{1-\left(\frac{c}{2\sqrt{k}}\right)^2}\sqrt{k}=\sqrt{k-\frac{c^2k}{4k}}=\sqrt{k-\frac{c^2}{4}}\,,\\
  \Rightarrow y(t;k)&=a\exp\left(-\frac{c}{2}t\right)\cos\left(t\sqrt{k-\frac{c^2}{4}}+\phi_k\right)\,,
\end{split}
\end{equation}


The sensitivity of the solution $y(t;k)$ with respect to the parameter $k$ can be
obtained by straight forward differentiation, albeit with many terms. We collect a list of derivatives:
\begin{align}
  a_k &= \frac{y_0}{\cos(\phi_k)} & \frac{da_k}{dk} &= \frac{y_0\tan(\phi_k)}{\cos(\phi_k)}\frac{d\phi_k}{dk}=a_k\tan(\phi_k)\frac{d\phi_k}{dk}\,,\\
  \omega_k&=\sqrt{k} & \frac{d\omega_k}{dk}&=\frac{1}{2\sqrt{k}}\,,\\
  r_k&=\frac{c}{2\omega_k} & \frac{dr_k}{dk}&=-\frac{c}{2\omega_k^2}\frac{d\omega_k}{dk}=-\frac{c}{2 k}\frac{1}{2\sqrt{k}}=-\frac{r_k}{2k}\,,\\
  r_k\omega_k&=\frac{c}{2} & \frac{d(r_k\omega_k )}{dk}&=0\,,\\
  \Gamma_k&=\sqrt{1-r_k^2} & \frac{d\Gamma_k}{dk}&=\frac{2 r_k}{2\sqrt{1-r_k^2}}\frac{dr_k}{dk}=-\frac{r_k}{k}\frac{r_k}{2\Gamma_k}\,,\\
  \Gamma_k\omega_k&=\sqrt{k-\frac{c^2}{4}} & \frac{d(\Gamma_k\omega_k)}{dk}&=\frac{1}{2\sqrt{k-\frac{c^2}{4}}}=\frac{1}{2\Gamma_k\omega_k}\,,
\end{align}
First we take the derivative of $\phi$ from~\eqref{eq:tanphi}:
\begin{align}
  -\frac{d}{dk}\left(\frac{2v_0 + y_0c}{2y_0\Gamma_k\omega_k}\right) &= \frac{2}{cos(2\phi+1)} \frac{d\phi}{dk}\,,\\
  -\left(-\frac{2v_0 + y_0c}{4y_0(\Gamma_k\omega_k)^3}\right) &= \frac{2}{cos(2\phi+1)} \frac{d\phi_k}{dk}\,,
\end{align}
We solve for the derivative of $\phi$:
\begin{equation}
  \frac{d\phi_k}{dk}=\cos(2\phi_k+1)\frac{2v_0 + y_0c}{8y_0(\Gamma_k\omega_k)^3}\,,
\end{equation}
Using these results, we assemble the full sensitivity:
\begin{equation}
  \label{eq:dydk}
  \begin{split}
    \frac{dy(t;k)}{dk}&=\frac{d}{dk}\left(a_k\exp\left(-\frac{c}{2}t\right)\cos(\Gamma_k\omega_k t+\phi_k)\right)\,,\\
    &=\exp\left(-\frac{c}{2}t\right)\frac{d}{dk}\left(a_k\cos(\Gamma_k\omega_k t+\phi_k)\right)\,,\\
    &=\exp\left(-\frac{c}{2}t\right)\left(\frac{da_k}{dk}\cos(\Gamma_k\omega_k t+\phi_k) - a_k \sin(\Gamma_k\omega_k t+\phi_k)\left(\frac{t}{2\Gamma_k\omega_k}+\frac{d\phi_k}{dk}\right)\right)\,,\\
    &=\exp\left(-\frac{c}{2}t\right)\left(a_k\tan(\phi_k)\frac{d\phi_k}{dk}\cos(\Gamma_k\omega_k t+\phi_k) - a_k \sin(\Gamma_k\omega_k t+\phi_k)\left(\frac{t}{2\Gamma_k\omega_k}+\frac{d\phi_k}{dk}\right)\right)\,,\\
    &=y(t;k)\left(\tan(\phi_k)\frac{d\phi_k}{dk} - \tan(\Gamma_k\omega_k t+\phi_k)\left(\frac{t}{2\Gamma_k\omega_k}+\frac{d\phi_k}{dk}\right)\right)\,,
  \end{split}
\end{equation}
The last line is suitable for direct evaluation, given the parameters $(k,c)$. In \texttt{R} we calculate this using the function defined in Listing~\ref{lst:sens}.  
\begin{lstlisting}[caption={Calculation of the sensitivity in \texttt{R}. The variables have slightly different names: \texttt{srr1(k)} is $\Gamma_k$ and \texttt{srr1w(k)} is $\Gamma_k\omega_k$. The independent variable $t$ is \texttt{x} in the code.},label={lst:sens},float]
  w <- function(k) sqrt(k)
  r <- function(k) c/(2*w(k))                          # damping ratio
  srr1 <- function(k) sqrt(1-r(k)^2)                   # convenience
  ## srr1*w
  srr1w <- function(k) sqrt(k-0.25*c^2)
  ## phase and amplitude
  f <- function(k) atan(-(2*v0+y0*c)/(2*y0*srr1w(k))) 
  a <- function(k) y0/cos(f(k))
  ## derivatives
  dwdk <- function(k) 1/(2*w(k))
  dfdk <- function(k) cos(2*f(k) + 1)*(2*v0 + y0*c)/(8*y0*srr1w(k)^3)
  dadk <- function(k) y0 * (tan(f(k))/cos(f(k))) * dfdk(k)
  ## y and S = dy/dk
  y <- function(x,k) a(k)*exp(-0.5*c*x)*cos(srr1w(k)*x + f(k))
  S <- function(x,k) {
    y(x,k)*(  tan(f(k))*dfdk(k)
            - tan(srr1w(k)*x+f(k))*(0.5*x/srr1w(k) + dfdk(k)))
  }
\end{lstlisting}


\section{Proof for the Solution}
\label{sec:proof}

We insert the proposed solution~\eqref{eq:sol} into~\eqref{eq:ivp}. First we perform al needed derivatives:
\begin{equation}
  \begin{split}
    y(t;k)&=a\exp(-r\omega t)\cos(\sqrt{1-r^2}\omega t+\phi)\,,\\
    \dot y(t;k)&=-y(t;k)r\omega - a\exp(-r\omega t)\sin(\sqrt{1-r^2}\omega t+\phi) \sqrt{1-r^2}\omega\,,
  \end{split}\label{eq:proof}
\end{equation}
The second derivative of $y$:
\begin{multline}
  \ddot y(t;k)=-\dot y(t;k) r\omega \\
  - \left(a\exp(-r\omega t)(-r\omega)\sin(\sqrt{1-r^2}\omega t+\phi) \sqrt{1-r^2}\omega\right.\\
  + \left.a\exp(-r\omega t)\cos(\sqrt{1-r^2}\omega t+\phi) \left(\sqrt{1-r^2}\omega\right)^2\right)\,,  
\end{multline}
which simplifies to:
\begin{equation}
  \label{eq:ddoty}
  \begin{split}
  \ddot y(t;k)&=-\dot y(t;k) r\omega - \left(\underbrace{-a\exp(-r\omega t)\sin(\sqrt{1-r^2}\omega t+\phi)\sqrt{1-r^2}\omega}_{\dot y + yr\omega}   r\omega
    + y(t;k) \left(\sqrt{1-r^2}\omega\right)^2\right)\,, \\
  &=-\dot y(t;k) r\omega - \left(\dot y(t;k)r\omega + y(t;k)(r\omega)^2 
    + y(t;k) \left(\sqrt{1-r^2}\omega\right)^2\right)\,,\\
  &=-2\dot y(t;k) r\omega - \left(y(t;k)(r\omega)^2 
    + y(t;k) (1-r^2)\omega^2\right)\,,\\
  &=-2\dot y(t;k) r\omega - \left(y(t;k)\omega^2\right)\,,\\
  &=-2\dot y(t;k) \frac{c}{2} - y(t;k)k = - k y(t;k) - c\dot y(t;k)\,,
\end{split}
\end{equation}
which reconstructs the original ODE in~\eqref{eq:ivp}.

\section{Numerical Simulation Outcomes}
\label{sec:results}

In the following sections, we show comparisons of numerical solutions
to the initial value problem~\eqref{eq:ivp} under various conditions:
with or without damping (friction), with and without a constant
driving force. In all cases, we do the numerical integration without
forward sensitivity analysis, but use the analytical solution
in~\eqref{eq:dydk}.

By choice, the sensitivity is a scalar function, so we show its values
directly. We have also included sensitivity calculations using finite
differences and using the Cauchy integral formula on the analytical
solution~\eqref{eq:sol}. The analytical solution for the state
variable $y$ is well known and simple enough to be implemented
correctly. So, we use these discrete methods to check the
analytical solution for the sensitivity.

In general, analytical solutions are not guaranteed to be
available. An alternative is to use the sensitivity to predict a
trajectory at slightly changed parameters $k+\Delta_k$, using a linear
predictor:
\begin{equation}
  \label{eq:linear}
 y(t;k+\Delta_k) = 
 y(t;k)+S_y(t;k)\cdot\Delta_k + \mathcal{O}(\Delta_k^2)\,,
\end{equation}
For small enough $\Delta_k$, we neglect the second order effects.
But, since we are using an approximation $\Sapprox$ of the
sensitivity, we make an order one error for each time point $j$:
\begin{equation}
  \label{eq:SapproxError}
  y(t_j;k+\Delta_k)\approx
  y(t_j;k)+(\Sapprox_y(t_j;k) + \Delta_S(t_j;k))\cdot\Delta_k \,,
\end{equation}
where $\Delta_S(t_j;k)$ is the error we made when approximating the sensitivity:
\begin{equation}
  \label{eq:SapproxError}
\Rightarrow \|\Delta_{S}(t_j;k)\cdot\Delta_k\|\approx \left\|y(t_j;k+\Delta_k)
  - (y(t_j;k)+\Sapprox_y(t_j;k)\cdot\Delta_k) \right\|\,.
\end{equation}
This trajectory can of course be simulated directly using the changed
parameters whenever analytical solutions are unavailable. This makes
the above error estimate generally applicable.

The average sensitivity error can be estimated using the discrepancy between
the prediction and direct simulation/solution:
\begin{equation}
  \label{eq:linerr}
  \delta_S:=\frac{1}{T}\sum_{j=1}^T \frac{\left\| y(t_j;k+\Delta_k) - (y(t_j;k) + S_y(t_j;k)\cdot\Delta_k)\right\|}{\|\Delta_k\|}\,,
\end{equation}
where $T$ is the number of discrete time points returned by the numerical solver. 

\subsection{Oscillations}
\label{sec:nodamping}

Under the conditions $c=0$, the system sustains oscilations of constant height
(given non-zero initial values). We integrate the model using the
\texttt{BDF} solver in the \textsc{gsl} library \texttt{gsl\_odeiv2}.

The first step is to verify that the numerical solution is
sufficiently accurate. Both the analytical and numerical trajectories
are shown in Figure~\ref{fig:dampedGSLvsASolTraj}. They agree within
the requested accuracy.

\begin{figure}\sffamily\firalining
  \centering
  % Created by tikzDevice version 0.12.3.1 on 2021-06-16 15:27:58
% !TEX encoding = UTF-8 Unicode
\begin{tikzpicture}[x=1pt,y=1pt]
\definecolor{fillColor}{RGB}{255,255,255}
\path[use as bounding box,fill=fillColor,fill opacity=0.00] (0,0) rectangle (462.53,346.90);
\begin{scope}
\path[clip] ( 49.20, 61.20) rectangle (437.33,297.70);
\definecolor{drawColor}{RGB}{0,0,0}

\path[draw=drawColor,line width= 0.4pt,line join=round,line cap=round] ( 63.58,278.98) --
	( 63.66,278.98) --
	( 63.76,278.98) --
	( 63.86,278.98) --
	( 63.97,278.97) --
	( 64.07,278.96) --
	( 64.18,278.95) --
	( 64.28,278.94) --
	( 64.43,278.92) --
	( 64.66,278.89) --
	( 65.06,278.81) --
	( 65.57,278.66) --
	( 66.09,278.47) --
	( 66.61,278.23) --
	( 67.13,277.96) --
	( 67.65,277.64) --
	( 68.17,277.27) --
	( 68.69,276.87) --
	( 69.21,276.42) --
	( 69.73,275.92) --
	( 70.38,275.24) --
	( 71.38,274.08) --
	( 73.59,270.95) --
	( 75.80,267.10) --
	( 78.00,262.55) --
	( 80.21,257.34) --
	( 82.41,251.52) --
	( 84.62,245.12) --
	( 86.83,238.20) --
	( 89.03,230.82) --
	( 91.24,223.03) --
	( 94.18,212.12) --
	( 96.82,201.93) --
	( 99.46,191.48) --
	(102.10,180.90) --
	(104.74,170.29) --
	(107.51,159.27) --
	(111.06,145.54) --
	(116.27,126.73) --
	(121.09,111.39) --
	(125.91, 98.62) --
	(130.73, 88.91) --
	(135.55, 82.62) --
	(140.37, 79.99) --
	(145.19, 81.12) --
	(150.01, 85.96) --
	(155.12, 94.95) --
	(160.23,107.53) --
	(165.34,123.16) --
	(170.45,141.18) --
	(174.75,157.63) --
	(179.05,174.74) --
	(183.41,192.23) --
	(188.93,213.76) --
	(194.08,232.30) --
	(199.22,248.57) --
	(204.37,261.86) --
	(209.51,271.61) --
	(214.04,276.91) --
	(218.57,278.96) --
	(223.10,277.69) --
	(228.36,272.11) --
	(233.87,261.80) --
	(238.49,249.99) --
	(243.11,235.74) --
	(247.88,218.99) --
	(253.44,197.62) --
	(256.52,185.34) --
	(259.60,172.97) --
	(263.48,157.52) --
	(267.90,140.63) --
	(273.64,120.62) --
	(278.18,106.97) --
	(282.71, 95.75) --
	(287.25, 87.34) --
	(292.62, 81.38) --
	(296.37, 79.93) --
	(300.13, 80.77) --
	(304.82, 85.01) --
	(310.77, 95.19) --
	(315.47,106.72) --
	(320.17,120.86) --
	(324.88,137.11) --
	(330.10,156.93) --
	(333.80,171.60) --
	(337.50,186.46) --
	(341.87,203.80) --
	(347.41,224.61) --
	(352.43,241.56) --
	(357.46,255.97) --
	(362.48,267.23) --
	(367.51,274.89) --
	(371.90,278.38) --
	(376.30,278.76) --
	(380.79,275.92) --
	(386.20,268.31) --
	(391.77,256.06) --
	(396.26,243.35) --
	(400.75,228.55) --
	(405.57,210.84) --
	(410.40,191.93) --
	(413.79,178.35) --
	(417.17,164.79) --
	(422.95,142.41);
\end{scope}
\begin{scope}
\path[clip] (  0.00,  0.00) rectangle (462.53,346.90);
\definecolor{drawColor}{RGB}{0,0,0}

\path[draw=drawColor,line width= 0.4pt,line join=round,line cap=round] ( 63.58, 61.20) -- (363.85, 61.20);

\path[draw=drawColor,line width= 0.4pt,line join=round,line cap=round] ( 63.58, 61.20) -- ( 63.58, 55.20);

\path[draw=drawColor,line width= 0.4pt,line join=round,line cap=round] (138.64, 61.20) -- (138.64, 55.20);

\path[draw=drawColor,line width= 0.4pt,line join=round,line cap=round] (213.71, 61.20) -- (213.71, 55.20);

\path[draw=drawColor,line width= 0.4pt,line join=round,line cap=round] (288.78, 61.20) -- (288.78, 55.20);

\path[draw=drawColor,line width= 0.4pt,line join=round,line cap=round] (363.85, 61.20) -- (363.85, 55.20);

\node[text=drawColor,anchor=base,inner sep=0pt, outer sep=0pt, scale=  1.00] at ( 63.58, 39.60) {0};

\node[text=drawColor,anchor=base,inner sep=0pt, outer sep=0pt, scale=  1.00] at (138.64, 39.60) {5};

\node[text=drawColor,anchor=base,inner sep=0pt, outer sep=0pt, scale=  1.00] at (213.71, 39.60) {10};

\node[text=drawColor,anchor=base,inner sep=0pt, outer sep=0pt, scale=  1.00] at (288.78, 39.60) {15};

\node[text=drawColor,anchor=base,inner sep=0pt, outer sep=0pt, scale=  1.00] at (363.85, 39.60) {20};

\path[draw=drawColor,line width= 0.4pt,line join=round,line cap=round] ( 49.20, 79.91) -- ( 49.20,278.98);

\path[draw=drawColor,line width= 0.4pt,line join=round,line cap=round] ( 49.20, 79.91) -- ( 43.20, 79.91);

\path[draw=drawColor,line width= 0.4pt,line join=round,line cap=round] ( 49.20,129.68) -- ( 43.20,129.68);

\path[draw=drawColor,line width= 0.4pt,line join=round,line cap=round] ( 49.20,179.45) -- ( 43.20,179.45);

\path[draw=drawColor,line width= 0.4pt,line join=round,line cap=round] ( 49.20,229.22) -- ( 43.20,229.22);

\path[draw=drawColor,line width= 0.4pt,line join=round,line cap=round] ( 49.20,278.98) -- ( 43.20,278.98);

\node[text=drawColor,rotate= 90.00,anchor=base,inner sep=0pt, outer sep=0pt, scale=  1.00] at ( 34.80, 79.91) {-1.0};

\node[text=drawColor,rotate= 90.00,anchor=base,inner sep=0pt, outer sep=0pt, scale=  1.00] at ( 34.80,129.68) {-0.5};

\node[text=drawColor,rotate= 90.00,anchor=base,inner sep=0pt, outer sep=0pt, scale=  1.00] at ( 34.80,179.45) {0.0};

\node[text=drawColor,rotate= 90.00,anchor=base,inner sep=0pt, outer sep=0pt, scale=  1.00] at ( 34.80,229.22) {0.5};

\node[text=drawColor,rotate= 90.00,anchor=base,inner sep=0pt, outer sep=0pt, scale=  1.00] at ( 34.80,278.98) {1.0};

\path[draw=drawColor,line width= 0.4pt,line join=round,line cap=round] ( 49.20, 61.20) --
	(437.33, 61.20) --
	(437.33,297.70) --
	( 49.20,297.70) --
	( 49.20, 61.20);
\end{scope}
\begin{scope}
\path[clip] (  0.00,  0.00) rectangle (462.53,346.90);
\definecolor{drawColor}{RGB}{0,0,0}

\node[text=drawColor,anchor=base,inner sep=0pt, outer sep=0pt, scale=  1.20] at (243.26,318.16) {\bfseries  no damping no driving force};

\node[text=drawColor,anchor=base,inner sep=0pt, outer sep=0pt, scale=  1.00] at (243.26,  3.60) {analytical solution (missmatch: 0.00012)};

\node[text=drawColor,anchor=base,inner sep=0pt, outer sep=0pt, scale=  1.00] at (243.26, 15.60) {$t$};

\node[text=drawColor,rotate= 90.00,anchor=base,inner sep=0pt, outer sep=0pt, scale=  1.00] at ( 10.80,179.45) {state variables $(v,y)$};
\end{scope}
\begin{scope}
\path[clip] ( 49.20, 61.20) rectangle (437.33,297.70);
\definecolor{drawColor}{RGB}{0,0,0}

\path[draw=drawColor,line width= 0.4pt,line join=round,line cap=round] ( 63.58,278.98) circle (  2.25);

\path[draw=drawColor,line width= 0.4pt,line join=round,line cap=round] ( 63.66,278.98) circle (  2.25);

\path[draw=drawColor,line width= 0.4pt,line join=round,line cap=round] ( 63.76,278.98) circle (  2.25);

\path[draw=drawColor,line width= 0.4pt,line join=round,line cap=round] ( 63.86,278.97) circle (  2.25);

\path[draw=drawColor,line width= 0.4pt,line join=round,line cap=round] ( 63.97,278.97) circle (  2.25);

\path[draw=drawColor,line width= 0.4pt,line join=round,line cap=round] ( 64.07,278.96) circle (  2.25);

\path[draw=drawColor,line width= 0.4pt,line join=round,line cap=round] ( 64.18,278.95) circle (  2.25);

\path[draw=drawColor,line width= 0.4pt,line join=round,line cap=round] ( 64.28,278.94) circle (  2.25);

\path[draw=drawColor,line width= 0.4pt,line join=round,line cap=round] ( 64.43,278.92) circle (  2.25);

\path[draw=drawColor,line width= 0.4pt,line join=round,line cap=round] ( 64.66,278.88) circle (  2.25);

\path[draw=drawColor,line width= 0.4pt,line join=round,line cap=round] ( 65.06,278.80) circle (  2.25);

\path[draw=drawColor,line width= 0.4pt,line join=round,line cap=round] ( 65.57,278.65) circle (  2.25);

\path[draw=drawColor,line width= 0.4pt,line join=round,line cap=round] ( 66.09,278.46) circle (  2.25);

\path[draw=drawColor,line width= 0.4pt,line join=round,line cap=round] ( 66.61,278.23) circle (  2.25);

\path[draw=drawColor,line width= 0.4pt,line join=round,line cap=round] ( 67.13,277.95) circle (  2.25);

\path[draw=drawColor,line width= 0.4pt,line join=round,line cap=round] ( 67.65,277.63) circle (  2.25);

\path[draw=drawColor,line width= 0.4pt,line join=round,line cap=round] ( 68.17,277.27) circle (  2.25);

\path[draw=drawColor,line width= 0.4pt,line join=round,line cap=round] ( 68.69,276.86) circle (  2.25);

\path[draw=drawColor,line width= 0.4pt,line join=round,line cap=round] ( 69.21,276.41) circle (  2.25);

\path[draw=drawColor,line width= 0.4pt,line join=round,line cap=round] ( 69.73,275.92) circle (  2.25);

\path[draw=drawColor,line width= 0.4pt,line join=round,line cap=round] ( 70.38,275.24) circle (  2.25);

\path[draw=drawColor,line width= 0.4pt,line join=round,line cap=round] ( 71.38,274.08) circle (  2.25);

\path[draw=drawColor,line width= 0.4pt,line join=round,line cap=round] ( 73.59,270.94) circle (  2.25);

\path[draw=drawColor,line width= 0.4pt,line join=round,line cap=round] ( 75.80,267.09) circle (  2.25);

\path[draw=drawColor,line width= 0.4pt,line join=round,line cap=round] ( 78.00,262.55) circle (  2.25);

\path[draw=drawColor,line width= 0.4pt,line join=round,line cap=round] ( 80.21,257.35) circle (  2.25);

\path[draw=drawColor,line width= 0.4pt,line join=round,line cap=round] ( 82.41,251.52) circle (  2.25);

\path[draw=drawColor,line width= 0.4pt,line join=round,line cap=round] ( 84.62,245.13) circle (  2.25);

\path[draw=drawColor,line width= 0.4pt,line join=round,line cap=round] ( 86.83,238.21) circle (  2.25);

\path[draw=drawColor,line width= 0.4pt,line join=round,line cap=round] ( 89.03,230.82) circle (  2.25);

\path[draw=drawColor,line width= 0.4pt,line join=round,line cap=round] ( 91.24,223.03) circle (  2.25);

\path[draw=drawColor,line width= 0.4pt,line join=round,line cap=round] ( 94.18,212.13) circle (  2.25);

\path[draw=drawColor,line width= 0.4pt,line join=round,line cap=round] ( 96.82,201.93) circle (  2.25);

\path[draw=drawColor,line width= 0.4pt,line join=round,line cap=round] ( 99.46,191.49) circle (  2.25);

\path[draw=drawColor,line width= 0.4pt,line join=round,line cap=round] (102.10,180.90) circle (  2.25);

\path[draw=drawColor,line width= 0.4pt,line join=round,line cap=round] (104.74,170.30) circle (  2.25);

\path[draw=drawColor,line width= 0.4pt,line join=round,line cap=round] (107.51,159.28) circle (  2.25);

\path[draw=drawColor,line width= 0.4pt,line join=round,line cap=round] (111.06,145.54) circle (  2.25);

\path[draw=drawColor,line width= 0.4pt,line join=round,line cap=round] (116.27,126.73) circle (  2.25);

\path[draw=drawColor,line width= 0.4pt,line join=round,line cap=round] (121.09,111.39) circle (  2.25);

\path[draw=drawColor,line width= 0.4pt,line join=round,line cap=round] (125.91, 98.62) circle (  2.25);

\path[draw=drawColor,line width= 0.4pt,line join=round,line cap=round] (130.73, 88.91) circle (  2.25);

\path[draw=drawColor,line width= 0.4pt,line join=round,line cap=round] (135.55, 82.62) circle (  2.25);

\path[draw=drawColor,line width= 0.4pt,line join=round,line cap=round] (140.37, 79.99) circle (  2.25);

\path[draw=drawColor,line width= 0.4pt,line join=round,line cap=round] (145.19, 81.12) circle (  2.25);

\path[draw=drawColor,line width= 0.4pt,line join=round,line cap=round] (150.01, 85.96) circle (  2.25);

\path[draw=drawColor,line width= 0.4pt,line join=round,line cap=round] (155.12, 94.95) circle (  2.25);

\path[draw=drawColor,line width= 0.4pt,line join=round,line cap=round] (160.23,107.53) circle (  2.25);

\path[draw=drawColor,line width= 0.4pt,line join=round,line cap=round] (165.34,123.17) circle (  2.25);

\path[draw=drawColor,line width= 0.4pt,line join=round,line cap=round] (170.45,141.19) circle (  2.25);

\path[draw=drawColor,line width= 0.4pt,line join=round,line cap=round] (174.75,157.64) circle (  2.25);

\path[draw=drawColor,line width= 0.4pt,line join=round,line cap=round] (179.05,174.74) circle (  2.25);

\path[draw=drawColor,line width= 0.4pt,line join=round,line cap=round] (183.41,192.23) circle (  2.25);

\path[draw=drawColor,line width= 0.4pt,line join=round,line cap=round] (188.93,213.76) circle (  2.25);

\path[draw=drawColor,line width= 0.4pt,line join=round,line cap=round] (194.08,232.30) circle (  2.25);

\path[draw=drawColor,line width= 0.4pt,line join=round,line cap=round] (199.22,248.57) circle (  2.25);

\path[draw=drawColor,line width= 0.4pt,line join=round,line cap=round] (204.37,261.85) circle (  2.25);

\path[draw=drawColor,line width= 0.4pt,line join=round,line cap=round] (209.51,271.59) circle (  2.25);

\path[draw=drawColor,line width= 0.4pt,line join=round,line cap=round] (214.04,276.89) circle (  2.25);

\path[draw=drawColor,line width= 0.4pt,line join=round,line cap=round] (218.57,278.94) circle (  2.25);

\path[draw=drawColor,line width= 0.4pt,line join=round,line cap=round] (223.10,277.67) circle (  2.25);

\path[draw=drawColor,line width= 0.4pt,line join=round,line cap=round] (228.36,272.09) circle (  2.25);

\path[draw=drawColor,line width= 0.4pt,line join=round,line cap=round] (233.87,261.77) circle (  2.25);

\path[draw=drawColor,line width= 0.4pt,line join=round,line cap=round] (238.49,249.97) circle (  2.25);

\path[draw=drawColor,line width= 0.4pt,line join=round,line cap=round] (243.11,235.72) circle (  2.25);

\path[draw=drawColor,line width= 0.4pt,line join=round,line cap=round] (247.88,218.97) circle (  2.25);

\path[draw=drawColor,line width= 0.4pt,line join=round,line cap=round] (253.44,197.61) circle (  2.25);

\path[draw=drawColor,line width= 0.4pt,line join=round,line cap=round] (256.52,185.33) circle (  2.25);

\path[draw=drawColor,line width= 0.4pt,line join=round,line cap=round] (259.60,172.96) circle (  2.25);

\path[draw=drawColor,line width= 0.4pt,line join=round,line cap=round] (263.48,157.52) circle (  2.25);

\path[draw=drawColor,line width= 0.4pt,line join=round,line cap=round] (267.90,140.63) circle (  2.25);

\path[draw=drawColor,line width= 0.4pt,line join=round,line cap=round] (273.64,120.63) circle (  2.25);

\path[draw=drawColor,line width= 0.4pt,line join=round,line cap=round] (278.18,106.98) circle (  2.25);

\path[draw=drawColor,line width= 0.4pt,line join=round,line cap=round] (282.71, 95.77) circle (  2.25);

\path[draw=drawColor,line width= 0.4pt,line join=round,line cap=round] (287.25, 87.36) circle (  2.25);

\path[draw=drawColor,line width= 0.4pt,line join=round,line cap=round] (292.62, 81.41) circle (  2.25);

\path[draw=drawColor,line width= 0.4pt,line join=round,line cap=round] (296.37, 79.96) circle (  2.25);

\path[draw=drawColor,line width= 0.4pt,line join=round,line cap=round] (300.13, 80.80) circle (  2.25);

\path[draw=drawColor,line width= 0.4pt,line join=round,line cap=round] (304.82, 85.04) circle (  2.25);

\path[draw=drawColor,line width= 0.4pt,line join=round,line cap=round] (310.77, 95.23) circle (  2.25);

\path[draw=drawColor,line width= 0.4pt,line join=round,line cap=round] (315.47,106.75) circle (  2.25);

\path[draw=drawColor,line width= 0.4pt,line join=round,line cap=round] (320.17,120.90) circle (  2.25);

\path[draw=drawColor,line width= 0.4pt,line join=round,line cap=round] (324.88,137.14) circle (  2.25);

\path[draw=drawColor,line width= 0.4pt,line join=round,line cap=round] (330.10,156.95) circle (  2.25);

\path[draw=drawColor,line width= 0.4pt,line join=round,line cap=round] (333.80,171.62) circle (  2.25);

\path[draw=drawColor,line width= 0.4pt,line join=round,line cap=round] (337.50,186.47) circle (  2.25);

\path[draw=drawColor,line width= 0.4pt,line join=round,line cap=round] (341.87,203.81) circle (  2.25);

\path[draw=drawColor,line width= 0.4pt,line join=round,line cap=round] (347.41,224.60) circle (  2.25);

\path[draw=drawColor,line width= 0.4pt,line join=round,line cap=round] (352.43,241.55) circle (  2.25);

\path[draw=drawColor,line width= 0.4pt,line join=round,line cap=round] (357.46,255.95) circle (  2.25);

\path[draw=drawColor,line width= 0.4pt,line join=round,line cap=round] (362.48,267.20) circle (  2.25);

\path[draw=drawColor,line width= 0.4pt,line join=round,line cap=round] (367.51,274.85) circle (  2.25);

\path[draw=drawColor,line width= 0.4pt,line join=round,line cap=round] (371.90,278.34) circle (  2.25);

\path[draw=drawColor,line width= 0.4pt,line join=round,line cap=round] (376.30,278.71) circle (  2.25);

\path[draw=drawColor,line width= 0.4pt,line join=round,line cap=round] (380.79,275.88) circle (  2.25);

\path[draw=drawColor,line width= 0.4pt,line join=round,line cap=round] (386.20,268.26) circle (  2.25);

\path[draw=drawColor,line width= 0.4pt,line join=round,line cap=round] (391.77,256.01) circle (  2.25);

\path[draw=drawColor,line width= 0.4pt,line join=round,line cap=round] (396.26,243.30) circle (  2.25);

\path[draw=drawColor,line width= 0.4pt,line join=round,line cap=round] (400.75,228.51) circle (  2.25);

\path[draw=drawColor,line width= 0.4pt,line join=round,line cap=round] (405.57,210.80) circle (  2.25);

\path[draw=drawColor,line width= 0.4pt,line join=round,line cap=round] (410.40,191.90) circle (  2.25);

\path[draw=drawColor,line width= 0.4pt,line join=round,line cap=round] (413.79,178.33) circle (  2.25);

\path[draw=drawColor,line width= 0.4pt,line join=round,line cap=round] (417.17,164.78) circle (  2.25);

\path[draw=drawColor,line width= 0.4pt,line join=round,line cap=round] (422.95,142.41) circle (  2.25);

\path[draw=drawColor,line width= 0.4pt,dash pattern=on 1pt off 3pt ,line join=round,line cap=round] ( 63.58,179.45) --
	( 63.66,179.25) --
	( 63.76,179.00) --
	( 63.86,178.74) --
	( 63.97,178.49) --
	( 64.07,178.23) --
	( 64.18,177.98) --
	( 64.28,177.72) --
	( 64.43,177.37) --
	( 64.66,176.80) --
	( 65.06,175.84) --
	( 65.57,174.58) --
	( 66.09,173.32) --
	( 66.61,172.06) --
	( 67.13,170.80) --
	( 67.65,169.55) --
	( 68.17,168.31) --
	( 68.69,167.06) --
	( 69.21,165.83) --
	( 69.73,164.60) --
	( 70.38,163.06) --
	( 71.38,160.74) --
	( 73.59,155.69) --
	( 75.80,150.84) --
	( 78.00,146.22) --
	( 80.21,141.87) --
	( 82.41,137.81) --
	( 84.62,134.08) --
	( 86.83,130.72) --
	( 89.03,127.74) --
	( 91.24,125.17) --
	( 94.18,122.42) --
	( 96.82,120.63) --
	( 99.46,119.52) --
	(102.10,119.08) --
	(104.74,119.33) --
	(107.51,120.33) --
	(111.06,122.69) --
	(116.27,128.24) --
	(121.09,135.39) --
	(125.91,144.21) --
	(130.73,154.36) --
	(135.55,165.46) --
	(140.37,177.09) --
	(145.19,188.81) --
	(150.01,200.17) --
	(155.12,211.35) --
	(160.23,221.18) --
	(165.34,229.24) --
	(170.45,235.18) --
	(174.75,238.35) --
	(179.05,239.75) --
	(183.41,239.31) --
	(188.93,236.11) --
	(194.08,230.60) --
	(199.22,222.88) --
	(204.37,213.29) --
	(209.51,202.25) --
	(214.04,191.70) --
	(218.57,180.75) --
	(223.10,169.75) --
	(228.36,157.40) --
	(233.87,145.54) --
	(238.49,136.86) --
	(243.11,129.66) --
	(247.88,124.05) --
	(253.44,120.10) --
	(256.52,119.20) --
	(259.60,119.22) --
	(263.48,120.58) --
	(267.90,123.87) --
	(273.64,130.77) --
	(278.18,138.08) --
	(282.71,146.79) --
	(287.25,156.59) --
	(292.62,169.13) --
	(296.37,178.24) --
	(300.13,187.37) --
	(304.82,198.51) --
	(310.77,211.59) --
	(315.47,220.66) --
	(320.17,228.24) --
	(324.88,234.07) --
	(330.10,238.23) --
	(333.80,239.61) --
	(337.50,239.65) --
	(341.87,237.96) --
	(347.41,233.22) --
	(352.43,226.60) --
	(357.46,218.04) --
	(362.48,207.89) --
	(367.51,196.58) --
	(371.90,186.08) --
	(376.30,175.38) --
	(380.79,164.59) --
	(386.20,152.25) --
	(391.77,140.91) --
	(396.26,133.18) --
	(400.75,126.95) --
	(405.57,122.18) --
	(410.40,119.58) --
	(413.79,119.11) --
	(417.17,119.77) --
	(422.95,123.44);
\definecolor{fillColor}{RGB}{255,255,255}

\path[draw=drawColor,line width= 0.4pt,line join=round,line cap=round,fill=fillColor] (334.80,297.70) rectangle (437.33,249.70);

\path[draw=drawColor,line width= 0.4pt,line join=round,line cap=round] (337.50,285.70) -- (355.50,285.70);

\path[draw=drawColor,line width= 0.4pt,dash pattern=on 1pt off 3pt ,line join=round,line cap=round] (337.50,261.70) -- (355.50,261.70);

\path[draw=drawColor,line width= 0.4pt,line join=round,line cap=round] (346.50,273.70) circle (  2.25);

\node[text=drawColor,anchor=base west,inner sep=0pt, outer sep=0pt, scale=  1.00] at (364.50,282.25) {$y$ exact};

\node[text=drawColor,anchor=base west,inner sep=0pt, outer sep=0pt, scale=  1.00] at (364.50,270.25) {gsl odeiv $y(t;k)$};

\node[text=drawColor,anchor=base west,inner sep=0pt, outer sep=0pt, scale=  1.00] at (364.50,258.25) {gsl $v(t;k)$};
\end{scope}
\end{tikzpicture}

  \caption{Comparison between \emph{numerical integration} and the
    \emph{analytical trajectory} solution. A perfect match within
    visual precision.\label{fig:dampedGSLvsASolTraj}}
\end{figure}

Figure~\ref{fig:dampedlinerr} shows the difference between a
prediction using the estimated sensitivity, and analytical solution
for a small shift in $k$, as motivated in the previous Section. But,
rather than a second simulation we use the analytical solution for
$y$.

\begin{figure}\sffamily\firalining
  \centering
  % Created by tikzDevice version 0.12.3.1 on 2021-06-16 15:27:58
% !TEX encoding = UTF-8 Unicode
\begin{tikzpicture}[x=1pt,y=1pt]
\definecolor{fillColor}{RGB}{255,255,255}
\path[use as bounding box,fill=fillColor,fill opacity=0.00] (0,0) rectangle (462.53,346.90);
\begin{scope}
\path[clip] ( 49.20, 61.20) rectangle (437.33,297.70);
\definecolor{drawColor}{RGB}{0,0,0}

\path[draw=drawColor,line width= 0.4pt,line join=round,line cap=round] ( 63.58,288.94) --
	( 63.66,288.94) --
	( 63.76,288.93) --
	( 63.86,288.93) --
	( 63.97,288.92) --
	( 64.07,288.91) --
	( 64.18,288.90) --
	( 64.28,288.89) --
	( 64.43,288.86) --
	( 64.66,288.82) --
	( 65.06,288.71) --
	( 65.57,288.53) --
	( 66.09,288.29) --
	( 66.61,288.00) --
	( 67.13,287.66) --
	( 67.65,287.25) --
	( 68.17,286.80) --
	( 68.69,286.29) --
	( 69.21,285.73) --
	( 69.73,285.12) --
	( 70.38,284.27) --
	( 71.38,282.82) --
	( 73.59,278.91) --
	( 75.80,274.13) --
	( 78.00,268.48) --
	( 80.21,262.04) --
	( 82.41,254.85) --
	( 84.62,246.98) --
	( 86.83,238.50) --
	( 89.03,229.48) --
	( 91.24,220.02) --
	( 94.18,206.87) --
	( 96.82,194.67) --
	( 99.46,182.27) --
	(102.10,169.84) --
	(104.74,157.53) --
	(107.51,144.91) --
	(111.06,129.48) --
	(116.27,109.06) --
	(121.09, 93.29) --
	(125.91, 81.21) --
	(130.73, 73.36) --
	(135.55, 70.05) --
	(140.37, 71.44) --
	(145.19, 77.46) --
	(150.01, 87.86) --
	(155.12,103.17) --
	(160.23,122.16) --
	(165.34,143.91) --
	(170.45,167.37) --
	(174.75,187.60) --
	(179.05,207.55) --
	(183.41,226.80) --
	(188.93,248.73) --
	(194.08,265.67) --
	(199.22,278.39) --
	(204.37,286.27) --
	(209.51,288.94) --
	(214.04,286.85) --
	(218.57,280.70) --
	(223.10,270.71) --
	(228.36,254.80) --
	(233.87,234.01) --
	(238.49,214.18) --
	(243.11,192.97) --
	(247.88,170.55) --
	(253.44,144.90) --
	(256.52,131.47) --
	(259.60,118.88) --
	(263.48,104.54) --
	(267.90, 90.78) --
	(273.64, 77.77) --
	(278.18, 71.82) --
	(282.71, 69.96) --
	(287.25, 72.27) --
	(292.62, 80.24) --
	(296.37, 88.99) --
	(300.13,100.10) --
	(304.82,116.87) --
	(310.77,141.67) --
	(315.47,163.11) --
	(320.17,185.21) --
	(324.88,207.09) --
	(330.10,230.02) --
	(333.80,244.78) --
	(337.50,257.88) --
	(341.87,270.79) --
	(347.41,282.47) --
	(352.43,288.03) --
	(357.46,288.52) --
	(362.48,283.93) --
	(367.51,274.47) --
	(371.90,262.54) --
	(376.30,247.64) --
	(380.79,229.92) --
	(386.20,206.09) --
	(391.77,180.10) --
	(396.26,159.07) --
	(400.75,138.80) --
	(405.57,118.69) --
	(410.40,101.19) --
	(413.79, 90.90) --
	(417.17, 82.49) --
	(422.95, 72.95);
\end{scope}
\begin{scope}
\path[clip] (  0.00,  0.00) rectangle (462.53,346.90);
\definecolor{drawColor}{RGB}{0,0,0}

\path[draw=drawColor,line width= 0.4pt,line join=round,line cap=round] ( 63.58, 61.20) -- (363.85, 61.20);

\path[draw=drawColor,line width= 0.4pt,line join=round,line cap=round] ( 63.58, 61.20) -- ( 63.58, 55.20);

\path[draw=drawColor,line width= 0.4pt,line join=round,line cap=round] (138.64, 61.20) -- (138.64, 55.20);

\path[draw=drawColor,line width= 0.4pt,line join=round,line cap=round] (213.71, 61.20) -- (213.71, 55.20);

\path[draw=drawColor,line width= 0.4pt,line join=round,line cap=round] (288.78, 61.20) -- (288.78, 55.20);

\path[draw=drawColor,line width= 0.4pt,line join=round,line cap=round] (363.85, 61.20) -- (363.85, 55.20);

\node[text=drawColor,anchor=base,inner sep=0pt, outer sep=0pt, scale=  1.00] at ( 63.58, 39.60) {0};

\node[text=drawColor,anchor=base,inner sep=0pt, outer sep=0pt, scale=  1.00] at (138.64, 39.60) {5};

\node[text=drawColor,anchor=base,inner sep=0pt, outer sep=0pt, scale=  1.00] at (213.71, 39.60) {10};

\node[text=drawColor,anchor=base,inner sep=0pt, outer sep=0pt, scale=  1.00] at (288.78, 39.60) {15};

\node[text=drawColor,anchor=base,inner sep=0pt, outer sep=0pt, scale=  1.00] at (363.85, 39.60) {20};

\path[draw=drawColor,line width= 0.4pt,line join=round,line cap=round] ( 49.20, 69.95) -- ( 49.20,288.94);

\path[draw=drawColor,line width= 0.4pt,line join=round,line cap=round] ( 49.20, 69.95) -- ( 43.20, 69.95);

\path[draw=drawColor,line width= 0.4pt,line join=round,line cap=round] ( 49.20,124.70) -- ( 43.20,124.70);

\path[draw=drawColor,line width= 0.4pt,line join=round,line cap=round] ( 49.20,179.44) -- ( 43.20,179.44);

\path[draw=drawColor,line width= 0.4pt,line join=round,line cap=round] ( 49.20,234.19) -- ( 43.20,234.19);

\path[draw=drawColor,line width= 0.4pt,line join=round,line cap=round] ( 49.20,288.94) -- ( 43.20,288.94);

\node[text=drawColor,rotate= 90.00,anchor=base,inner sep=0pt, outer sep=0pt, scale=  1.00] at ( 34.80, 69.95) {-1.0};

\node[text=drawColor,rotate= 90.00,anchor=base,inner sep=0pt, outer sep=0pt, scale=  1.00] at ( 34.80,124.70) {-0.5};

\node[text=drawColor,rotate= 90.00,anchor=base,inner sep=0pt, outer sep=0pt, scale=  1.00] at ( 34.80,179.44) {0.0};

\node[text=drawColor,rotate= 90.00,anchor=base,inner sep=0pt, outer sep=0pt, scale=  1.00] at ( 34.80,234.19) {0.5};

\node[text=drawColor,rotate= 90.00,anchor=base,inner sep=0pt, outer sep=0pt, scale=  1.00] at ( 34.80,288.94) {1.0};

\path[draw=drawColor,line width= 0.4pt,line join=round,line cap=round] ( 49.20, 61.20) --
	(437.33, 61.20) --
	(437.33,297.70) --
	( 49.20,297.70) --
	( 49.20, 61.20);
\end{scope}
\begin{scope}
\path[clip] (  0.00,  0.00) rectangle (462.53,346.90);
\definecolor{drawColor}{RGB}{0,0,0}

\node[text=drawColor,anchor=base,inner sep=0pt, outer sep=0pt, scale=  1.20] at (243.26,318.16) {\bfseries  no damping no driving force at $\Delta_k = 0.05$};

\node[text=drawColor,anchor=base,inner sep=0pt, outer sep=0pt, scale=  1.00] at (243.26,  3.60) {average missmatch: $1.4 \Delta_k$};

\node[text=drawColor,anchor=base,inner sep=0pt, outer sep=0pt, scale=  1.00] at (243.26, 15.60) {x};

\node[text=drawColor,rotate= 90.00,anchor=base,inner sep=0pt, outer sep=0pt, scale=  1.00] at ( 10.80,179.45) {y};
\end{scope}
\begin{scope}
\path[clip] ( 49.20, 61.20) rectangle (437.33,297.70);
\definecolor{drawColor}{RGB}{0,0,0}

\path[draw=drawColor,line width= 0.4pt,line join=round,line cap=round] ( 63.58,288.94) circle (  2.25);

\path[draw=drawColor,line width= 0.4pt,line join=round,line cap=round] ( 63.66,288.94) circle (  2.25);

\path[draw=drawColor,line width= 0.4pt,line join=round,line cap=round] ( 63.76,288.93) circle (  2.25);

\path[draw=drawColor,line width= 0.4pt,line join=round,line cap=round] ( 63.86,288.93) circle (  2.25);

\path[draw=drawColor,line width= 0.4pt,line join=round,line cap=round] ( 63.97,288.92) circle (  2.25);

\path[draw=drawColor,line width= 0.4pt,line join=round,line cap=round] ( 64.07,288.91) circle (  2.25);

\path[draw=drawColor,line width= 0.4pt,line join=round,line cap=round] ( 64.18,288.89) circle (  2.25);

\path[draw=drawColor,line width= 0.4pt,line join=round,line cap=round] ( 64.28,288.88) circle (  2.25);

\path[draw=drawColor,line width= 0.4pt,line join=round,line cap=round] ( 64.43,288.86) circle (  2.25);

\path[draw=drawColor,line width= 0.4pt,line join=round,line cap=round] ( 64.66,288.81) circle (  2.25);

\path[draw=drawColor,line width= 0.4pt,line join=round,line cap=round] ( 65.06,288.71) circle (  2.25);

\path[draw=drawColor,line width= 0.4pt,line join=round,line cap=round] ( 65.57,288.53) circle (  2.25);

\path[draw=drawColor,line width= 0.4pt,line join=round,line cap=round] ( 66.09,288.29) circle (  2.25);

\path[draw=drawColor,line width= 0.4pt,line join=round,line cap=round] ( 66.61,288.00) circle (  2.25);

\path[draw=drawColor,line width= 0.4pt,line join=round,line cap=round] ( 67.13,287.65) circle (  2.25);

\path[draw=drawColor,line width= 0.4pt,line join=round,line cap=round] ( 67.65,287.25) circle (  2.25);

\path[draw=drawColor,line width= 0.4pt,line join=round,line cap=round] ( 68.17,286.79) circle (  2.25);

\path[draw=drawColor,line width= 0.4pt,line join=round,line cap=round] ( 68.69,286.29) circle (  2.25);

\path[draw=drawColor,line width= 0.4pt,line join=round,line cap=round] ( 69.21,285.73) circle (  2.25);

\path[draw=drawColor,line width= 0.4pt,line join=round,line cap=round] ( 69.73,285.11) circle (  2.25);

\path[draw=drawColor,line width= 0.4pt,line join=round,line cap=round] ( 70.38,284.26) circle (  2.25);

\path[draw=drawColor,line width= 0.4pt,line join=round,line cap=round] ( 71.38,282.81) circle (  2.25);

\path[draw=drawColor,line width= 0.4pt,line join=round,line cap=round] ( 73.59,278.91) circle (  2.25);

\path[draw=drawColor,line width= 0.4pt,line join=round,line cap=round] ( 75.80,274.12) circle (  2.25);

\path[draw=drawColor,line width= 0.4pt,line join=round,line cap=round] ( 78.00,268.47) circle (  2.25);

\path[draw=drawColor,line width= 0.4pt,line join=round,line cap=round] ( 80.21,262.02) circle (  2.25);

\path[draw=drawColor,line width= 0.4pt,line join=round,line cap=round] ( 82.41,254.82) circle (  2.25);

\path[draw=drawColor,line width= 0.4pt,line join=round,line cap=round] ( 84.62,246.94) circle (  2.25);

\path[draw=drawColor,line width= 0.4pt,line join=round,line cap=round] ( 86.83,238.44) circle (  2.25);

\path[draw=drawColor,line width= 0.4pt,line join=round,line cap=round] ( 89.03,229.41) circle (  2.25);

\path[draw=drawColor,line width= 0.4pt,line join=round,line cap=round] ( 91.24,219.91) circle (  2.25);

\path[draw=drawColor,line width= 0.4pt,line join=round,line cap=round] ( 94.18,206.70) circle (  2.25);

\path[draw=drawColor,line width= 0.4pt,line join=round,line cap=round] ( 96.82,194.44) circle (  2.25);

\path[draw=drawColor,line width= 0.4pt,line join=round,line cap=round] ( 99.46,181.98) circle (  2.25);

\path[draw=drawColor,line width= 0.4pt,line join=round,line cap=round] (102.10,169.46) circle (  2.25);

\path[draw=drawColor,line width= 0.4pt,line join=round,line cap=round] (104.74,157.05) circle (  2.25);

\path[draw=drawColor,line width= 0.4pt,line join=round,line cap=round] (107.51,144.31) circle (  2.25);

\path[draw=drawColor,line width= 0.4pt,line join=round,line cap=round] (111.06,128.71) circle (  2.25);

\path[draw=drawColor,line width= 0.4pt,line join=round,line cap=round] (116.27,107.98) circle (  2.25);

\path[draw=drawColor,line width= 0.4pt,line join=round,line cap=round] (121.09, 91.91) circle (  2.25);

\path[draw=drawColor,line width= 0.4pt,line join=round,line cap=round] (125.91, 79.52) circle (  2.25);

\path[draw=drawColor,line width= 0.4pt,line join=round,line cap=round] (130.73, 71.36) circle (  2.25);

\path[draw=drawColor,line width= 0.4pt,line join=round,line cap=round] (135.55, 67.79) circle (  2.25);

\path[draw=drawColor,line width= 0.4pt,line join=round,line cap=round] (140.37, 68.99) circle (  2.25);

\path[draw=drawColor,line width= 0.4pt,line join=round,line cap=round] (145.19, 74.92) circle (  2.25);

\path[draw=drawColor,line width= 0.4pt,line join=round,line cap=round] (150.01, 85.35) circle (  2.25);

\path[draw=drawColor,line width= 0.4pt,line join=round,line cap=round] (155.12,100.85) circle (  2.25);

\path[draw=drawColor,line width= 0.4pt,line join=round,line cap=round] (160.23,120.24) circle (  2.25);

\path[draw=drawColor,line width= 0.4pt,line join=round,line cap=round] (165.34,142.59) circle (  2.25);

\path[draw=drawColor,line width= 0.4pt,line join=round,line cap=round] (170.45,166.88) circle (  2.25);

\path[draw=drawColor,line width= 0.4pt,line join=round,line cap=round] (174.75,187.95) circle (  2.25);

\path[draw=drawColor,line width= 0.4pt,line join=round,line cap=round] (179.05,208.87) circle (  2.25);

\path[draw=drawColor,line width= 0.4pt,line join=round,line cap=round] (183.41,229.21) circle (  2.25);

\path[draw=drawColor,line width= 0.4pt,line join=round,line cap=round] (188.93,252.63) circle (  2.25);

\path[draw=drawColor,line width= 0.4pt,line join=round,line cap=round] (194.08,270.98) circle (  2.25);

\path[draw=drawColor,line width= 0.4pt,line join=round,line cap=round] (199.22,285.06) circle (  2.25);

\path[draw=drawColor,line width= 0.4pt,line join=round,line cap=round] (204.37,294.16) circle (  2.25);

\path[draw=drawColor,line width= 0.4pt,line join=round,line cap=round] (209.51,297.80) circle (  2.25);

\path[draw=drawColor,line width= 0.4pt,line join=round,line cap=round] (214.04,296.30) circle (  2.25);

\path[draw=drawColor,line width= 0.4pt,line join=round,line cap=round] (218.57,290.42) circle (  2.25);

\path[draw=drawColor,line width= 0.4pt,line join=round,line cap=round] (223.10,280.35) circle (  2.25);

\path[draw=drawColor,line width= 0.4pt,line join=round,line cap=round] (228.36,263.84) circle (  2.25);

\path[draw=drawColor,line width= 0.4pt,line join=round,line cap=round] (233.87,241.82) circle (  2.25);

\path[draw=drawColor,line width= 0.4pt,line join=round,line cap=round] (238.49,220.46) circle (  2.25);

\path[draw=drawColor,line width= 0.4pt,line join=round,line cap=round] (243.11,197.30) circle (  2.25);

\path[draw=drawColor,line width= 0.4pt,line join=round,line cap=round] (247.88,172.45) circle (  2.25);

\path[draw=drawColor,line width= 0.4pt,line join=round,line cap=round] (253.44,143.54) circle (  2.25);

\path[draw=drawColor,line width= 0.4pt,line join=round,line cap=round] (256.52,128.17) circle (  2.25);

\path[draw=drawColor,line width= 0.4pt,line join=round,line cap=round] (259.60,113.57) circle (  2.25);

\path[draw=drawColor,line width= 0.4pt,line join=round,line cap=round] (263.48, 96.64) circle (  2.25);

\path[draw=drawColor,line width= 0.4pt,line join=round,line cap=round] (267.90, 79.97) circle (  2.25);

\path[draw=drawColor,line width= 0.4pt,line join=round,line cap=round] (273.64, 63.36) circle (  2.25);

\path[draw=drawColor,line width= 0.4pt,line join=round,line cap=round] (278.18, 54.91) circle (  2.25);

\path[draw=drawColor,line width= 0.4pt,line join=round,line cap=round] (282.71, 50.99) circle (  2.25);

\path[draw=drawColor,line width= 0.4pt,line join=round,line cap=round] (287.25, 51.78) circle (  2.25);

\path[draw=drawColor,line width= 0.4pt,line join=round,line cap=round] (292.62, 58.79) circle (  2.25);

\path[draw=drawColor,line width= 0.4pt,line join=round,line cap=round] (296.37, 67.51) circle (  2.25);

\path[draw=drawColor,line width= 0.4pt,line join=round,line cap=round] (300.13, 79.14) circle (  2.25);

\path[draw=drawColor,line width= 0.4pt,line join=round,line cap=round] (304.82, 97.36) circle (  2.25);

\path[draw=drawColor,line width= 0.4pt,line join=round,line cap=round] (310.77,125.24) circle (  2.25);

\path[draw=drawColor,line width= 0.4pt,line join=round,line cap=round] (315.47,150.14) circle (  2.25);

\path[draw=drawColor,line width= 0.4pt,line join=round,line cap=round] (320.17,176.49) circle (  2.25);

\path[draw=drawColor,line width= 0.4pt,line join=round,line cap=round] (324.88,203.27) circle (  2.25);

\path[draw=drawColor,line width= 0.4pt,line join=round,line cap=round] (330.10,232.25) circle (  2.25);

\path[draw=drawColor,line width= 0.4pt,line join=round,line cap=round] (333.80,251.53) circle (  2.25);

\path[draw=drawColor,line width= 0.4pt,line join=round,line cap=round] (337.50,269.23) circle (  2.25);

\path[draw=drawColor,line width= 0.4pt,line join=round,line cap=round] (341.87,287.55) circle (  2.25);

\path[draw=drawColor,line width= 0.4pt,line join=round,line cap=round] (347.41,305.71) circle (  2.25);

\path[draw=drawColor,line width= 0.4pt,line join=round,line cap=round] (352.43,316.50) circle (  2.25);

\path[draw=drawColor,line width= 0.4pt,line join=round,line cap=round] (357.46,321.30) circle (  2.25);

\path[draw=drawColor,line width= 0.4pt,line join=round,line cap=round] (362.48,319.82) circle (  2.25);

\path[draw=drawColor,line width= 0.4pt,line join=round,line cap=round] (367.51,312.06) circle (  2.25);

\path[draw=drawColor,line width= 0.4pt,line join=round,line cap=round] (371.90,300.30) circle (  2.25);

\path[draw=drawColor,line width= 0.4pt,line join=round,line cap=round] (376.30,284.27) circle (  2.25);

\path[draw=drawColor,line width= 0.4pt,line join=round,line cap=round] (380.79,264.04) circle (  2.25);

\path[draw=drawColor,line width= 0.4pt,line join=round,line cap=round] (386.20,235.39) circle (  2.25);

\path[draw=drawColor,line width= 0.4pt,line join=round,line cap=round] (391.77,202.48) circle (  2.25);

\path[draw=drawColor,line width= 0.4pt,line join=round,line cap=round] (396.26,174.61) circle (  2.25);

\path[draw=drawColor,line width= 0.4pt,line join=round,line cap=round] (400.75,146.57) circle (  2.25);

\path[draw=drawColor,line width= 0.4pt,line join=round,line cap=round] (405.57,117.34) circle (  2.25);

\path[draw=drawColor,line width= 0.4pt,line join=round,line cap=round] (410.40, 90.25) circle (  2.25);

\path[draw=drawColor,line width= 0.4pt,line join=round,line cap=round] (413.79, 73.17) circle (  2.25);

\path[draw=drawColor,line width= 0.4pt,line join=round,line cap=round] (417.17, 58.07) circle (  2.25);

\path[draw=drawColor,line width= 0.4pt,line join=round,line cap=round] (422.95, 37.72) circle (  2.25);

\path[draw=drawColor,line width= 0.4pt,dash pattern=on 1pt off 3pt ,line join=round,line cap=round] ( 63.58,288.94) --
	( 63.66,288.94) --
	( 63.76,288.93) --
	( 63.86,288.93) --
	( 63.97,288.92) --
	( 64.07,288.91) --
	( 64.18,288.90) --
	( 64.28,288.89) --
	( 64.43,288.87) --
	( 64.66,288.83) --
	( 65.06,288.74) --
	( 65.57,288.58) --
	( 66.09,288.37) --
	( 66.61,288.11) --
	( 67.13,287.81) --
	( 67.65,287.46) --
	( 68.17,287.06) --
	( 68.69,286.61) --
	( 69.21,286.11) --
	( 69.73,285.57) --
	( 70.38,284.82) --
	( 71.38,283.54) --
	( 73.59,280.10) --
	( 75.80,275.86) --
	( 78.00,270.86) --
	( 80.21,265.13) --
	( 82.41,258.73) --
	( 84.62,251.69) --
	( 86.83,244.08) --
	( 89.03,235.95) --
	( 91.24,227.38) --
	( 94.18,215.39) --
	( 96.82,204.18) --
	( 99.46,192.68) --
	(102.10,181.04) --
	(104.74,169.38) --
	(107.51,157.25) --
	(111.06,142.14) --
	(116.27,121.45) --
	(121.09,104.58) --
	(125.91, 90.54) --
	(130.73, 79.85) --
	(135.55, 72.93) --
	(140.37, 70.04) --
	(145.19, 71.28) --
	(150.01, 76.60) --
	(155.12, 86.49) --
	(160.23,100.33) --
	(165.34,117.52) --
	(170.45,137.35) --
	(174.75,155.45) --
	(179.05,174.26) --
	(183.41,193.50) --
	(188.93,217.19) --
	(194.08,237.59) --
	(199.22,255.48) --
	(204.37,270.10) --
	(209.51,280.82) --
	(214.04,286.65) --
	(218.57,288.91) --
	(223.10,287.52) --
	(228.36,281.38) --
	(233.87,270.03) --
	(238.49,257.05) --
	(243.11,241.36) --
	(247.88,222.94) --
	(253.44,199.44) --
	(256.52,185.93) --
	(259.60,172.32) --
	(263.48,155.33) --
	(267.90,136.74) --
	(273.64,114.74) --
	(278.18, 99.72) --
	(282.71, 87.38) --
	(287.25, 78.12) --
	(292.62, 71.57) --
	(296.37, 69.98) --
	(300.13, 70.90) --
	(304.82, 75.56) --
	(310.77, 86.76) --
	(315.47, 99.44) --
	(320.17,115.00) --
	(324.88,132.88) --
	(330.10,154.67) --
	(333.80,170.82) --
	(337.50,187.15) --
	(341.87,206.23) --
	(347.41,229.12) --
	(352.43,247.77) --
	(357.46,263.62) --
	(362.48,276.00) --
	(367.51,284.43) --
	(371.90,288.27) --
	(376.30,288.69) --
	(380.79,285.57) --
	(386.20,277.19) --
	(391.77,263.72) --
	(396.26,249.74) --
	(400.75,233.46) --
	(405.57,213.97) --
	(410.40,193.17) --
	(413.79,178.24) --
	(417.17,163.32) --
	(422.95,138.71);
\definecolor{fillColor}{RGB}{255,255,255}

\path[draw=drawColor,line width= 0.4pt,line join=round,line cap=round,fill=fillColor] (318.74,297.70) rectangle (437.33,249.70);

\path[draw=drawColor,line width= 0.4pt,line join=round,line cap=round] (321.44,285.70) -- (339.44,285.70);

\path[draw=drawColor,line width= 0.4pt,dash pattern=on 4pt off 4pt ,line join=round,line cap=round] (321.44,261.70) -- (339.44,261.70);

\path[draw=drawColor,line width= 0.4pt,line join=round,line cap=round] (330.44,273.70) circle (  2.25);

\node[text=drawColor,anchor=base west,inner sep=0pt, outer sep=0pt, scale=  1.00] at (348.44,282.25) {exact $y(t;k+\Delta_k)$};

\node[text=drawColor,anchor=base west,inner sep=0pt, outer sep=0pt, scale=  1.00] at (348.44,270.25) {$y(t;k) + S_y(t;k) \Delta_k$};

\node[text=drawColor,anchor=base west,inner sep=0pt, outer sep=0pt, scale=  1.00] at (348.44,258.25) {exact $y(t;k)$};
\end{scope}
\end{tikzpicture}

  \caption{Linearization Error. The trajectory is shifted linearly to
    slightly changed parameters using the approximate sensitivity,
    plotted against a direct solution at
    $k+\Delta_k$.  \label{fig:dampedlinerr}}
\end{figure}

But, because the analytical solution for the sensitivity \emph{is}
available, we can also show it directly, see
Figure~\ref{fig:dampedGSLvsASolSensitivity}. This is usually not
possible. Additionally, in many cases, obtaining a sensitivity check
through finite differences requires quite a bit of work for larger
systems, maybe more than conventional forward sensitivity
analysis. That is why the linearization error is fairly useful.

\begin{figure}\sffamily\firalining
  \centering
  % Created by tikzDevice version 0.12.3.1 on 2021-05-03 23:03:22
% !TEX encoding = UTF-8 Unicode
\begin{tikzpicture}[x=1pt,y=1pt]
\definecolor{fillColor}{RGB}{255,255,255}
\path[use as bounding box,fill=fillColor,fill opacity=0.00] (0,0) rectangle (462.53,346.90);
\begin{scope}
\path[clip] ( 49.20, 61.20) rectangle (437.33,297.70);
\definecolor{drawColor}{RGB}{0,0,0}

\path[draw=drawColor,line width= 0.4pt,line join=round,line cap=round] ( 63.58,179.45) circle (  2.25);

\path[draw=drawColor,line width= 0.4pt,line join=round,line cap=round] ( 63.67,179.45) circle (  2.25);

\path[draw=drawColor,line width= 0.4pt,line join=round,line cap=round] ( 63.79,179.45) circle (  2.25);

\path[draw=drawColor,line width= 0.4pt,line join=round,line cap=round] ( 63.91,179.45) circle (  2.25);

\path[draw=drawColor,line width= 0.4pt,line join=round,line cap=round] ( 64.03,179.45) circle (  2.25);

\path[draw=drawColor,line width= 0.4pt,line join=round,line cap=round] ( 64.16,179.44) circle (  2.25);

\path[draw=drawColor,line width= 0.4pt,line join=round,line cap=round] ( 64.28,179.44) circle (  2.25);

\path[draw=drawColor,line width= 0.4pt,line join=round,line cap=round] ( 64.40,179.44) circle (  2.25);

\path[draw=drawColor,line width= 0.4pt,line join=round,line cap=round] ( 64.57,179.44) circle (  2.25);

\path[draw=drawColor,line width= 0.4pt,line join=round,line cap=round] ( 64.84,179.43) circle (  2.25);

\path[draw=drawColor,line width= 0.4pt,line join=round,line cap=round] ( 65.30,179.41) circle (  2.25);

\path[draw=drawColor,line width= 0.4pt,line join=round,line cap=round] ( 65.91,179.38) circle (  2.25);

\path[draw=drawColor,line width= 0.4pt,line join=round,line cap=round] ( 66.51,179.35) circle (  2.25);

\path[draw=drawColor,line width= 0.4pt,line join=round,line cap=round] ( 67.12,179.30) circle (  2.25);

\path[draw=drawColor,line width= 0.4pt,line join=round,line cap=round] ( 67.72,179.25) circle (  2.25);

\path[draw=drawColor,line width= 0.4pt,line join=round,line cap=round] ( 68.33,179.18) circle (  2.25);

\path[draw=drawColor,line width= 0.4pt,line join=round,line cap=round] ( 68.93,179.11) circle (  2.25);

\path[draw=drawColor,line width= 0.4pt,line join=round,line cap=round] ( 69.54,179.03) circle (  2.25);

\path[draw=drawColor,line width= 0.4pt,line join=round,line cap=round] ( 70.14,178.94) circle (  2.25);

\path[draw=drawColor,line width= 0.4pt,line join=round,line cap=round] ( 70.75,178.85) circle (  2.25);

\path[draw=drawColor,line width= 0.4pt,line join=round,line cap=round] ( 71.51,178.71) circle (  2.25);

\path[draw=drawColor,line width= 0.4pt,line join=round,line cap=round] ( 72.67,178.49) circle (  2.25);

\path[draw=drawColor,line width= 0.4pt,line join=round,line cap=round] ( 75.25,177.88) circle (  2.25);

\path[draw=drawColor,line width= 0.4pt,line join=round,line cap=round] ( 77.82,177.14) circle (  2.25);

\path[draw=drawColor,line width= 0.4pt,line join=round,line cap=round] ( 80.39,176.28) circle (  2.25);

\path[draw=drawColor,line width= 0.4pt,line join=round,line cap=round] ( 82.96,175.31) circle (  2.25);

\path[draw=drawColor,line width= 0.4pt,line join=round,line cap=round] ( 85.53,174.24) circle (  2.25);

\path[draw=drawColor,line width= 0.4pt,line join=round,line cap=round] ( 88.11,173.10) circle (  2.25);

\path[draw=drawColor,line width= 0.4pt,line join=round,line cap=round] ( 90.68,171.90) circle (  2.25);

\path[draw=drawColor,line width= 0.4pt,line join=round,line cap=round] ( 93.25,170.67) circle (  2.25);

\path[draw=drawColor,line width= 0.4pt,line join=round,line cap=round] ( 95.82,169.41) circle (  2.25);

\path[draw=drawColor,line width= 0.4pt,line join=round,line cap=round] ( 99.24,167.76) circle (  2.25);

\path[draw=drawColor,line width= 0.4pt,line join=round,line cap=round] (102.32,166.32) circle (  2.25);

\path[draw=drawColor,line width= 0.4pt,line join=round,line cap=round] (105.40,164.96) circle (  2.25);

\path[draw=drawColor,line width= 0.4pt,line join=round,line cap=round] (108.48,163.73) circle (  2.25);

\path[draw=drawColor,line width= 0.4pt,line join=round,line cap=round] (111.55,162.67) circle (  2.25);

\path[draw=drawColor,line width= 0.4pt,line join=round,line cap=round] (114.78,161.77) circle (  2.25);

\path[draw=drawColor,line width= 0.4pt,line join=round,line cap=round] (118.93,160.99) circle (  2.25);

\path[draw=drawColor,line width= 0.4pt,line join=round,line cap=round] (125.00,160.75) circle (  2.25);

\path[draw=drawColor,line width= 0.4pt,line join=round,line cap=round] (130.62,161.60) circle (  2.25);

\path[draw=drawColor,line width= 0.4pt,line join=round,line cap=round] (136.23,163.57) circle (  2.25);

\path[draw=drawColor,line width= 0.4pt,line join=round,line cap=round] (141.85,166.70) circle (  2.25);

\path[draw=drawColor,line width= 0.4pt,line join=round,line cap=round] (147.47,170.95) circle (  2.25);

\path[draw=drawColor,line width= 0.4pt,line join=round,line cap=round] (153.09,176.22) circle (  2.25);

\path[draw=drawColor,line width= 0.4pt,line join=round,line cap=round] (158.71,182.35) circle (  2.25);

\path[draw=drawColor,line width= 0.4pt,line join=round,line cap=round] (164.32,189.11) circle (  2.25);

\path[draw=drawColor,line width= 0.4pt,line join=round,line cap=round] (170.28,196.65) circle (  2.25);

\path[draw=drawColor,line width= 0.4pt,line join=round,line cap=round] (176.24,204.21) circle (  2.25);

\path[draw=drawColor,line width= 0.4pt,line join=round,line cap=round] (182.19,211.37) circle (  2.25);

\path[draw=drawColor,line width= 0.4pt,line join=round,line cap=round] (188.15,217.72) circle (  2.25);

\path[draw=drawColor,line width= 0.4pt,line join=round,line cap=round] (193.16,222.12) circle (  2.25);

\path[draw=drawColor,line width= 0.4pt,line join=round,line cap=round] (198.17,225.42) circle (  2.25);

\path[draw=drawColor,line width= 0.4pt,line join=round,line cap=round] (203.25,227.43) circle (  2.25);

\path[draw=drawColor,line width= 0.4pt,line join=round,line cap=round] (209.69,227.80) circle (  2.25);

\path[draw=drawColor,line width= 0.4pt,line join=round,line cap=round] (215.69,225.78) circle (  2.25);

\path[draw=drawColor,line width= 0.4pt,line join=round,line cap=round] (221.68,221.40) circle (  2.25);

\path[draw=drawColor,line width= 0.4pt,line join=round,line cap=round] (227.68,214.69) circle (  2.25);

\path[draw=drawColor,line width= 0.4pt,line join=round,line cap=round] (233.68,205.81) circle (  2.25);

\path[draw=drawColor,line width= 0.4pt,line join=round,line cap=round] (238.95,196.44) circle (  2.25);

\path[draw=drawColor,line width= 0.4pt,line join=round,line cap=round] (244.23,185.88) circle (  2.25);

\path[draw=drawColor,line width= 0.4pt,line join=round,line cap=round] (249.51,174.45) circle (  2.25);

\path[draw=drawColor,line width= 0.4pt,line join=round,line cap=round] (255.64,160.59) circle (  2.25);

\path[draw=drawColor,line width= 0.4pt,line join=round,line cap=round] (262.06,146.07) circle (  2.25);

\path[draw=drawColor,line width= 0.4pt,line join=round,line cap=round] (267.45,134.49) circle (  2.25);

\path[draw=drawColor,line width= 0.4pt,line join=round,line cap=round] (272.83,123.94) circle (  2.25);

\path[draw=drawColor,line width= 0.4pt,line join=round,line cap=round] (278.39,114.62) circle (  2.25);

\path[draw=drawColor,line width= 0.4pt,line join=round,line cap=round] (284.88,106.36) circle (  2.25);

\path[draw=drawColor,line width= 0.4pt,line join=round,line cap=round] (288.47,103.22) circle (  2.25);

\path[draw=drawColor,line width= 0.4pt,line join=round,line cap=round] (292.05,101.21) circle (  2.25);

\path[draw=drawColor,line width= 0.4pt,line join=round,line cap=round] (296.58,100.42) circle (  2.25);

\path[draw=drawColor,line width= 0.4pt,line join=round,line cap=round] (301.74,101.96) circle (  2.25);

\path[draw=drawColor,line width= 0.4pt,line join=round,line cap=round] (308.42,107.90) circle (  2.25);

\path[draw=drawColor,line width= 0.4pt,line join=round,line cap=round] (313.71,115.70) circle (  2.25);

\path[draw=drawColor,line width= 0.4pt,line join=round,line cap=round] (318.99,126.07) circle (  2.25);

\path[draw=drawColor,line width= 0.4pt,line join=round,line cap=round] (324.28,138.74) circle (  2.25);

\path[draw=drawColor,line width= 0.4pt,line join=round,line cap=round] (330.54,156.20) circle (  2.25);

\path[draw=drawColor,line width= 0.4pt,line join=round,line cap=round] (334.92,169.64) circle (  2.25);

\path[draw=drawColor,line width= 0.4pt,line join=round,line cap=round] (339.29,183.76) circle (  2.25);

\path[draw=drawColor,line width= 0.4pt,line join=round,line cap=round] (344.77,201.87) circle (  2.25);

\path[draw=drawColor,line width= 0.4pt,line join=round,line cap=round] (351.70,224.54) circle (  2.25);

\path[draw=drawColor,line width= 0.4pt,line join=round,line cap=round] (357.18,241.42) circle (  2.25);

\path[draw=drawColor,line width= 0.4pt,line join=round,line cap=round] (362.66,256.63) circle (  2.25);

\path[draw=drawColor,line width= 0.4pt,line join=round,line cap=round] (368.14,269.53) circle (  2.25);

\path[draw=drawColor,line width= 0.4pt,line join=round,line cap=round] (374.23,280.46) circle (  2.25);

\path[draw=drawColor,line width= 0.4pt,line join=round,line cap=round] (378.54,285.69) circle (  2.25);

\path[draw=drawColor,line width= 0.4pt,line join=round,line cap=round] (382.85,288.61) circle (  2.25);

\path[draw=drawColor,line width= 0.4pt,line join=round,line cap=round] (387.95,288.94) circle (  2.25);

\path[draw=drawColor,line width= 0.4pt,line join=round,line cap=round] (394.40,284.32) circle (  2.25);

\path[draw=drawColor,line width= 0.4pt,line join=round,line cap=round] (400.26,275.30) circle (  2.25);

\path[draw=drawColor,line width= 0.4pt,line join=round,line cap=round] (406.12,261.91) circle (  2.25);

\path[draw=drawColor,line width= 0.4pt,line join=round,line cap=round] (411.97,244.57) circle (  2.25);

\path[draw=drawColor,line width= 0.4pt,line join=round,line cap=round] (417.83,223.89) circle (  2.25);

\path[draw=drawColor,line width= 0.4pt,line join=round,line cap=round] (422.95,203.66) circle (  2.25);
\end{scope}
\begin{scope}
\path[clip] (  0.00,  0.00) rectangle (462.53,346.90);
\definecolor{drawColor}{RGB}{0,0,0}

\path[draw=drawColor,line width= 0.4pt,line join=round,line cap=round] ( 63.58, 61.20) -- (413.56, 61.20);

\path[draw=drawColor,line width= 0.4pt,line join=round,line cap=round] ( 63.58, 61.20) -- ( 63.58, 55.20);

\path[draw=drawColor,line width= 0.4pt,line join=round,line cap=round] (151.07, 61.20) -- (151.07, 55.20);

\path[draw=drawColor,line width= 0.4pt,line join=round,line cap=round] (238.57, 61.20) -- (238.57, 55.20);

\path[draw=drawColor,line width= 0.4pt,line join=round,line cap=round] (326.07, 61.20) -- (326.07, 55.20);

\path[draw=drawColor,line width= 0.4pt,line join=round,line cap=round] (413.56, 61.20) -- (413.56, 55.20);

\node[text=drawColor,anchor=base,inner sep=0pt, outer sep=0pt, scale=  1.00] at ( 63.58, 39.60) {0};

\node[text=drawColor,anchor=base,inner sep=0pt, outer sep=0pt, scale=  1.00] at (151.07, 39.60) {5};

\node[text=drawColor,anchor=base,inner sep=0pt, outer sep=0pt, scale=  1.00] at (238.57, 39.60) {10};

\node[text=drawColor,anchor=base,inner sep=0pt, outer sep=0pt, scale=  1.00] at (326.07, 39.60) {15};

\node[text=drawColor,anchor=base,inner sep=0pt, outer sep=0pt, scale=  1.00] at (413.56, 39.60) {20};

\path[draw=drawColor,line width= 0.4pt,line join=round,line cap=round] ( 49.20, 70.83) -- ( 49.20,288.06);

\path[draw=drawColor,line width= 0.4pt,line join=round,line cap=round] ( 49.20, 70.83) -- ( 43.20, 70.83);

\path[draw=drawColor,line width= 0.4pt,line join=round,line cap=round] ( 49.20,107.04) -- ( 43.20,107.04);

\path[draw=drawColor,line width= 0.4pt,line join=round,line cap=round] ( 49.20,143.24) -- ( 43.20,143.24);

\path[draw=drawColor,line width= 0.4pt,line join=round,line cap=round] ( 49.20,179.45) -- ( 43.20,179.45);

\path[draw=drawColor,line width= 0.4pt,line join=round,line cap=round] ( 49.20,215.65) -- ( 43.20,215.65);

\path[draw=drawColor,line width= 0.4pt,line join=round,line cap=round] ( 49.20,251.86) -- ( 43.20,251.86);

\path[draw=drawColor,line width= 0.4pt,line join=round,line cap=round] ( 49.20,288.06) -- ( 43.20,288.06);

\node[text=drawColor,rotate= 90.00,anchor=base,inner sep=0pt, outer sep=0pt, scale=  1.00] at ( 34.80, 70.83) {-15};

\node[text=drawColor,rotate= 90.00,anchor=base,inner sep=0pt, outer sep=0pt, scale=  1.00] at ( 34.80,107.04) {-10};

\node[text=drawColor,rotate= 90.00,anchor=base,inner sep=0pt, outer sep=0pt, scale=  1.00] at ( 34.80,143.24) {-5};

\node[text=drawColor,rotate= 90.00,anchor=base,inner sep=0pt, outer sep=0pt, scale=  1.00] at ( 34.80,179.45) {0};

\node[text=drawColor,rotate= 90.00,anchor=base,inner sep=0pt, outer sep=0pt, scale=  1.00] at ( 34.80,215.65) {5};

\node[text=drawColor,rotate= 90.00,anchor=base,inner sep=0pt, outer sep=0pt, scale=  1.00] at ( 34.80,251.86) {10};

\node[text=drawColor,rotate= 90.00,anchor=base,inner sep=0pt, outer sep=0pt, scale=  1.00] at ( 34.80,288.06) {15};

\path[draw=drawColor,line width= 0.4pt,line join=round,line cap=round] ( 49.20, 61.20) --
	(437.33, 61.20) --
	(437.33,297.70) --
	( 49.20,297.70) --
	( 49.20, 61.20);
\end{scope}
\begin{scope}
\path[clip] (  0.00,  0.00) rectangle (462.53,346.90);
\definecolor{drawColor}{RGB}{0,0,0}

\node[text=drawColor,anchor=base,inner sep=0pt, outer sep=0pt, scale=  1.20] at (243.26,318.16) {\bfseries sensitivity approximation};

\node[text=drawColor,anchor=base,inner sep=0pt, outer sep=0pt, scale=  1.00] at (243.26, 15.60) {time};

\node[text=drawColor,rotate= 90.00,anchor=base,inner sep=0pt, outer sep=0pt, scale=  1.00] at ( 10.80,179.45) {sensitivity};
\end{scope}
\begin{scope}
\path[clip] ( 49.20, 61.20) rectangle (437.33,297.70);
\definecolor{drawColor}{RGB}{0,0,0}

\path[draw=drawColor,line width= 0.4pt,line join=round,line cap=round] ( 63.58,179.45) --
	( 63.67,179.45) --
	( 63.79,179.45) --
	( 63.91,179.45) --
	( 64.03,179.45) --
	( 64.16,179.44) --
	( 64.28,179.44) --
	( 64.40,179.44) --
	( 64.57,179.44) --
	( 64.84,179.43) --
	( 65.30,179.41) --
	( 65.91,179.38) --
	( 66.51,179.35) --
	( 67.12,179.30) --
	( 67.72,179.25) --
	( 68.33,179.18) --
	( 68.93,179.11) --
	( 69.54,179.03) --
	( 70.14,178.94) --
	( 70.75,178.85) --
	( 71.51,178.71) --
	( 72.67,178.49) --
	( 75.25,177.88) --
	( 77.82,177.15) --
	( 80.39,176.29) --
	( 82.96,175.33) --
	( 85.53,174.28) --
	( 88.11,173.16) --
	( 90.68,171.99) --
	( 93.25,170.78) --
	( 95.82,169.56) --
	( 99.24,167.96) --
	(102.32,166.57) --
	(105.40,165.29) --
	(108.48,164.13) --
	(111.55,163.15) --
	(114.78,162.34) --
	(118.93,161.70) --
	(125.00,161.68) --
	(130.62,162.76) --
	(136.23,164.98) --
	(141.85,168.35) --
	(147.47,172.82) --
	(153.09,178.26) --
	(158.71,184.48) --
	(164.32,191.24) --
	(170.28,198.68) --
	(176.24,206.01) --
	(182.19,212.82) --
	(188.15,218.68) --
	(193.16,222.58) --
	(198.17,225.31) --
	(203.25,226.70) --
	(209.69,226.23) --
	(215.69,223.41) --
	(221.68,218.25) --
	(227.68,210.84) --
	(233.68,201.37) --
	(238.95,191.60) --
	(244.23,180.78) --
	(249.51,169.26) --
	(255.64,155.52) --
	(262.06,141.42) --
	(267.45,130.39) --
	(272.83,120.58) --
	(278.39,112.20) --
	(284.88,105.23) --
	(288.47,102.87) --
	(292.05,101.68) --
	(296.58,101.92) --
	(301.74,104.64) --
	(308.42,112.08) --
	(313.71,120.96) --
	(318.99,132.29) --
	(324.28,145.75) --
	(330.54,163.88) --
	(334.92,177.58) --
	(339.29,191.79) --
	(344.77,209.74) --
	(351.70,231.78) --
	(357.18,247.82) --
	(362.66,261.92) --
	(368.14,273.47) --
	(374.23,282.67) --
	(378.54,286.55) --
	(382.85,288.08) --
	(387.95,286.73) --
	(394.40,280.01) --
	(400.26,269.19) --
	(406.12,254.18) --
	(411.97,235.48) --
	(417.83,213.76) --
	(422.95,192.95);

\path[draw=drawColor,line width= 0.4pt,dash pattern=on 4pt off 4pt ,line join=round,line cap=round] ( 63.58,179.45) --
	( 63.67,179.45) --
	( 63.79,179.45) --
	( 63.91,179.45) --
	( 64.03,179.45) --
	( 64.16,179.44) --
	( 64.28,179.44) --
	( 64.40,179.44) --
	( 64.57,179.44) --
	( 64.84,179.43) --
	( 65.30,179.41) --
	( 65.91,179.38) --
	( 66.51,179.35) --
	( 67.12,179.30) --
	( 67.72,179.25) --
	( 68.33,179.18) --
	( 68.93,179.11) --
	( 69.54,179.03) --
	( 70.14,178.94) --
	( 70.75,178.85) --
	( 71.51,178.71) --
	( 72.67,178.49) --
	( 75.25,177.88) --
	( 77.82,177.15) --
	( 80.39,176.29) --
	( 82.96,175.33) --
	( 85.53,174.28) --
	( 88.11,173.16) --
	( 90.68,171.99) --
	( 93.25,170.78) --
	( 95.82,169.56) --
	( 99.24,167.96) --
	(102.32,166.57) --
	(105.40,165.29) --
	(108.48,164.13) --
	(111.55,163.15) --
	(114.78,162.34) --
	(118.93,161.70) --
	(125.00,161.68) --
	(130.62,162.76) --
	(136.23,164.98) --
	(141.85,168.35) --
	(147.47,172.82) --
	(153.09,178.26) --
	(158.71,184.48) --
	(164.32,191.24) --
	(170.28,198.68) --
	(176.24,206.01) --
	(182.19,212.82) --
	(188.15,218.68) --
	(193.16,222.58) --
	(198.17,225.31) --
	(203.25,226.70) --
	(209.69,226.23) --
	(215.69,223.41) --
	(221.68,218.25) --
	(227.68,210.84) --
	(233.68,201.37) --
	(238.95,191.60) --
	(244.23,180.78) --
	(249.51,169.26) --
	(255.64,155.52) --
	(262.06,141.42) --
	(267.45,130.39) --
	(272.83,120.58) --
	(278.39,112.20) --
	(284.88,105.23) --
	(288.47,102.87) --
	(292.05,101.68) --
	(296.58,101.92) --
	(301.74,104.64) --
	(308.42,112.08) --
	(313.71,120.96) --
	(318.99,132.29) --
	(324.28,145.75) --
	(330.54,163.88) --
	(334.92,177.58) --
	(339.29,191.79) --
	(344.77,209.74) --
	(351.70,231.78) --
	(357.18,247.82) --
	(362.66,261.92) --
	(368.14,273.47) --
	(374.23,282.67) --
	(378.54,286.55) --
	(382.85,288.08) --
	(387.95,286.73) --
	(394.40,280.01) --
	(400.26,269.19) --
	(406.12,254.18) --
	(411.97,235.48) --
	(417.83,213.76) --
	(422.95,192.95);

\path[draw=drawColor,line width= 0.4pt,line join=round,line cap=round] ( 61.33,177.20) -- ( 65.83,181.70);

\path[draw=drawColor,line width= 0.4pt,line join=round,line cap=round] ( 61.33,181.70) -- ( 65.83,177.20);

\path[draw=drawColor,line width= 0.4pt,line join=round,line cap=round] ( 61.42,177.20) -- ( 65.92,181.70);

\path[draw=drawColor,line width= 0.4pt,line join=round,line cap=round] ( 61.42,181.70) -- ( 65.92,177.20);

\path[draw=drawColor,line width= 0.4pt,line join=round,line cap=round] ( 61.54,177.20) -- ( 66.04,181.70);

\path[draw=drawColor,line width= 0.4pt,line join=round,line cap=round] ( 61.54,181.70) -- ( 66.04,177.20);

\path[draw=drawColor,line width= 0.4pt,line join=round,line cap=round] ( 61.66,177.20) -- ( 66.16,181.70);

\path[draw=drawColor,line width= 0.4pt,line join=round,line cap=round] ( 61.66,181.70) -- ( 66.16,177.20);

\path[draw=drawColor,line width= 0.4pt,line join=round,line cap=round] ( 61.78,177.20) -- ( 66.28,181.70);

\path[draw=drawColor,line width= 0.4pt,line join=round,line cap=round] ( 61.78,181.70) -- ( 66.28,177.20);

\path[draw=drawColor,line width= 0.4pt,line join=round,line cap=round] ( 61.91,177.19) -- ( 66.41,181.69);

\path[draw=drawColor,line width= 0.4pt,line join=round,line cap=round] ( 61.91,181.69) -- ( 66.41,177.19);

\path[draw=drawColor,line width= 0.4pt,line join=round,line cap=round] ( 62.03,177.19) -- ( 66.53,181.69);

\path[draw=drawColor,line width= 0.4pt,line join=round,line cap=round] ( 62.03,181.69) -- ( 66.53,177.19);

\path[draw=drawColor,line width= 0.4pt,line join=round,line cap=round] ( 62.15,177.19) -- ( 66.65,181.69);

\path[draw=drawColor,line width= 0.4pt,line join=round,line cap=round] ( 62.15,181.69) -- ( 66.65,177.19);

\path[draw=drawColor,line width= 0.4pt,line join=round,line cap=round] ( 62.32,177.19) -- ( 66.82,181.69);

\path[draw=drawColor,line width= 0.4pt,line join=round,line cap=round] ( 62.32,181.69) -- ( 66.82,177.19);

\path[draw=drawColor,line width= 0.4pt,line join=round,line cap=round] ( 62.59,177.18) -- ( 67.09,181.68);

\path[draw=drawColor,line width= 0.4pt,line join=round,line cap=round] ( 62.59,181.68) -- ( 67.09,177.18);

\path[draw=drawColor,line width= 0.4pt,line join=round,line cap=round] ( 63.05,177.16) -- ( 67.55,181.66);

\path[draw=drawColor,line width= 0.4pt,line join=round,line cap=round] ( 63.05,181.66) -- ( 67.55,177.16);

\path[draw=drawColor,line width= 0.4pt,line join=round,line cap=round] ( 63.66,177.13) -- ( 68.16,181.63);

\path[draw=drawColor,line width= 0.4pt,line join=round,line cap=round] ( 63.66,181.63) -- ( 68.16,177.13);

\path[draw=drawColor,line width= 0.4pt,line join=round,line cap=round] ( 64.26,177.10) -- ( 68.76,181.60);

\path[draw=drawColor,line width= 0.4pt,line join=round,line cap=round] ( 64.26,181.60) -- ( 68.76,177.10);

\path[draw=drawColor,line width= 0.4pt,line join=round,line cap=round] ( 64.87,177.05) -- ( 69.37,181.55);

\path[draw=drawColor,line width= 0.4pt,line join=round,line cap=round] ( 64.87,181.55) -- ( 69.37,177.05);

\path[draw=drawColor,line width= 0.4pt,line join=round,line cap=round] ( 65.47,177.00) -- ( 69.97,181.50);

\path[draw=drawColor,line width= 0.4pt,line join=round,line cap=round] ( 65.47,181.50) -- ( 69.97,177.00);

\path[draw=drawColor,line width= 0.4pt,line join=round,line cap=round] ( 66.08,176.93) -- ( 70.58,181.43);

\path[draw=drawColor,line width= 0.4pt,line join=round,line cap=round] ( 66.08,181.43) -- ( 70.58,176.93);

\path[draw=drawColor,line width= 0.4pt,line join=round,line cap=round] ( 66.68,176.86) -- ( 71.18,181.36);

\path[draw=drawColor,line width= 0.4pt,line join=round,line cap=round] ( 66.68,181.36) -- ( 71.18,176.86);

\path[draw=drawColor,line width= 0.4pt,line join=round,line cap=round] ( 67.29,176.78) -- ( 71.79,181.28);

\path[draw=drawColor,line width= 0.4pt,line join=round,line cap=round] ( 67.29,181.28) -- ( 71.79,176.78);

\path[draw=drawColor,line width= 0.4pt,line join=round,line cap=round] ( 67.89,176.69) -- ( 72.39,181.19);

\path[draw=drawColor,line width= 0.4pt,line join=round,line cap=round] ( 67.89,181.19) -- ( 72.39,176.69);

\path[draw=drawColor,line width= 0.4pt,line join=round,line cap=round] ( 68.50,176.60) -- ( 73.00,181.10);

\path[draw=drawColor,line width= 0.4pt,line join=round,line cap=round] ( 68.50,181.10) -- ( 73.00,176.60);

\path[draw=drawColor,line width= 0.4pt,line join=round,line cap=round] ( 69.26,176.46) -- ( 73.76,180.96);

\path[draw=drawColor,line width= 0.4pt,line join=round,line cap=round] ( 69.26,180.96) -- ( 73.76,176.46);

\path[draw=drawColor,line width= 0.4pt,line join=round,line cap=round] ( 70.42,176.24) -- ( 74.92,180.74);

\path[draw=drawColor,line width= 0.4pt,line join=round,line cap=round] ( 70.42,180.74) -- ( 74.92,176.24);

\path[draw=drawColor,line width= 0.4pt,line join=round,line cap=round] ( 73.00,175.63) -- ( 77.50,180.13);

\path[draw=drawColor,line width= 0.4pt,line join=round,line cap=round] ( 73.00,180.13) -- ( 77.50,175.63);

\path[draw=drawColor,line width= 0.4pt,line join=round,line cap=round] ( 75.57,174.90) -- ( 80.07,179.40);

\path[draw=drawColor,line width= 0.4pt,line join=round,line cap=round] ( 75.57,179.40) -- ( 80.07,174.90);

\path[draw=drawColor,line width= 0.4pt,line join=round,line cap=round] ( 78.14,174.04) -- ( 82.64,178.54);

\path[draw=drawColor,line width= 0.4pt,line join=round,line cap=round] ( 78.14,178.54) -- ( 82.64,174.04);

\path[draw=drawColor,line width= 0.4pt,line join=round,line cap=round] ( 80.71,173.08) -- ( 85.21,177.58);

\path[draw=drawColor,line width= 0.4pt,line join=round,line cap=round] ( 80.71,177.58) -- ( 85.21,173.08);

\path[draw=drawColor,line width= 0.4pt,line join=round,line cap=round] ( 83.28,172.03) -- ( 87.78,176.53);

\path[draw=drawColor,line width= 0.4pt,line join=round,line cap=round] ( 83.28,176.53) -- ( 87.78,172.03);

\path[draw=drawColor,line width= 0.4pt,line join=round,line cap=round] ( 85.86,170.91) -- ( 90.36,175.41);

\path[draw=drawColor,line width= 0.4pt,line join=round,line cap=round] ( 85.86,175.41) -- ( 90.36,170.91);

\path[draw=drawColor,line width= 0.4pt,line join=round,line cap=round] ( 88.43,169.74) -- ( 92.93,174.24);

\path[draw=drawColor,line width= 0.4pt,line join=round,line cap=round] ( 88.43,174.24) -- ( 92.93,169.74);

\path[draw=drawColor,line width= 0.4pt,line join=round,line cap=round] ( 91.00,168.53) -- ( 95.50,173.03);

\path[draw=drawColor,line width= 0.4pt,line join=round,line cap=round] ( 91.00,173.03) -- ( 95.50,168.53);

\path[draw=drawColor,line width= 0.4pt,line join=round,line cap=round] ( 93.57,167.31) -- ( 98.07,171.81);

\path[draw=drawColor,line width= 0.4pt,line join=round,line cap=round] ( 93.57,171.81) -- ( 98.07,167.31);

\path[draw=drawColor,line width= 0.4pt,line join=round,line cap=round] ( 96.99,165.71) -- (101.49,170.21);

\path[draw=drawColor,line width= 0.4pt,line join=round,line cap=round] ( 96.99,170.21) -- (101.49,165.71);

\path[draw=drawColor,line width= 0.4pt,line join=round,line cap=round] (100.07,164.32) -- (104.57,168.82);

\path[draw=drawColor,line width= 0.4pt,line join=round,line cap=round] (100.07,168.82) -- (104.57,164.32);

\path[draw=drawColor,line width= 0.4pt,line join=round,line cap=round] (103.15,163.04) -- (107.65,167.54);

\path[draw=drawColor,line width= 0.4pt,line join=round,line cap=round] (103.15,167.54) -- (107.65,163.04);

\path[draw=drawColor,line width= 0.4pt,line join=round,line cap=round] (106.23,161.88) -- (110.73,166.38);

\path[draw=drawColor,line width= 0.4pt,line join=round,line cap=round] (106.23,166.38) -- (110.73,161.88);

\path[draw=drawColor,line width= 0.4pt,line join=round,line cap=round] (109.30,160.90) -- (113.80,165.40);

\path[draw=drawColor,line width= 0.4pt,line join=round,line cap=round] (109.30,165.40) -- (113.80,160.90);

\path[draw=drawColor,line width= 0.4pt,line join=round,line cap=round] (112.53,160.09) -- (117.03,164.59);

\path[draw=drawColor,line width= 0.4pt,line join=round,line cap=round] (112.53,164.59) -- (117.03,160.09);

\path[draw=drawColor,line width= 0.4pt,line join=round,line cap=round] (116.68,159.45) -- (121.18,163.95);

\path[draw=drawColor,line width= 0.4pt,line join=round,line cap=round] (116.68,163.95) -- (121.18,159.45);

\path[draw=drawColor,line width= 0.4pt,line join=round,line cap=round] (122.75,159.43) -- (127.25,163.93);

\path[draw=drawColor,line width= 0.4pt,line join=round,line cap=round] (122.75,163.93) -- (127.25,159.43);

\path[draw=drawColor,line width= 0.4pt,line join=round,line cap=round] (128.37,160.51) -- (132.87,165.01);

\path[draw=drawColor,line width= 0.4pt,line join=round,line cap=round] (128.37,165.01) -- (132.87,160.51);

\path[draw=drawColor,line width= 0.4pt,line join=round,line cap=round] (133.98,162.73) -- (138.48,167.23);

\path[draw=drawColor,line width= 0.4pt,line join=round,line cap=round] (133.98,167.23) -- (138.48,162.73);

\path[draw=drawColor,line width= 0.4pt,line join=round,line cap=round] (139.60,166.10) -- (144.10,170.60);

\path[draw=drawColor,line width= 0.4pt,line join=round,line cap=round] (139.60,170.60) -- (144.10,166.10);

\path[draw=drawColor,line width= 0.4pt,line join=round,line cap=round] (145.22,170.57) -- (149.72,175.07);

\path[draw=drawColor,line width= 0.4pt,line join=round,line cap=round] (145.22,175.07) -- (149.72,170.57);

\path[draw=drawColor,line width= 0.4pt,line join=round,line cap=round] (150.84,176.01) -- (155.34,180.51);

\path[draw=drawColor,line width= 0.4pt,line join=round,line cap=round] (150.84,180.51) -- (155.34,176.01);

\path[draw=drawColor,line width= 0.4pt,line join=round,line cap=round] (156.46,182.23) -- (160.96,186.73);

\path[draw=drawColor,line width= 0.4pt,line join=round,line cap=round] (156.46,186.73) -- (160.96,182.23);

\path[draw=drawColor,line width= 0.4pt,line join=round,line cap=round] (162.07,188.99) -- (166.57,193.49);

\path[draw=drawColor,line width= 0.4pt,line join=round,line cap=round] (162.07,193.49) -- (166.57,188.99);

\path[draw=drawColor,line width= 0.4pt,line join=round,line cap=round] (168.03,196.43) -- (172.53,200.93);

\path[draw=drawColor,line width= 0.4pt,line join=round,line cap=round] (168.03,200.93) -- (172.53,196.43);

\path[draw=drawColor,line width= 0.4pt,line join=round,line cap=round] (173.99,203.76) -- (178.49,208.26);

\path[draw=drawColor,line width= 0.4pt,line join=round,line cap=round] (173.99,208.26) -- (178.49,203.76);

\path[draw=drawColor,line width= 0.4pt,line join=round,line cap=round] (179.94,210.57) -- (184.44,215.07);

\path[draw=drawColor,line width= 0.4pt,line join=round,line cap=round] (179.94,215.07) -- (184.44,210.57);

\path[draw=drawColor,line width= 0.4pt,line join=round,line cap=round] (185.90,216.43) -- (190.40,220.93);

\path[draw=drawColor,line width= 0.4pt,line join=round,line cap=round] (185.90,220.93) -- (190.40,216.43);

\path[draw=drawColor,line width= 0.4pt,line join=round,line cap=round] (190.91,220.33) -- (195.41,224.83);

\path[draw=drawColor,line width= 0.4pt,line join=round,line cap=round] (190.91,224.83) -- (195.41,220.33);

\path[draw=drawColor,line width= 0.4pt,line join=round,line cap=round] (195.92,223.06) -- (200.42,227.56);

\path[draw=drawColor,line width= 0.4pt,line join=round,line cap=round] (195.92,227.56) -- (200.42,223.06);

\path[draw=drawColor,line width= 0.4pt,line join=round,line cap=round] (201.00,224.45) -- (205.50,228.95);

\path[draw=drawColor,line width= 0.4pt,line join=round,line cap=round] (201.00,228.95) -- (205.50,224.45);

\path[draw=drawColor,line width= 0.4pt,line join=round,line cap=round] (207.44,223.98) -- (211.94,228.48);

\path[draw=drawColor,line width= 0.4pt,line join=round,line cap=round] (207.44,228.48) -- (211.94,223.98);

\path[draw=drawColor,line width= 0.4pt,line join=round,line cap=round] (213.44,221.16) -- (217.94,225.66);

\path[draw=drawColor,line width= 0.4pt,line join=round,line cap=round] (213.44,225.66) -- (217.94,221.16);

\path[draw=drawColor,line width= 0.4pt,line join=round,line cap=round] (219.43,216.00) -- (223.93,220.50);

\path[draw=drawColor,line width= 0.4pt,line join=round,line cap=round] (219.43,220.50) -- (223.93,216.00);

\path[draw=drawColor,line width= 0.4pt,line join=round,line cap=round] (225.43,208.59) -- (229.93,213.09);

\path[draw=drawColor,line width= 0.4pt,line join=round,line cap=round] (225.43,213.09) -- (229.93,208.59);

\path[draw=drawColor,line width= 0.4pt,line join=round,line cap=round] (231.43,199.12) -- (235.93,203.62);

\path[draw=drawColor,line width= 0.4pt,line join=round,line cap=round] (231.43,203.62) -- (235.93,199.12);

\path[draw=drawColor,line width= 0.4pt,line join=round,line cap=round] (236.70,189.35) -- (241.20,193.85);

\path[draw=drawColor,line width= 0.4pt,line join=round,line cap=round] (236.70,193.85) -- (241.20,189.35);

\path[draw=drawColor,line width= 0.4pt,line join=round,line cap=round] (241.98,178.53) -- (246.48,183.03);

\path[draw=drawColor,line width= 0.4pt,line join=round,line cap=round] (241.98,183.03) -- (246.48,178.53);

\path[draw=drawColor,line width= 0.4pt,line join=round,line cap=round] (247.26,167.01) -- (251.76,171.51);

\path[draw=drawColor,line width= 0.4pt,line join=round,line cap=round] (247.26,171.51) -- (251.76,167.01);

\path[draw=drawColor,line width= 0.4pt,line join=round,line cap=round] (253.39,153.27) -- (257.89,157.77);

\path[draw=drawColor,line width= 0.4pt,line join=round,line cap=round] (253.39,157.77) -- (257.89,153.27);

\path[draw=drawColor,line width= 0.4pt,line join=round,line cap=round] (259.81,139.17) -- (264.31,143.67);

\path[draw=drawColor,line width= 0.4pt,line join=round,line cap=round] (259.81,143.67) -- (264.31,139.17);

\path[draw=drawColor,line width= 0.4pt,line join=round,line cap=round] (265.20,128.14) -- (269.70,132.64);

\path[draw=drawColor,line width= 0.4pt,line join=round,line cap=round] (265.20,132.64) -- (269.70,128.14);

\path[draw=drawColor,line width= 0.4pt,line join=round,line cap=round] (270.58,118.33) -- (275.08,122.83);

\path[draw=drawColor,line width= 0.4pt,line join=round,line cap=round] (270.58,122.83) -- (275.08,118.33);

\path[draw=drawColor,line width= 0.4pt,line join=round,line cap=round] (276.14,109.95) -- (280.64,114.45);

\path[draw=drawColor,line width= 0.4pt,line join=round,line cap=round] (276.14,114.45) -- (280.64,109.95);

\path[draw=drawColor,line width= 0.4pt,line join=round,line cap=round] (282.63,102.98) -- (287.13,107.48);

\path[draw=drawColor,line width= 0.4pt,line join=round,line cap=round] (282.63,107.48) -- (287.13,102.98);

\path[draw=drawColor,line width= 0.4pt,line join=round,line cap=round] (286.22,100.62) -- (290.72,105.12);

\path[draw=drawColor,line width= 0.4pt,line join=round,line cap=round] (286.22,105.12) -- (290.72,100.62);

\path[draw=drawColor,line width= 0.4pt,line join=round,line cap=round] (289.80, 99.43) -- (294.30,103.93);

\path[draw=drawColor,line width= 0.4pt,line join=round,line cap=round] (289.80,103.93) -- (294.30, 99.43);

\path[draw=drawColor,line width= 0.4pt,line join=round,line cap=round] (294.33, 99.67) -- (298.83,104.17);

\path[draw=drawColor,line width= 0.4pt,line join=round,line cap=round] (294.33,104.17) -- (298.83, 99.67);

\path[draw=drawColor,line width= 0.4pt,line join=round,line cap=round] (299.49,102.39) -- (303.99,106.89);

\path[draw=drawColor,line width= 0.4pt,line join=round,line cap=round] (299.49,106.89) -- (303.99,102.39);

\path[draw=drawColor,line width= 0.4pt,line join=round,line cap=round] (306.17,109.83) -- (310.67,114.33);

\path[draw=drawColor,line width= 0.4pt,line join=round,line cap=round] (306.17,114.33) -- (310.67,109.83);

\path[draw=drawColor,line width= 0.4pt,line join=round,line cap=round] (311.46,118.71) -- (315.96,123.21);

\path[draw=drawColor,line width= 0.4pt,line join=round,line cap=round] (311.46,123.21) -- (315.96,118.71);

\path[draw=drawColor,line width= 0.4pt,line join=round,line cap=round] (316.74,130.04) -- (321.24,134.54);

\path[draw=drawColor,line width= 0.4pt,line join=round,line cap=round] (316.74,134.54) -- (321.24,130.04);

\path[draw=drawColor,line width= 0.4pt,line join=round,line cap=round] (322.03,143.50) -- (326.53,148.00);

\path[draw=drawColor,line width= 0.4pt,line join=round,line cap=round] (322.03,148.00) -- (326.53,143.50);

\path[draw=drawColor,line width= 0.4pt,line join=round,line cap=round] (328.29,161.63) -- (332.79,166.13);

\path[draw=drawColor,line width= 0.4pt,line join=round,line cap=round] (328.29,166.13) -- (332.79,161.63);

\path[draw=drawColor,line width= 0.4pt,line join=round,line cap=round] (332.67,175.33) -- (337.17,179.83);

\path[draw=drawColor,line width= 0.4pt,line join=round,line cap=round] (332.67,179.83) -- (337.17,175.33);

\path[draw=drawColor,line width= 0.4pt,line join=round,line cap=round] (337.04,189.54) -- (341.54,194.04);

\path[draw=drawColor,line width= 0.4pt,line join=round,line cap=round] (337.04,194.04) -- (341.54,189.54);

\path[draw=drawColor,line width= 0.4pt,line join=round,line cap=round] (342.52,207.49) -- (347.02,211.99);

\path[draw=drawColor,line width= 0.4pt,line join=round,line cap=round] (342.52,211.99) -- (347.02,207.49);

\path[draw=drawColor,line width= 0.4pt,line join=round,line cap=round] (349.45,229.53) -- (353.95,234.03);

\path[draw=drawColor,line width= 0.4pt,line join=round,line cap=round] (349.45,234.03) -- (353.95,229.53);

\path[draw=drawColor,line width= 0.4pt,line join=round,line cap=round] (354.93,245.57) -- (359.43,250.07);

\path[draw=drawColor,line width= 0.4pt,line join=round,line cap=round] (354.93,250.07) -- (359.43,245.57);

\path[draw=drawColor,line width= 0.4pt,line join=round,line cap=round] (360.41,259.67) -- (364.91,264.17);

\path[draw=drawColor,line width= 0.4pt,line join=round,line cap=round] (360.41,264.17) -- (364.91,259.67);

\path[draw=drawColor,line width= 0.4pt,line join=round,line cap=round] (365.89,271.22) -- (370.39,275.72);

\path[draw=drawColor,line width= 0.4pt,line join=round,line cap=round] (365.89,275.72) -- (370.39,271.22);

\path[draw=drawColor,line width= 0.4pt,line join=round,line cap=round] (371.98,280.42) -- (376.48,284.92);

\path[draw=drawColor,line width= 0.4pt,line join=round,line cap=round] (371.98,284.92) -- (376.48,280.42);

\path[draw=drawColor,line width= 0.4pt,line join=round,line cap=round] (376.29,284.30) -- (380.79,288.80);

\path[draw=drawColor,line width= 0.4pt,line join=round,line cap=round] (376.29,288.80) -- (380.79,284.30);

\path[draw=drawColor,line width= 0.4pt,line join=round,line cap=round] (380.60,285.83) -- (385.10,290.33);

\path[draw=drawColor,line width= 0.4pt,line join=round,line cap=round] (380.60,290.33) -- (385.10,285.83);

\path[draw=drawColor,line width= 0.4pt,line join=round,line cap=round] (385.70,284.48) -- (390.20,288.98);

\path[draw=drawColor,line width= 0.4pt,line join=round,line cap=round] (385.70,288.98) -- (390.20,284.48);

\path[draw=drawColor,line width= 0.4pt,line join=round,line cap=round] (392.15,277.76) -- (396.65,282.26);

\path[draw=drawColor,line width= 0.4pt,line join=round,line cap=round] (392.15,282.26) -- (396.65,277.76);

\path[draw=drawColor,line width= 0.4pt,line join=round,line cap=round] (398.01,266.94) -- (402.51,271.44);

\path[draw=drawColor,line width= 0.4pt,line join=round,line cap=round] (398.01,271.44) -- (402.51,266.94);

\path[draw=drawColor,line width= 0.4pt,line join=round,line cap=round] (403.87,251.93) -- (408.37,256.43);

\path[draw=drawColor,line width= 0.4pt,line join=round,line cap=round] (403.87,256.43) -- (408.37,251.93);

\path[draw=drawColor,line width= 0.4pt,line join=round,line cap=round] (409.72,233.23) -- (414.22,237.73);

\path[draw=drawColor,line width= 0.4pt,line join=round,line cap=round] (409.72,237.73) -- (414.22,233.23);

\path[draw=drawColor,line width= 0.4pt,line join=round,line cap=round] (415.58,211.53) -- (420.08,216.03);

\path[draw=drawColor,line width= 0.4pt,line join=round,line cap=round] (415.58,216.03) -- (420.08,211.53);

\path[draw=drawColor,line width= 0.4pt,line join=round,line cap=round] (420.70,190.72) -- (425.20,195.22);

\path[draw=drawColor,line width= 0.4pt,line join=round,line cap=round] (420.70,195.22) -- (425.20,190.72);
\definecolor{fillColor}{RGB}{255,255,255}

\path[draw=drawColor,line width= 0.4pt,line join=round,line cap=round,fill=fillColor] ( 49.20,297.70) rectangle (188.40,237.70);

\path[draw=drawColor,line width= 0.4pt,line join=round,line cap=round] ( 51.90,273.70) -- ( 69.90,273.70);

\path[draw=drawColor,line width= 0.4pt,dash pattern=on 4pt off 4pt ,line join=round,line cap=round] ( 51.90,261.70) -- ( 69.90,261.70);

\path[draw=drawColor,line width= 0.4pt,line join=round,line cap=round] ( 60.90,285.70) circle (  2.25);

\path[draw=drawColor,line width= 0.4pt,line join=round,line cap=round] ( 58.65,247.45) -- ( 63.15,251.95);

\path[draw=drawColor,line width= 0.4pt,line join=round,line cap=round] ( 58.65,251.95) -- ( 63.15,247.45);

\node[text=drawColor,anchor=base west,inner sep=0pt, outer sep=0pt, scale=  1.00] at ( 78.90,282.25) {approximation};

\node[text=drawColor,anchor=base west,inner sep=0pt, outer sep=0pt, scale=  1.00] at ( 78.90,270.25) {exact};

\node[text=drawColor,anchor=base west,inner sep=0pt, outer sep=0pt, scale=  1.00] at ( 78.90,258.25) {finite differences};

\node[text=drawColor,anchor=base west,inner sep=0pt, outer sep=0pt, scale=  1.00] at ( 78.90,246.25) {Cauchy int. formula};
\end{scope}
\end{tikzpicture}

  \caption{Comparison between the (scalar) approximate sensitivity and the
    analytical sensitivity solution
    $dy(t;k)/dk$ \label{fig:dampedGSLvsASolSensitivity} }
\end{figure}

\subsection{Damping}
\label{sec:damping}

In the case of a damped oscillation, we continue to use the analytical
solution and calculate the same error measures as in the previous
section. Figure~\ref{fig:dampedGSLvsASol} shows the solution accuracy
(disregarding sensitivities).
\begin{figure}\sffamily\firalining
  \centering
  % Created by tikzDevice version 0.12.3.1 on 2021-06-16 15:27:58
% !TEX encoding = UTF-8 Unicode
\begin{tikzpicture}[x=1pt,y=1pt]
\definecolor{fillColor}{RGB}{255,255,255}
\path[use as bounding box,fill=fillColor,fill opacity=0.00] (0,0) rectangle (462.53,346.90);
\begin{scope}
\path[clip] ( 49.20, 61.20) rectangle (437.33,297.70);
\definecolor{drawColor}{RGB}{0,0,0}

\path[draw=drawColor,line width= 0.4pt,line join=round,line cap=round] ( 63.58,278.98) --
	( 63.60,278.98) --
	( 63.63,278.98) --
	( 63.66,278.98) --
	( 63.70,278.98) --
	( 63.73,278.98) --
	( 63.76,278.98) --
	( 63.79,278.98) --
	( 63.84,278.98) --
	( 63.91,278.97) --
	( 64.04,278.97) --
	( 64.27,278.95) --
	( 64.77,278.88) --
	( 65.18,278.79) --
	( 65.60,278.68) --
	( 66.02,278.54) --
	( 66.44,278.38) --
	( 66.86,278.20) --
	( 67.28,277.99) --
	( 67.70,277.77) --
	( 68.22,277.46) --
	( 69.00,276.93) --
	( 70.92,275.39) --
	( 72.46,273.90) --
	( 74.01,272.21) --
	( 75.55,270.35) --
	( 77.22,268.19) --
	( 78.88,265.88) --
	( 80.55,263.45) --
	( 82.21,260.92) --
	( 84.02,258.10) --
	( 85.83,255.21) --
	( 87.63,252.28) --
	( 89.44,249.32) --
	( 91.25,246.35) --
	( 93.05,243.39) --
	( 95.00,240.22) --
	( 96.95,237.09) --
	( 98.89,234.02) --
	(100.84,231.00) --
	(103.01,227.72) --
	(106.14,223.18) --
	(109.94,217.98) --
	(113.19,213.82) --
	(116.45,209.94) --
	(119.81,206.23) --
	(123.68,202.34) --
	(127.77,198.65) --
	(131.31,195.81) --
	(134.84,193.27) --
	(138.37,191.01) --
	(142.35,188.78) --
	(146.61,186.74) --
	(150.86,185.01) --
	(155.11,183.56) --
	(159.37,182.36) --
	(163.62,181.39) --
	(168.05,180.57) --
	(172.71,179.89) --
	(177.38,179.39) --
	(182.04,179.01) --
	(187.03,178.74) --
	(192.57,178.55) --
	(198.66,178.44) --
	(204.76,178.42) --
	(210.85,178.46) --
	(216.94,178.53) --
	(223.04,178.62) --
	(229.13,178.72) --
	(235.23,178.83) --
	(241.32,178.92) --
	(247.41,179.02) --
	(253.51,179.10) --
	(259.89,179.17) --
	(266.67,179.24) --
	(273.77,179.30) --
	(280.87,179.34) --
	(288.35,179.38) --
	(296.52,179.41) --
	(305.61,179.43) --
	(315.56,179.45) --
	(325.94,179.45) --
	(336.72,179.46) --
	(348.51,179.46) --
	(363.00,179.46) --
	(377.99,179.45) --
	(392.98,179.45) --
	(407.96,179.45) --
	(422.95,179.45);
\end{scope}
\begin{scope}
\path[clip] (  0.00,  0.00) rectangle (462.53,346.90);
\definecolor{drawColor}{RGB}{0,0,0}

\path[draw=drawColor,line width= 0.4pt,line join=round,line cap=round] ( 63.58, 61.20) -- (369.49, 61.20);

\path[draw=drawColor,line width= 0.4pt,line join=round,line cap=round] ( 63.58, 61.20) -- ( 63.58, 55.20);

\path[draw=drawColor,line width= 0.4pt,line join=round,line cap=round] (140.06, 61.20) -- (140.06, 55.20);

\path[draw=drawColor,line width= 0.4pt,line join=round,line cap=round] (216.54, 61.20) -- (216.54, 55.20);

\path[draw=drawColor,line width= 0.4pt,line join=round,line cap=round] (293.02, 61.20) -- (293.02, 55.20);

\path[draw=drawColor,line width= 0.4pt,line join=round,line cap=round] (369.49, 61.20) -- (369.49, 55.20);

\node[text=drawColor,anchor=base,inner sep=0pt, outer sep=0pt, scale=  1.00] at ( 63.58, 39.60) {0};

\node[text=drawColor,anchor=base,inner sep=0pt, outer sep=0pt, scale=  1.00] at (140.06, 39.60) {5};

\node[text=drawColor,anchor=base,inner sep=0pt, outer sep=0pt, scale=  1.00] at (216.54, 39.60) {10};

\node[text=drawColor,anchor=base,inner sep=0pt, outer sep=0pt, scale=  1.00] at (293.02, 39.60) {15};

\node[text=drawColor,anchor=base,inner sep=0pt, outer sep=0pt, scale=  1.00] at (369.49, 39.60) {20};

\path[draw=drawColor,line width= 0.4pt,line join=round,line cap=round] ( 49.20, 79.91) -- ( 49.20,278.98);

\path[draw=drawColor,line width= 0.4pt,line join=round,line cap=round] ( 49.20, 79.91) -- ( 43.20, 79.91);

\path[draw=drawColor,line width= 0.4pt,line join=round,line cap=round] ( 49.20,129.68) -- ( 43.20,129.68);

\path[draw=drawColor,line width= 0.4pt,line join=round,line cap=round] ( 49.20,179.45) -- ( 43.20,179.45);

\path[draw=drawColor,line width= 0.4pt,line join=round,line cap=round] ( 49.20,229.22) -- ( 43.20,229.22);

\path[draw=drawColor,line width= 0.4pt,line join=round,line cap=round] ( 49.20,278.98) -- ( 43.20,278.98);

\node[text=drawColor,rotate= 90.00,anchor=base,inner sep=0pt, outer sep=0pt, scale=  1.00] at ( 34.80, 79.91) {-1.0};

\node[text=drawColor,rotate= 90.00,anchor=base,inner sep=0pt, outer sep=0pt, scale=  1.00] at ( 34.80,129.68) {-0.5};

\node[text=drawColor,rotate= 90.00,anchor=base,inner sep=0pt, outer sep=0pt, scale=  1.00] at ( 34.80,179.45) {0.0};

\node[text=drawColor,rotate= 90.00,anchor=base,inner sep=0pt, outer sep=0pt, scale=  1.00] at ( 34.80,229.22) {0.5};

\node[text=drawColor,rotate= 90.00,anchor=base,inner sep=0pt, outer sep=0pt, scale=  1.00] at ( 34.80,278.98) {1.0};

\path[draw=drawColor,line width= 0.4pt,line join=round,line cap=round] ( 49.20, 61.20) --
	(437.33, 61.20) --
	(437.33,297.70) --
	( 49.20,297.70) --
	( 49.20, 61.20);
\end{scope}
\begin{scope}
\path[clip] (  0.00,  0.00) rectangle (462.53,346.90);
\definecolor{drawColor}{RGB}{0,0,0}

\node[text=drawColor,anchor=base,inner sep=0pt, outer sep=0pt, scale=  1.20] at (243.26,318.16) {\bfseries  with damping but no driving force};

\node[text=drawColor,anchor=base,inner sep=0pt, outer sep=0pt, scale=  1.00] at (243.26,  3.60) {analytical solution (missmatch: 1.3e-05)};

\node[text=drawColor,anchor=base,inner sep=0pt, outer sep=0pt, scale=  1.00] at (243.26, 15.60) {$t$};

\node[text=drawColor,rotate= 90.00,anchor=base,inner sep=0pt, outer sep=0pt, scale=  1.00] at ( 10.80,179.45) {state variables $(v,y)$};
\end{scope}
\begin{scope}
\path[clip] ( 49.20, 61.20) rectangle (437.33,297.70);
\definecolor{drawColor}{RGB}{0,0,0}

\path[draw=drawColor,line width= 0.4pt,line join=round,line cap=round] ( 63.58,278.98) circle (  2.25);

\path[draw=drawColor,line width= 0.4pt,line join=round,line cap=round] ( 63.60,278.98) circle (  2.25);

\path[draw=drawColor,line width= 0.4pt,line join=round,line cap=round] ( 63.63,278.98) circle (  2.25);

\path[draw=drawColor,line width= 0.4pt,line join=round,line cap=round] ( 63.66,278.98) circle (  2.25);

\path[draw=drawColor,line width= 0.4pt,line join=round,line cap=round] ( 63.70,278.98) circle (  2.25);

\path[draw=drawColor,line width= 0.4pt,line join=round,line cap=round] ( 63.73,278.98) circle (  2.25);

\path[draw=drawColor,line width= 0.4pt,line join=round,line cap=round] ( 63.76,278.98) circle (  2.25);

\path[draw=drawColor,line width= 0.4pt,line join=round,line cap=round] ( 63.79,278.98) circle (  2.25);

\path[draw=drawColor,line width= 0.4pt,line join=round,line cap=round] ( 63.84,278.98) circle (  2.25);

\path[draw=drawColor,line width= 0.4pt,line join=round,line cap=round] ( 63.91,278.97) circle (  2.25);

\path[draw=drawColor,line width= 0.4pt,line join=round,line cap=round] ( 64.04,278.97) circle (  2.25);

\path[draw=drawColor,line width= 0.4pt,line join=round,line cap=round] ( 64.27,278.95) circle (  2.25);

\path[draw=drawColor,line width= 0.4pt,line join=round,line cap=round] ( 64.77,278.87) circle (  2.25);

\path[draw=drawColor,line width= 0.4pt,line join=round,line cap=round] ( 65.18,278.79) circle (  2.25);

\path[draw=drawColor,line width= 0.4pt,line join=round,line cap=round] ( 65.60,278.68) circle (  2.25);

\path[draw=drawColor,line width= 0.4pt,line join=round,line cap=round] ( 66.02,278.54) circle (  2.25);

\path[draw=drawColor,line width= 0.4pt,line join=round,line cap=round] ( 66.44,278.38) circle (  2.25);

\path[draw=drawColor,line width= 0.4pt,line join=round,line cap=round] ( 66.86,278.20) circle (  2.25);

\path[draw=drawColor,line width= 0.4pt,line join=round,line cap=round] ( 67.28,278.00) circle (  2.25);

\path[draw=drawColor,line width= 0.4pt,line join=round,line cap=round] ( 67.70,277.77) circle (  2.25);

\path[draw=drawColor,line width= 0.4pt,line join=round,line cap=round] ( 68.22,277.46) circle (  2.25);

\path[draw=drawColor,line width= 0.4pt,line join=round,line cap=round] ( 69.00,276.94) circle (  2.25);

\path[draw=drawColor,line width= 0.4pt,line join=round,line cap=round] ( 70.92,275.39) circle (  2.25);

\path[draw=drawColor,line width= 0.4pt,line join=round,line cap=round] ( 72.46,273.90) circle (  2.25);

\path[draw=drawColor,line width= 0.4pt,line join=round,line cap=round] ( 74.01,272.21) circle (  2.25);

\path[draw=drawColor,line width= 0.4pt,line join=round,line cap=round] ( 75.55,270.35) circle (  2.25);

\path[draw=drawColor,line width= 0.4pt,line join=round,line cap=round] ( 77.22,268.19) circle (  2.25);

\path[draw=drawColor,line width= 0.4pt,line join=round,line cap=round] ( 78.88,265.88) circle (  2.25);

\path[draw=drawColor,line width= 0.4pt,line join=round,line cap=round] ( 80.55,263.45) circle (  2.25);

\path[draw=drawColor,line width= 0.4pt,line join=round,line cap=round] ( 82.21,260.92) circle (  2.25);

\path[draw=drawColor,line width= 0.4pt,line join=round,line cap=round] ( 84.02,258.10) circle (  2.25);

\path[draw=drawColor,line width= 0.4pt,line join=round,line cap=round] ( 85.83,255.21) circle (  2.25);

\path[draw=drawColor,line width= 0.4pt,line join=round,line cap=round] ( 87.63,252.28) circle (  2.25);

\path[draw=drawColor,line width= 0.4pt,line join=round,line cap=round] ( 89.44,249.32) circle (  2.25);

\path[draw=drawColor,line width= 0.4pt,line join=round,line cap=round] ( 91.25,246.35) circle (  2.25);

\path[draw=drawColor,line width= 0.4pt,line join=round,line cap=round] ( 93.05,243.39) circle (  2.25);

\path[draw=drawColor,line width= 0.4pt,line join=round,line cap=round] ( 95.00,240.22) circle (  2.25);

\path[draw=drawColor,line width= 0.4pt,line join=round,line cap=round] ( 96.95,237.09) circle (  2.25);

\path[draw=drawColor,line width= 0.4pt,line join=round,line cap=round] ( 98.89,234.01) circle (  2.25);

\path[draw=drawColor,line width= 0.4pt,line join=round,line cap=round] (100.84,231.00) circle (  2.25);

\path[draw=drawColor,line width= 0.4pt,line join=round,line cap=round] (103.01,227.72) circle (  2.25);

\path[draw=drawColor,line width= 0.4pt,line join=round,line cap=round] (106.14,223.18) circle (  2.25);

\path[draw=drawColor,line width= 0.4pt,line join=round,line cap=round] (109.94,217.97) circle (  2.25);

\path[draw=drawColor,line width= 0.4pt,line join=round,line cap=round] (113.19,213.82) circle (  2.25);

\path[draw=drawColor,line width= 0.4pt,line join=round,line cap=round] (116.45,209.94) circle (  2.25);

\path[draw=drawColor,line width= 0.4pt,line join=round,line cap=round] (119.81,206.23) circle (  2.25);

\path[draw=drawColor,line width= 0.4pt,line join=round,line cap=round] (123.68,202.34) circle (  2.25);

\path[draw=drawColor,line width= 0.4pt,line join=round,line cap=round] (127.77,198.66) circle (  2.25);

\path[draw=drawColor,line width= 0.4pt,line join=round,line cap=round] (131.31,195.82) circle (  2.25);

\path[draw=drawColor,line width= 0.4pt,line join=round,line cap=round] (134.84,193.27) circle (  2.25);

\path[draw=drawColor,line width= 0.4pt,line join=round,line cap=round] (138.37,191.01) circle (  2.25);

\path[draw=drawColor,line width= 0.4pt,line join=round,line cap=round] (142.35,188.79) circle (  2.25);

\path[draw=drawColor,line width= 0.4pt,line join=round,line cap=round] (146.61,186.74) circle (  2.25);

\path[draw=drawColor,line width= 0.4pt,line join=round,line cap=round] (150.86,185.01) circle (  2.25);

\path[draw=drawColor,line width= 0.4pt,line join=round,line cap=round] (155.11,183.56) circle (  2.25);

\path[draw=drawColor,line width= 0.4pt,line join=round,line cap=round] (159.37,182.37) circle (  2.25);

\path[draw=drawColor,line width= 0.4pt,line join=round,line cap=round] (163.62,181.39) circle (  2.25);

\path[draw=drawColor,line width= 0.4pt,line join=round,line cap=round] (168.05,180.57) circle (  2.25);

\path[draw=drawColor,line width= 0.4pt,line join=round,line cap=round] (172.71,179.90) circle (  2.25);

\path[draw=drawColor,line width= 0.4pt,line join=round,line cap=round] (177.38,179.39) circle (  2.25);

\path[draw=drawColor,line width= 0.4pt,line join=round,line cap=round] (182.04,179.01) circle (  2.25);

\path[draw=drawColor,line width= 0.4pt,line join=round,line cap=round] (187.03,178.74) circle (  2.25);

\path[draw=drawColor,line width= 0.4pt,line join=round,line cap=round] (192.57,178.54) circle (  2.25);

\path[draw=drawColor,line width= 0.4pt,line join=round,line cap=round] (198.66,178.44) circle (  2.25);

\path[draw=drawColor,line width= 0.4pt,line join=round,line cap=round] (204.76,178.42) circle (  2.25);

\path[draw=drawColor,line width= 0.4pt,line join=round,line cap=round] (210.85,178.46) circle (  2.25);

\path[draw=drawColor,line width= 0.4pt,line join=round,line cap=round] (216.94,178.53) circle (  2.25);

\path[draw=drawColor,line width= 0.4pt,line join=round,line cap=round] (223.04,178.62) circle (  2.25);

\path[draw=drawColor,line width= 0.4pt,line join=round,line cap=round] (229.13,178.72) circle (  2.25);

\path[draw=drawColor,line width= 0.4pt,line join=round,line cap=round] (235.23,178.82) circle (  2.25);

\path[draw=drawColor,line width= 0.4pt,line join=round,line cap=round] (241.32,178.92) circle (  2.25);

\path[draw=drawColor,line width= 0.4pt,line join=round,line cap=round] (247.41,179.02) circle (  2.25);

\path[draw=drawColor,line width= 0.4pt,line join=round,line cap=round] (253.51,179.10) circle (  2.25);

\path[draw=drawColor,line width= 0.4pt,line join=round,line cap=round] (259.89,179.17) circle (  2.25);

\path[draw=drawColor,line width= 0.4pt,line join=round,line cap=round] (266.67,179.24) circle (  2.25);

\path[draw=drawColor,line width= 0.4pt,line join=round,line cap=round] (273.77,179.30) circle (  2.25);

\path[draw=drawColor,line width= 0.4pt,line join=round,line cap=round] (280.87,179.34) circle (  2.25);

\path[draw=drawColor,line width= 0.4pt,line join=round,line cap=round] (288.35,179.38) circle (  2.25);

\path[draw=drawColor,line width= 0.4pt,line join=round,line cap=round] (296.52,179.41) circle (  2.25);

\path[draw=drawColor,line width= 0.4pt,line join=round,line cap=round] (305.61,179.43) circle (  2.25);

\path[draw=drawColor,line width= 0.4pt,line join=round,line cap=round] (315.56,179.45) circle (  2.25);

\path[draw=drawColor,line width= 0.4pt,line join=round,line cap=round] (325.94,179.45) circle (  2.25);

\path[draw=drawColor,line width= 0.4pt,line join=round,line cap=round] (336.72,179.46) circle (  2.25);

\path[draw=drawColor,line width= 0.4pt,line join=round,line cap=round] (348.51,179.46) circle (  2.25);

\path[draw=drawColor,line width= 0.4pt,line join=round,line cap=round] (363.00,179.46) circle (  2.25);

\path[draw=drawColor,line width= 0.4pt,line join=round,line cap=round] (377.99,179.45) circle (  2.25);

\path[draw=drawColor,line width= 0.4pt,line join=round,line cap=round] (392.98,179.45) circle (  2.25);

\path[draw=drawColor,line width= 0.4pt,line join=round,line cap=round] (407.96,179.45) circle (  2.25);

\path[draw=drawColor,line width= 0.4pt,line join=round,line cap=round] (422.95,179.45) circle (  2.25);

\path[draw=drawColor,line width= 0.4pt,dash pattern=on 1pt off 3pt ,line join=round,line cap=round] ( 63.58,179.45) --
	( 63.60,179.39) --
	( 63.63,179.32) --
	( 63.66,179.24) --
	( 63.70,179.16) --
	( 63.73,179.09) --
	( 63.76,179.01) --
	( 63.79,178.93) --
	( 63.84,178.83) --
	( 63.91,178.65) --
	( 64.04,178.36) --
	( 64.27,177.83) --
	( 64.77,176.71) --
	( 65.18,175.79) --
	( 65.60,174.91) --
	( 66.02,174.04) --
	( 66.44,173.21) --
	( 66.86,172.39) --
	( 67.28,171.60) --
	( 67.70,170.84) --
	( 68.22,169.92) --
	( 69.00,168.59) --
	( 70.92,165.68) --
	( 72.46,163.64) --
	( 74.01,161.85) --
	( 75.55,160.30) --
	( 77.22,158.87) --
	( 78.88,157.67) --
	( 80.55,156.68) --
	( 82.21,155.89) --
	( 84.02,155.23) --
	( 85.83,154.77) --
	( 87.63,154.48) --
	( 89.44,154.33) --
	( 91.25,154.33) --
	( 93.05,154.45) --
	( 95.00,154.70) --
	( 96.95,155.05) --
	( 98.89,155.51) --
	(100.84,156.04) --
	(103.01,156.71) --
	(106.14,157.79) --
	(109.94,159.24) --
	(113.19,160.55) --
	(116.45,161.89) --
	(119.81,163.28) --
	(123.68,164.86) --
	(127.77,166.48) --
	(131.31,167.81) --
	(134.84,169.07) --
	(138.37,170.25) --
	(142.35,171.48) --
	(146.61,172.68) --
	(150.86,173.76) --
	(155.11,174.71) --
	(159.37,175.55) --
	(163.62,176.28) --
	(168.05,176.94) --
	(172.71,177.53) --
	(177.38,178.02) --
	(182.04,178.42) --
	(187.03,178.77) --
	(192.57,179.06) --
	(198.66,179.31) --
	(204.76,179.48) --
	(210.85,179.59) --
	(216.94,179.66) --
	(223.04,179.69) --
	(229.13,179.71) --
	(235.23,179.70) --
	(241.32,179.69) --
	(247.41,179.67) --
	(253.51,179.64) --
	(259.89,179.61) --
	(266.67,179.59) --
	(273.77,179.56) --
	(280.87,179.53) --
	(288.35,179.51) --
	(296.52,179.49) --
	(305.61,179.48) --
	(315.56,179.46) --
	(325.94,179.46) --
	(336.72,179.45) --
	(348.51,179.45) --
	(363.00,179.45) --
	(377.99,179.45) --
	(392.98,179.45) --
	(407.96,179.45) --
	(422.95,179.45);
\definecolor{fillColor}{RGB}{255,255,255}

\path[draw=drawColor,line width= 0.4pt,line join=round,line cap=round,fill=fillColor] (334.80,297.70) rectangle (437.33,249.70);

\path[draw=drawColor,line width= 0.4pt,line join=round,line cap=round] (337.50,285.70) -- (355.50,285.70);

\path[draw=drawColor,line width= 0.4pt,dash pattern=on 1pt off 3pt ,line join=round,line cap=round] (337.50,261.70) -- (355.50,261.70);

\path[draw=drawColor,line width= 0.4pt,line join=round,line cap=round] (346.50,273.70) circle (  2.25);

\node[text=drawColor,anchor=base west,inner sep=0pt, outer sep=0pt, scale=  1.00] at (364.50,282.25) {$y$ exact};

\node[text=drawColor,anchor=base west,inner sep=0pt, outer sep=0pt, scale=  1.00] at (364.50,270.25) {gsl odeiv $y(t;k)$};

\node[text=drawColor,anchor=base west,inner sep=0pt, outer sep=0pt, scale=  1.00] at (364.50,258.25) {gsl $v(t;k)$};
\end{scope}
\end{tikzpicture}

  \caption{Comparison of the solution obatined using the solvers in
    the \textsc{gsl} to the analytical
    solution.  \label{fig:dampedGSLvsASol}}
\end{figure}
Figure~\ref{fig:dampedlinerr} shows the linearization error, as
described in Section~\ref{sec:results}.
\begin{figure}\sffamily\firalining
  \centering
  % Created by tikzDevice version 0.12.3.1 on 2021-06-16 15:27:58
% !TEX encoding = UTF-8 Unicode
\begin{tikzpicture}[x=1pt,y=1pt]
\definecolor{fillColor}{RGB}{255,255,255}
\path[use as bounding box,fill=fillColor,fill opacity=0.00] (0,0) rectangle (462.53,346.90);
\begin{scope}
\path[clip] ( 49.20, 61.20) rectangle (437.33,297.70);
\definecolor{drawColor}{RGB}{0,0,0}

\path[draw=drawColor,line width= 0.4pt,line join=round,line cap=round] ( 63.58,288.94) --
	( 63.60,288.94) --
	( 63.63,288.94) --
	( 63.66,288.94) --
	( 63.70,288.93) --
	( 63.73,288.93) --
	( 63.76,288.93) --
	( 63.79,288.93) --
	( 63.84,288.92) --
	( 63.91,288.92) --
	( 64.04,288.90) --
	( 64.27,288.85) --
	( 64.77,288.67) --
	( 65.18,288.46) --
	( 65.60,288.18) --
	( 66.02,287.85) --
	( 66.44,287.46) --
	( 66.86,287.02) --
	( 67.28,286.52) --
	( 67.70,285.97) --
	( 68.22,285.21) --
	( 69.00,283.93) --
	( 70.92,280.16) --
	( 72.46,276.51) --
	( 74.01,272.40) --
	( 75.55,267.88) --
	( 77.22,262.61) --
	( 78.88,257.00) --
	( 80.55,251.12) --
	( 82.21,245.01) --
	( 84.02,238.20) --
	( 85.83,231.25) --
	( 87.63,224.21) --
	( 89.44,217.14) --
	( 91.25,210.07) --
	( 93.05,203.04) --
	( 95.00,195.55) --
	( 96.95,188.19) --
	( 98.89,180.99) --
	(100.84,173.97) --
	(103.01,166.38) --
	(106.14,155.97) --
	(109.94,144.18) --
	(113.19,134.89) --
	(116.45,126.35) --
	(119.81,118.31) --
	(123.68,110.06) --
	(127.77,102.41) --
	(131.31, 96.68) --
	(134.84, 91.69) --
	(138.37, 87.39) --
	(142.35, 83.30) --
	(146.61, 79.72) --
	(150.86, 76.87) --
	(155.11, 74.64) --
	(159.37, 72.96) --
	(163.62, 71.72) --
	(168.05, 70.85) --
	(172.71, 70.28) --
	(177.38, 70.01) --
	(182.04, 69.96) --
	(187.03, 70.09) --
	(192.57, 70.39) --
	(198.66, 70.84) --
	(204.76, 71.34) --
	(210.85, 71.85) --
	(216.94, 72.35) --
	(223.04, 72.80) --
	(229.13, 73.20) --
	(235.23, 73.55) --
	(241.32, 73.84) --
	(247.41, 74.07) --
	(253.51, 74.26) --
	(259.89, 74.41) --
	(266.67, 74.52) --
	(273.77, 74.60) --
	(280.87, 74.65) --
	(288.35, 74.68) --
	(296.52, 74.69) --
	(305.61, 74.69) --
	(315.56, 74.67) --
	(325.94, 74.65) --
	(336.72, 74.63) --
	(348.51, 74.62) --
	(363.00, 74.60) --
	(377.99, 74.59) --
	(392.98, 74.59) --
	(407.96, 74.59) --
	(422.95, 74.59);
\end{scope}
\begin{scope}
\path[clip] (  0.00,  0.00) rectangle (462.53,346.90);
\definecolor{drawColor}{RGB}{0,0,0}

\path[draw=drawColor,line width= 0.4pt,line join=round,line cap=round] ( 63.58, 61.20) -- (369.49, 61.20);

\path[draw=drawColor,line width= 0.4pt,line join=round,line cap=round] ( 63.58, 61.20) -- ( 63.58, 55.20);

\path[draw=drawColor,line width= 0.4pt,line join=round,line cap=round] (140.06, 61.20) -- (140.06, 55.20);

\path[draw=drawColor,line width= 0.4pt,line join=round,line cap=round] (216.54, 61.20) -- (216.54, 55.20);

\path[draw=drawColor,line width= 0.4pt,line join=round,line cap=round] (293.02, 61.20) -- (293.02, 55.20);

\path[draw=drawColor,line width= 0.4pt,line join=round,line cap=round] (369.49, 61.20) -- (369.49, 55.20);

\node[text=drawColor,anchor=base,inner sep=0pt, outer sep=0pt, scale=  1.00] at ( 63.58, 39.60) {0};

\node[text=drawColor,anchor=base,inner sep=0pt, outer sep=0pt, scale=  1.00] at (140.06, 39.60) {5};

\node[text=drawColor,anchor=base,inner sep=0pt, outer sep=0pt, scale=  1.00] at (216.54, 39.60) {10};

\node[text=drawColor,anchor=base,inner sep=0pt, outer sep=0pt, scale=  1.00] at (293.02, 39.60) {15};

\node[text=drawColor,anchor=base,inner sep=0pt, outer sep=0pt, scale=  1.00] at (369.49, 39.60) {20};

\path[draw=drawColor,line width= 0.4pt,line join=round,line cap=round] ( 49.20, 74.59) -- ( 49.20,288.94);

\path[draw=drawColor,line width= 0.4pt,line join=round,line cap=round] ( 49.20, 74.59) -- ( 43.20, 74.59);

\path[draw=drawColor,line width= 0.4pt,line join=round,line cap=round] ( 49.20,117.46) -- ( 43.20,117.46);

\path[draw=drawColor,line width= 0.4pt,line join=round,line cap=round] ( 49.20,160.33) -- ( 43.20,160.33);

\path[draw=drawColor,line width= 0.4pt,line join=round,line cap=round] ( 49.20,203.20) -- ( 43.20,203.20);

\path[draw=drawColor,line width= 0.4pt,line join=round,line cap=round] ( 49.20,246.07) -- ( 43.20,246.07);

\path[draw=drawColor,line width= 0.4pt,line join=round,line cap=round] ( 49.20,288.94) -- ( 43.20,288.94);

\node[text=drawColor,rotate= 90.00,anchor=base,inner sep=0pt, outer sep=0pt, scale=  1.00] at ( 34.80, 74.59) {0.0};

\node[text=drawColor,rotate= 90.00,anchor=base,inner sep=0pt, outer sep=0pt, scale=  1.00] at ( 34.80,117.46) {0.2};

\node[text=drawColor,rotate= 90.00,anchor=base,inner sep=0pt, outer sep=0pt, scale=  1.00] at ( 34.80,160.33) {0.4};

\node[text=drawColor,rotate= 90.00,anchor=base,inner sep=0pt, outer sep=0pt, scale=  1.00] at ( 34.80,203.20) {0.6};

\node[text=drawColor,rotate= 90.00,anchor=base,inner sep=0pt, outer sep=0pt, scale=  1.00] at ( 34.80,246.07) {0.8};

\node[text=drawColor,rotate= 90.00,anchor=base,inner sep=0pt, outer sep=0pt, scale=  1.00] at ( 34.80,288.94) {1.0};

\path[draw=drawColor,line width= 0.4pt,line join=round,line cap=round] ( 49.20, 61.20) --
	(437.33, 61.20) --
	(437.33,297.70) --
	( 49.20,297.70) --
	( 49.20, 61.20);
\end{scope}
\begin{scope}
\path[clip] (  0.00,  0.00) rectangle (462.53,346.90);
\definecolor{drawColor}{RGB}{0,0,0}

\node[text=drawColor,anchor=base,inner sep=0pt, outer sep=0pt, scale=  1.20] at (243.26,318.16) {\bfseries  with damping but no driving force at $\Delta_k = 0.05$};

\node[text=drawColor,anchor=base,inner sep=0pt, outer sep=0pt, scale=  1.00] at (243.26,  3.60) {average missmatch: $0.035 \Delta_k$};

\node[text=drawColor,anchor=base,inner sep=0pt, outer sep=0pt, scale=  1.00] at (243.26, 15.60) {x};

\node[text=drawColor,rotate= 90.00,anchor=base,inner sep=0pt, outer sep=0pt, scale=  1.00] at ( 10.80,179.45) {y};
\end{scope}
\begin{scope}
\path[clip] ( 49.20, 61.20) rectangle (437.33,297.70);
\definecolor{drawColor}{RGB}{0,0,0}

\path[draw=drawColor,line width= 0.4pt,line join=round,line cap=round] ( 63.58,288.94) circle (  2.25);

\path[draw=drawColor,line width= 0.4pt,line join=round,line cap=round] ( 63.60,288.94) circle (  2.25);

\path[draw=drawColor,line width= 0.4pt,line join=round,line cap=round] ( 63.63,288.94) circle (  2.25);

\path[draw=drawColor,line width= 0.4pt,line join=round,line cap=round] ( 63.66,288.93) circle (  2.25);

\path[draw=drawColor,line width= 0.4pt,line join=round,line cap=round] ( 63.70,288.93) circle (  2.25);

\path[draw=drawColor,line width= 0.4pt,line join=round,line cap=round] ( 63.73,288.93) circle (  2.25);

\path[draw=drawColor,line width= 0.4pt,line join=round,line cap=round] ( 63.76,288.93) circle (  2.25);

\path[draw=drawColor,line width= 0.4pt,line join=round,line cap=round] ( 63.79,288.93) circle (  2.25);

\path[draw=drawColor,line width= 0.4pt,line join=round,line cap=round] ( 63.84,288.92) circle (  2.25);

\path[draw=drawColor,line width= 0.4pt,line join=round,line cap=round] ( 63.91,288.91) circle (  2.25);

\path[draw=drawColor,line width= 0.4pt,line join=round,line cap=round] ( 64.04,288.90) circle (  2.25);

\path[draw=drawColor,line width= 0.4pt,line join=round,line cap=round] ( 64.27,288.85) circle (  2.25);

\path[draw=drawColor,line width= 0.4pt,line join=round,line cap=round] ( 64.77,288.67) circle (  2.25);

\path[draw=drawColor,line width= 0.4pt,line join=round,line cap=round] ( 65.18,288.46) circle (  2.25);

\path[draw=drawColor,line width= 0.4pt,line join=round,line cap=round] ( 65.60,288.18) circle (  2.25);

\path[draw=drawColor,line width= 0.4pt,line join=round,line cap=round] ( 66.02,287.85) circle (  2.25);

\path[draw=drawColor,line width= 0.4pt,line join=round,line cap=round] ( 66.44,287.46) circle (  2.25);

\path[draw=drawColor,line width= 0.4pt,line join=round,line cap=round] ( 66.86,287.02) circle (  2.25);

\path[draw=drawColor,line width= 0.4pt,line join=round,line cap=round] ( 67.28,286.52) circle (  2.25);

\path[draw=drawColor,line width= 0.4pt,line join=round,line cap=round] ( 67.70,285.97) circle (  2.25);

\path[draw=drawColor,line width= 0.4pt,line join=round,line cap=round] ( 68.22,285.22) circle (  2.25);

\path[draw=drawColor,line width= 0.4pt,line join=round,line cap=round] ( 69.00,283.93) circle (  2.25);

\path[draw=drawColor,line width= 0.4pt,line join=round,line cap=round] ( 70.92,280.16) circle (  2.25);

\path[draw=drawColor,line width= 0.4pt,line join=round,line cap=round] ( 72.46,276.51) circle (  2.25);

\path[draw=drawColor,line width= 0.4pt,line join=round,line cap=round] ( 74.01,272.39) circle (  2.25);

\path[draw=drawColor,line width= 0.4pt,line join=round,line cap=round] ( 75.55,267.87) circle (  2.25);

\path[draw=drawColor,line width= 0.4pt,line join=round,line cap=round] ( 77.22,262.60) circle (  2.25);

\path[draw=drawColor,line width= 0.4pt,line join=round,line cap=round] ( 78.88,256.99) circle (  2.25);

\path[draw=drawColor,line width= 0.4pt,line join=round,line cap=round] ( 80.55,251.09) circle (  2.25);

\path[draw=drawColor,line width= 0.4pt,line join=round,line cap=round] ( 82.21,244.98) circle (  2.25);

\path[draw=drawColor,line width= 0.4pt,line join=round,line cap=round] ( 84.02,238.16) circle (  2.25);

\path[draw=drawColor,line width= 0.4pt,line join=round,line cap=round] ( 85.83,231.19) circle (  2.25);

\path[draw=drawColor,line width= 0.4pt,line join=round,line cap=round] ( 87.63,224.14) circle (  2.25);

\path[draw=drawColor,line width= 0.4pt,line join=round,line cap=round] ( 89.44,217.04) circle (  2.25);

\path[draw=drawColor,line width= 0.4pt,line join=round,line cap=round] ( 91.25,209.95) circle (  2.25);

\path[draw=drawColor,line width= 0.4pt,line join=round,line cap=round] ( 93.05,202.90) circle (  2.25);

\path[draw=drawColor,line width= 0.4pt,line join=round,line cap=round] ( 95.00,195.38) circle (  2.25);

\path[draw=drawColor,line width= 0.4pt,line join=round,line cap=round] ( 96.95,187.98) circle (  2.25);

\path[draw=drawColor,line width= 0.4pt,line join=round,line cap=round] ( 98.89,180.75) circle (  2.25);

\path[draw=drawColor,line width= 0.4pt,line join=round,line cap=round] (100.84,173.69) circle (  2.25);

\path[draw=drawColor,line width= 0.4pt,line join=round,line cap=round] (103.01,166.05) circle (  2.25);

\path[draw=drawColor,line width= 0.4pt,line join=round,line cap=round] (106.14,155.57) circle (  2.25);

\path[draw=drawColor,line width= 0.4pt,line join=round,line cap=round] (109.94,143.68) circle (  2.25);

\path[draw=drawColor,line width= 0.4pt,line join=round,line cap=round] (113.19,134.30) circle (  2.25);

\path[draw=drawColor,line width= 0.4pt,line join=round,line cap=round] (116.45,125.68) circle (  2.25);

\path[draw=drawColor,line width= 0.4pt,line join=round,line cap=round] (119.81,117.54) circle (  2.25);

\path[draw=drawColor,line width= 0.4pt,line join=round,line cap=round] (123.68,109.19) circle (  2.25);

\path[draw=drawColor,line width= 0.4pt,line join=round,line cap=round] (127.77,101.43) circle (  2.25);

\path[draw=drawColor,line width= 0.4pt,line join=round,line cap=round] (131.31, 95.60) circle (  2.25);

\path[draw=drawColor,line width= 0.4pt,line join=round,line cap=round] (134.84, 90.53) circle (  2.25);

\path[draw=drawColor,line width= 0.4pt,line join=round,line cap=round] (138.37, 86.15) circle (  2.25);

\path[draw=drawColor,line width= 0.4pt,line join=round,line cap=round] (142.35, 81.98) circle (  2.25);

\path[draw=drawColor,line width= 0.4pt,line join=round,line cap=round] (146.61, 78.34) circle (  2.25);

\path[draw=drawColor,line width= 0.4pt,line join=round,line cap=round] (150.86, 75.43) circle (  2.25);

\path[draw=drawColor,line width= 0.4pt,line join=round,line cap=round] (155.11, 73.17) circle (  2.25);

\path[draw=drawColor,line width= 0.4pt,line join=round,line cap=round] (159.37, 71.46) circle (  2.25);

\path[draw=drawColor,line width= 0.4pt,line join=round,line cap=round] (163.62, 70.23) circle (  2.25);

\path[draw=drawColor,line width= 0.4pt,line join=round,line cap=round] (168.05, 69.36) circle (  2.25);

\path[draw=drawColor,line width= 0.4pt,line join=round,line cap=round] (172.71, 68.82) circle (  2.25);

\path[draw=drawColor,line width= 0.4pt,line join=round,line cap=round] (177.38, 68.59) circle (  2.25);

\path[draw=drawColor,line width= 0.4pt,line join=round,line cap=round] (182.04, 68.60) circle (  2.25);

\path[draw=drawColor,line width= 0.4pt,line join=round,line cap=round] (187.03, 68.80) circle (  2.25);

\path[draw=drawColor,line width= 0.4pt,line join=round,line cap=round] (192.57, 69.20) circle (  2.25);

\path[draw=drawColor,line width= 0.4pt,line join=round,line cap=round] (198.66, 69.75) circle (  2.25);

\path[draw=drawColor,line width= 0.4pt,line join=round,line cap=round] (204.76, 70.37) circle (  2.25);

\path[draw=drawColor,line width= 0.4pt,line join=round,line cap=round] (210.85, 71.00) circle (  2.25);

\path[draw=drawColor,line width= 0.4pt,line join=round,line cap=round] (216.94, 71.61) circle (  2.25);

\path[draw=drawColor,line width= 0.4pt,line join=round,line cap=round] (223.04, 72.18) circle (  2.25);

\path[draw=drawColor,line width= 0.4pt,line join=round,line cap=round] (229.13, 72.69) circle (  2.25);

\path[draw=drawColor,line width= 0.4pt,line join=round,line cap=round] (235.23, 73.13) circle (  2.25);

\path[draw=drawColor,line width= 0.4pt,line join=round,line cap=round] (241.32, 73.51) circle (  2.25);

\path[draw=drawColor,line width= 0.4pt,line join=round,line cap=round] (247.41, 73.82) circle (  2.25);

\path[draw=drawColor,line width= 0.4pt,line join=round,line cap=round] (253.51, 74.07) circle (  2.25);

\path[draw=drawColor,line width= 0.4pt,line join=round,line cap=round] (259.89, 74.28) circle (  2.25);

\path[draw=drawColor,line width= 0.4pt,line join=round,line cap=round] (266.67, 74.44) circle (  2.25);

\path[draw=drawColor,line width= 0.4pt,line join=round,line cap=round] (273.77, 74.57) circle (  2.25);

\path[draw=drawColor,line width= 0.4pt,line join=round,line cap=round] (280.87, 74.65) circle (  2.25);

\path[draw=drawColor,line width= 0.4pt,line join=round,line cap=round] (288.35, 74.70) circle (  2.25);

\path[draw=drawColor,line width= 0.4pt,line join=round,line cap=round] (296.52, 74.72) circle (  2.25);

\path[draw=drawColor,line width= 0.4pt,line join=round,line cap=round] (305.61, 74.73) circle (  2.25);

\path[draw=drawColor,line width= 0.4pt,line join=round,line cap=round] (315.56, 74.72) circle (  2.25);

\path[draw=drawColor,line width= 0.4pt,line join=round,line cap=round] (325.94, 74.70) circle (  2.25);

\path[draw=drawColor,line width= 0.4pt,line join=round,line cap=round] (336.72, 74.67) circle (  2.25);

\path[draw=drawColor,line width= 0.4pt,line join=round,line cap=round] (348.51, 74.65) circle (  2.25);

\path[draw=drawColor,line width= 0.4pt,line join=round,line cap=round] (363.00, 74.62) circle (  2.25);

\path[draw=drawColor,line width= 0.4pt,line join=round,line cap=round] (377.99, 74.60) circle (  2.25);

\path[draw=drawColor,line width= 0.4pt,line join=round,line cap=round] (392.98, 74.59) circle (  2.25);

\path[draw=drawColor,line width= 0.4pt,line join=round,line cap=round] (407.96, 74.59) circle (  2.25);

\path[draw=drawColor,line width= 0.4pt,line join=round,line cap=round] (422.95, 74.59) circle (  2.25);

\path[draw=drawColor,line width= 0.4pt,dash pattern=on 1pt off 3pt ,line join=round,line cap=round] ( 63.58,288.94) --
	( 63.60,288.94) --
	( 63.63,288.94) --
	( 63.66,288.94) --
	( 63.70,288.93) --
	( 63.73,288.93) --
	( 63.76,288.93) --
	( 63.79,288.93) --
	( 63.84,288.93) --
	( 63.91,288.92) --
	( 64.04,288.90) --
	( 64.27,288.86) --
	( 64.77,288.70) --
	( 65.18,288.52) --
	( 65.60,288.27) --
	( 66.02,287.98) --
	( 66.44,287.64) --
	( 66.86,287.25) --
	( 67.28,286.81) --
	( 67.70,286.32) --
	( 68.22,285.66) --
	( 69.00,284.53) --
	( 70.92,281.20) --
	( 72.46,277.98) --
	( 74.01,274.35) --
	( 75.55,270.35) --
	( 77.22,265.69) --
	( 78.88,260.71) --
	( 80.55,255.49) --
	( 82.21,250.05) --
	( 84.02,243.97) --
	( 85.83,237.75) --
	( 87.63,231.43) --
	( 89.44,225.06) --
	( 91.25,218.67) --
	( 93.05,212.29) --
	( 95.00,205.47) --
	( 96.95,198.73) --
	( 98.89,192.10) --
	(100.84,185.61) --
	(103.01,178.54) --
	(106.14,168.77) --
	(109.94,157.56) --
	(113.19,148.60) --
	(116.45,140.25) --
	(119.81,132.26) --
	(123.68,123.90) --
	(127.77,115.95) --
	(131.31,109.83) --
	(134.84,104.36) --
	(138.37, 99.50) --
	(142.35, 94.69) --
	(146.61, 90.29) --
	(150.86, 86.57) --
	(155.11, 83.45) --
	(159.37, 80.87) --
	(163.62, 78.76) --
	(168.05, 77.00) --
	(172.71, 75.55) --
	(177.38, 74.46) --
	(182.04, 73.65) --
	(187.03, 73.06) --
	(192.57, 72.65) --
	(198.66, 72.43) --
	(204.76, 72.38) --
	(210.85, 72.46) --
	(216.94, 72.61) --
	(223.04, 72.81) --
	(229.13, 73.03) --
	(235.23, 73.25) --
	(241.32, 73.46) --
	(247.41, 73.66) --
	(253.51, 73.84) --
	(259.89, 74.00) --
	(266.67, 74.14) --
	(273.77, 74.27) --
	(280.87, 74.37) --
	(288.35, 74.44) --
	(296.52, 74.51) --
	(305.61, 74.56) --
	(315.56, 74.59) --
	(325.94, 74.60) --
	(336.72, 74.61) --
	(348.51, 74.61) --
	(363.00, 74.61) --
	(377.99, 74.60) --
	(392.98, 74.60) --
	(407.96, 74.59) --
	(422.95, 74.59);
\definecolor{fillColor}{RGB}{255,255,255}

\path[draw=drawColor,line width= 0.4pt,line join=round,line cap=round,fill=fillColor] (318.74,297.70) rectangle (437.33,249.70);

\path[draw=drawColor,line width= 0.4pt,line join=round,line cap=round] (321.44,285.70) -- (339.44,285.70);

\path[draw=drawColor,line width= 0.4pt,dash pattern=on 4pt off 4pt ,line join=round,line cap=round] (321.44,261.70) -- (339.44,261.70);

\path[draw=drawColor,line width= 0.4pt,line join=round,line cap=round] (330.44,273.70) circle (  2.25);

\node[text=drawColor,anchor=base west,inner sep=0pt, outer sep=0pt, scale=  1.00] at (348.44,282.25) {exact $y(t;k+\Delta_k)$};

\node[text=drawColor,anchor=base west,inner sep=0pt, outer sep=0pt, scale=  1.00] at (348.44,270.25) {$y(t;k) + S_y(t;k) \Delta_k$};

\node[text=drawColor,anchor=base west,inner sep=0pt, outer sep=0pt, scale=  1.00] at (348.44,258.25) {exact $y(t;k)$};
\end{scope}
\end{tikzpicture}

  \caption{Linearization Error as described in
    Section~\ref{sec:results}. The error is very small.  \label{fig:dampedlinerr}}
\end{figure}
Since we have the analytical solution for the sensitivity, we can
depict it directly, the graphs are shown in
Figure~\ref{fig:dampedGSLvsASolSensitivity}.
\begin{figure}\sffamily\firalining
  \centering
  % Created by tikzDevice version 0.12.3.1 on 2021-06-16 15:27:59
% !TEX encoding = UTF-8 Unicode
\begin{tikzpicture}[x=1pt,y=1pt]
\definecolor{fillColor}{RGB}{255,255,255}
\path[use as bounding box,fill=fillColor,fill opacity=0.00] (0,0) rectangle (462.53,346.90);
\begin{scope}
\path[clip] ( 49.20, 61.20) rectangle (437.33,297.70);
\definecolor{drawColor}{RGB}{0,0,0}

\path[draw=drawColor,line width= 0.4pt,line join=round,line cap=round] ( 63.58,179.45) circle (  2.25);

\path[draw=drawColor,line width= 0.4pt,line join=round,line cap=round] ( 63.60,179.45) circle (  2.25);

\path[draw=drawColor,line width= 0.4pt,line join=round,line cap=round] ( 63.63,179.45) circle (  2.25);

\path[draw=drawColor,line width= 0.4pt,line join=round,line cap=round] ( 63.66,179.45) circle (  2.25);

\path[draw=drawColor,line width= 0.4pt,line join=round,line cap=round] ( 63.70,179.45) circle (  2.25);

\path[draw=drawColor,line width= 0.4pt,line join=round,line cap=round] ( 63.73,179.44) circle (  2.25);

\path[draw=drawColor,line width= 0.4pt,line join=round,line cap=round] ( 63.76,179.44) circle (  2.25);

\path[draw=drawColor,line width= 0.4pt,line join=round,line cap=round] ( 63.79,179.44) circle (  2.25);

\path[draw=drawColor,line width= 0.4pt,line join=round,line cap=round] ( 63.84,179.44) circle (  2.25);

\path[draw=drawColor,line width= 0.4pt,line join=round,line cap=round] ( 63.91,179.43) circle (  2.25);

\path[draw=drawColor,line width= 0.4pt,line join=round,line cap=round] ( 64.04,179.41) circle (  2.25);

\path[draw=drawColor,line width= 0.4pt,line join=round,line cap=round] ( 64.27,179.37) circle (  2.25);

\path[draw=drawColor,line width= 0.4pt,line join=round,line cap=round] ( 64.77,179.21) circle (  2.25);

\path[draw=drawColor,line width= 0.4pt,line join=round,line cap=round] ( 65.18,179.02) circle (  2.25);

\path[draw=drawColor,line width= 0.4pt,line join=round,line cap=round] ( 65.60,178.78) circle (  2.25);

\path[draw=drawColor,line width= 0.4pt,line join=round,line cap=round] ( 66.02,178.48) circle (  2.25);

\path[draw=drawColor,line width= 0.4pt,line join=round,line cap=round] ( 66.44,178.14) circle (  2.25);

\path[draw=drawColor,line width= 0.4pt,line join=round,line cap=round] ( 66.86,177.74) circle (  2.25);

\path[draw=drawColor,line width= 0.4pt,line join=round,line cap=round] ( 67.28,177.30) circle (  2.25);

\path[draw=drawColor,line width= 0.4pt,line join=round,line cap=round] ( 67.70,176.81) circle (  2.25);

\path[draw=drawColor,line width= 0.4pt,line join=round,line cap=round] ( 68.22,176.14) circle (  2.25);

\path[draw=drawColor,line width= 0.4pt,line join=round,line cap=round] ( 69.00,175.01) circle (  2.25);

\path[draw=drawColor,line width= 0.4pt,line join=round,line cap=round] ( 70.92,171.68) circle (  2.25);

\path[draw=drawColor,line width= 0.4pt,line join=round,line cap=round] ( 72.46,168.49) circle (  2.25);

\path[draw=drawColor,line width= 0.4pt,line join=round,line cap=round] ( 74.01,164.91) circle (  2.25);

\path[draw=drawColor,line width= 0.4pt,line join=round,line cap=round] ( 75.55,161.00) circle (  2.25);

\path[draw=drawColor,line width= 0.4pt,line join=round,line cap=round] ( 77.22,156.49) circle (  2.25);

\path[draw=drawColor,line width= 0.4pt,line join=round,line cap=round] ( 78.88,151.74) circle (  2.25);

\path[draw=drawColor,line width= 0.4pt,line join=round,line cap=round] ( 80.55,146.82) circle (  2.25);

\path[draw=drawColor,line width= 0.4pt,line join=round,line cap=round] ( 82.21,141.78) circle (  2.25);

\path[draw=drawColor,line width= 0.4pt,line join=round,line cap=round] ( 84.02,136.25) circle (  2.25);

\path[draw=drawColor,line width= 0.4pt,line join=round,line cap=round] ( 85.83,130.72) circle (  2.25);

\path[draw=drawColor,line width= 0.4pt,line join=round,line cap=round] ( 87.63,125.25) circle (  2.25);

\path[draw=drawColor,line width= 0.4pt,line join=round,line cap=round] ( 89.44,119.87) circle (  2.25);

\path[draw=drawColor,line width= 0.4pt,line join=round,line cap=round] ( 91.25,114.65) circle (  2.25);

\path[draw=drawColor,line width= 0.4pt,line join=round,line cap=round] ( 93.05,109.62) circle (  2.25);

\path[draw=drawColor,line width= 0.4pt,line join=round,line cap=round] ( 95.00,104.45) circle (  2.25);

\path[draw=drawColor,line width= 0.4pt,line join=round,line cap=round] ( 96.95, 99.58) circle (  2.25);

\path[draw=drawColor,line width= 0.4pt,line join=round,line cap=round] ( 98.89, 95.03) circle (  2.25);

\path[draw=drawColor,line width= 0.4pt,line join=round,line cap=round] (100.84, 90.83) circle (  2.25);

\path[draw=drawColor,line width= 0.4pt,line join=round,line cap=round] (103.01, 86.58) circle (  2.25);

\path[draw=drawColor,line width= 0.4pt,line join=round,line cap=round] (106.14, 81.29) circle (  2.25);

\path[draw=drawColor,line width= 0.4pt,line join=round,line cap=round] (109.94, 76.21) circle (  2.25);

\path[draw=drawColor,line width= 0.4pt,line join=round,line cap=round] (113.19, 73.08) circle (  2.25);

\path[draw=drawColor,line width= 0.4pt,line join=round,line cap=round] (116.45, 71.02) circle (  2.25);

\path[draw=drawColor,line width= 0.4pt,line join=round,line cap=round] (119.81, 69.96) circle (  2.25);

\path[draw=drawColor,line width= 0.4pt,line join=round,line cap=round] (123.68, 69.99) circle (  2.25);

\path[draw=drawColor,line width= 0.4pt,line join=round,line cap=round] (127.77, 71.36) circle (  2.25);

\path[draw=drawColor,line width= 0.4pt,line join=round,line cap=round] (131.31, 73.53) circle (  2.25);

\path[draw=drawColor,line width= 0.4pt,line join=round,line cap=round] (134.84, 76.49) circle (  2.25);

\path[draw=drawColor,line width= 0.4pt,line join=round,line cap=round] (138.37, 80.11) circle (  2.25);

\path[draw=drawColor,line width= 0.4pt,line join=round,line cap=round] (142.35, 84.84) circle (  2.25);

\path[draw=drawColor,line width= 0.4pt,line join=round,line cap=round] (146.61, 90.48) circle (  2.25);

\path[draw=drawColor,line width= 0.4pt,line join=round,line cap=round] (150.86, 96.58) circle (  2.25);

\path[draw=drawColor,line width= 0.4pt,line join=round,line cap=round] (155.11,102.95) circle (  2.25);

\path[draw=drawColor,line width= 0.4pt,line join=round,line cap=round] (159.37,109.45) circle (  2.25);

\path[draw=drawColor,line width= 0.4pt,line join=round,line cap=round] (163.62,115.95) circle (  2.25);

\path[draw=drawColor,line width= 0.4pt,line join=round,line cap=round] (168.05,122.60) circle (  2.25);

\path[draw=drawColor,line width= 0.4pt,line join=round,line cap=round] (172.71,129.37) circle (  2.25);

\path[draw=drawColor,line width= 0.4pt,line join=round,line cap=round] (177.38,135.81) circle (  2.25);

\path[draw=drawColor,line width= 0.4pt,line join=round,line cap=round] (182.04,141.85) circle (  2.25);

\path[draw=drawColor,line width= 0.4pt,line join=round,line cap=round] (187.03,147.81) circle (  2.25);

\path[draw=drawColor,line width= 0.4pt,line join=round,line cap=round] (192.57,153.78) circle (  2.25);

\path[draw=drawColor,line width= 0.4pt,line join=round,line cap=round] (198.66,159.55) circle (  2.25);

\path[draw=drawColor,line width= 0.4pt,line join=round,line cap=round] (204.76,164.48) circle (  2.25);

\path[draw=drawColor,line width= 0.4pt,line join=round,line cap=round] (210.85,168.63) circle (  2.25);

\path[draw=drawColor,line width= 0.4pt,line join=round,line cap=round] (216.94,172.03) circle (  2.25);

\path[draw=drawColor,line width= 0.4pt,line join=round,line cap=round] (223.04,174.78) circle (  2.25);

\path[draw=drawColor,line width= 0.4pt,line join=round,line cap=round] (229.13,176.93) circle (  2.25);

\path[draw=drawColor,line width= 0.4pt,line join=round,line cap=round] (235.23,178.57) circle (  2.25);

\path[draw=drawColor,line width= 0.4pt,line join=round,line cap=round] (241.32,179.78) circle (  2.25);

\path[draw=drawColor,line width= 0.4pt,line join=round,line cap=round] (247.41,180.64) circle (  2.25);

\path[draw=drawColor,line width= 0.4pt,line join=round,line cap=round] (253.51,181.20) circle (  2.25);

\path[draw=drawColor,line width= 0.4pt,line join=round,line cap=round] (259.89,181.54) circle (  2.25);

\path[draw=drawColor,line width= 0.4pt,line join=round,line cap=round] (266.67,181.69) circle (  2.25);

\path[draw=drawColor,line width= 0.4pt,line join=round,line cap=round] (273.77,181.68) circle (  2.25);

\path[draw=drawColor,line width= 0.4pt,line join=round,line cap=round] (280.87,181.55) circle (  2.25);

\path[draw=drawColor,line width= 0.4pt,line join=round,line cap=round] (288.35,181.34) circle (  2.25);

\path[draw=drawColor,line width= 0.4pt,line join=round,line cap=round] (296.52,181.07) circle (  2.25);

\path[draw=drawColor,line width= 0.4pt,line join=round,line cap=round] (305.61,180.75) circle (  2.25);

\path[draw=drawColor,line width= 0.4pt,line join=round,line cap=round] (315.56,180.43) circle (  2.25);

\path[draw=drawColor,line width= 0.4pt,line join=round,line cap=round] (325.94,180.13) circle (  2.25);

\path[draw=drawColor,line width= 0.4pt,line join=round,line cap=round] (336.72,179.89) circle (  2.25);

\path[draw=drawColor,line width= 0.4pt,line join=round,line cap=round] (348.51,179.69) circle (  2.25);

\path[draw=drawColor,line width= 0.4pt,line join=round,line cap=round] (363.00,179.54) circle (  2.25);

\path[draw=drawColor,line width= 0.4pt,line join=round,line cap=round] (377.99,179.45) circle (  2.25);

\path[draw=drawColor,line width= 0.4pt,line join=round,line cap=round] (392.98,179.42) circle (  2.25);

\path[draw=drawColor,line width= 0.4pt,line join=round,line cap=round] (407.96,179.41) circle (  2.25);

\path[draw=drawColor,line width= 0.4pt,line join=round,line cap=round] (422.95,179.41) circle (  2.25);
\end{scope}
\begin{scope}
\path[clip] (  0.00,  0.00) rectangle (462.53,346.90);
\definecolor{drawColor}{RGB}{0,0,0}

\path[draw=drawColor,line width= 0.4pt,line join=round,line cap=round] ( 63.58, 61.20) -- (369.49, 61.20);

\path[draw=drawColor,line width= 0.4pt,line join=round,line cap=round] ( 63.58, 61.20) -- ( 63.58, 55.20);

\path[draw=drawColor,line width= 0.4pt,line join=round,line cap=round] (140.06, 61.20) -- (140.06, 55.20);

\path[draw=drawColor,line width= 0.4pt,line join=round,line cap=round] (216.54, 61.20) -- (216.54, 55.20);

\path[draw=drawColor,line width= 0.4pt,line join=round,line cap=round] (293.02, 61.20) -- (293.02, 55.20);

\path[draw=drawColor,line width= 0.4pt,line join=round,line cap=round] (369.49, 61.20) -- (369.49, 55.20);

\node[text=drawColor,anchor=base,inner sep=0pt, outer sep=0pt, scale=  1.00] at ( 63.58, 39.60) {0};

\node[text=drawColor,anchor=base,inner sep=0pt, outer sep=0pt, scale=  1.00] at (140.06, 39.60) {5};

\node[text=drawColor,anchor=base,inner sep=0pt, outer sep=0pt, scale=  1.00] at (216.54, 39.60) {10};

\node[text=drawColor,anchor=base,inner sep=0pt, outer sep=0pt, scale=  1.00] at (293.02, 39.60) {15};

\node[text=drawColor,anchor=base,inner sep=0pt, outer sep=0pt, scale=  1.00] at (369.49, 39.60) {20};

\path[draw=drawColor,line width= 0.4pt,line join=round,line cap=round] ( 49.20, 99.71) -- ( 49.20,259.19);

\path[draw=drawColor,line width= 0.4pt,line join=round,line cap=round] ( 49.20, 99.71) -- ( 43.20, 99.71);

\path[draw=drawColor,line width= 0.4pt,line join=round,line cap=round] ( 49.20,139.58) -- ( 43.20,139.58);

\path[draw=drawColor,line width= 0.4pt,line join=round,line cap=round] ( 49.20,179.45) -- ( 43.20,179.45);

\path[draw=drawColor,line width= 0.4pt,line join=round,line cap=round] ( 49.20,219.32) -- ( 43.20,219.32);

\path[draw=drawColor,line width= 0.4pt,line join=round,line cap=round] ( 49.20,259.19) -- ( 43.20,259.19);

\node[text=drawColor,rotate= 90.00,anchor=base,inner sep=0pt, outer sep=0pt, scale=  1.00] at ( 34.80, 99.71) {-1.0};

\node[text=drawColor,rotate= 90.00,anchor=base,inner sep=0pt, outer sep=0pt, scale=  1.00] at ( 34.80,139.58) {-0.5};

\node[text=drawColor,rotate= 90.00,anchor=base,inner sep=0pt, outer sep=0pt, scale=  1.00] at ( 34.80,179.45) {0.0};

\node[text=drawColor,rotate= 90.00,anchor=base,inner sep=0pt, outer sep=0pt, scale=  1.00] at ( 34.80,219.32) {0.5};

\node[text=drawColor,rotate= 90.00,anchor=base,inner sep=0pt, outer sep=0pt, scale=  1.00] at ( 34.80,259.19) {1.0};

\path[draw=drawColor,line width= 0.4pt,line join=round,line cap=round] ( 49.20, 61.20) --
	(437.33, 61.20) --
	(437.33,297.70) --
	( 49.20,297.70) --
	( 49.20, 61.20);
\end{scope}
\begin{scope}
\path[clip] (  0.00,  0.00) rectangle (462.53,346.90);
\definecolor{drawColor}{RGB}{0,0,0}

\node[text=drawColor,anchor=base,inner sep=0pt, outer sep=0pt, scale=  1.20] at (243.26,318.16) {\bfseries sensitivity approximation};

\node[text=drawColor,anchor=base,inner sep=0pt, outer sep=0pt, scale=  1.00] at (243.26, 15.60) {time};

\node[text=drawColor,rotate= 90.00,anchor=base,inner sep=0pt, outer sep=0pt, scale=  1.00] at ( 10.80,179.45) {sensitivity};
\end{scope}
\begin{scope}
\path[clip] ( 49.20, 61.20) rectangle (437.33,297.70);
\definecolor{drawColor}{RGB}{0,0,0}

\path[draw=drawColor,line width= 0.4pt,line join=round,line cap=round] ( 63.58,179.45) --
	( 63.60,179.47) --
	( 63.63,179.50) --
	( 63.66,179.52) --
	( 63.70,179.55) --
	( 63.73,179.57) --
	( 63.76,179.60) --
	( 63.79,179.62) --
	( 63.84,179.66) --
	( 63.91,179.71) --
	( 64.04,179.80) --
	( 64.27,179.95) --
	( 64.77,180.19) --
	( 65.18,180.33) --
	( 65.60,180.40) --
	( 66.02,180.41) --
	( 66.44,180.37) --
	( 66.86,180.26) --
	( 67.28,180.10) --
	( 67.70,179.88) --
	( 68.22,179.55) --
	( 69.00,178.88) --
	( 70.92,176.59) --
	( 72.46,174.13) --
	( 74.01,171.18) --
	( 75.55,167.83) --
	( 77.22,163.83) --
	( 78.88,159.51) --
	( 80.55,154.94) --
	( 82.21,150.18) --
	( 84.02,144.89) --
	( 85.83,139.53) --
	( 87.63,134.16) --
	( 89.44,128.84) --
	( 91.25,123.63) --
	( 93.05,118.56) --
	( 95.00,113.30) --
	( 96.95,108.30) --
	( 98.89,103.60) --
	(100.84, 99.22) --
	(103.01, 94.73) --
	(106.14, 89.08) --
	(109.94, 83.53) --
	(113.19, 79.95) --
	(116.45, 77.42) --
	(119.81, 75.88) --
	(123.68, 75.37) --
	(127.77, 76.18) --
	(131.31, 77.87) --
	(134.84, 80.37) --
	(138.37, 83.56) --
	(142.35, 87.85) --
	(146.61, 93.06) --
	(150.86, 98.76) --
	(155.11,104.77) --
	(159.37,110.96) --
	(163.62,117.18) --
	(168.05,123.58) --
	(172.71,130.12) --
	(177.38,136.36) --
	(182.04,142.23) --
	(187.03,148.05) --
	(192.57,153.89) --
	(198.66,159.55) --
	(204.76,164.41) --
	(210.85,168.50) --
	(216.94,171.87) --
	(223.04,174.59) --
	(229.13,176.74) --
	(235.23,178.39) --
	(241.32,179.61) --
	(247.41,180.47) --
	(253.51,181.05) --
	(259.89,181.41) --
	(266.67,181.58) --
	(273.77,181.59) --
	(280.87,181.48) --
	(288.35,181.29) --
	(296.52,181.03) --
	(305.61,180.73) --
	(315.56,180.41) --
	(325.94,180.13) --
	(336.72,179.89) --
	(348.51,179.70) --
	(363.00,179.55) --
	(377.99,179.47) --
	(392.98,179.43) --
	(407.96,179.41) --
	(422.95,179.42);

\path[draw=drawColor,line width= 0.4pt,dash pattern=on 4pt off 4pt ,line join=round,line cap=round] ( 63.58,179.45) --
	( 63.60,179.45) --
	( 63.63,179.45) --
	( 63.66,179.45) --
	( 63.70,179.45) --
	( 63.73,179.44) --
	( 63.76,179.44) --
	( 63.79,179.44) --
	( 63.84,179.44) --
	( 63.91,179.43) --
	( 64.04,179.41) --
	( 64.27,179.37) --
	( 64.77,179.21) --
	( 65.18,179.02) --
	( 65.60,178.78) --
	( 66.02,178.48) --
	( 66.44,178.14) --
	( 66.86,177.74) --
	( 67.28,177.30) --
	( 67.70,176.81) --
	( 68.22,176.14) --
	( 69.00,175.00) --
	( 70.92,171.68) --
	( 72.46,168.48) --
	( 74.01,164.89) --
	( 75.55,160.99) --
	( 77.22,156.48) --
	( 78.88,151.73) --
	( 80.55,146.80) --
	( 82.21,141.77) --
	( 84.02,136.24) --
	( 85.83,130.71) --
	( 87.63,125.24) --
	( 89.44,119.87) --
	( 91.25,114.65) --
	( 93.05,109.63) --
	( 95.00,104.46) --
	( 96.95, 99.59) --
	( 98.89, 95.04) --
	(100.84, 90.85) --
	(103.01, 86.61) --
	(106.14, 81.34) --
	(109.94, 76.31) --
	(113.19, 73.19) --
	(116.45, 71.15) --
	(119.81, 70.10) --
	(123.68, 70.16) --
	(127.77, 71.55) --
	(131.31, 73.72) --
	(134.84, 76.67) --
	(138.37, 80.28) --
	(142.35, 85.00) --
	(146.61, 90.64) --
	(150.86, 96.73) --
	(155.11,103.08) --
	(159.37,109.56) --
	(163.62,116.05) --
	(168.05,122.68) --
	(172.71,129.43) --
	(177.38,135.85) --
	(182.04,141.86) --
	(187.03,147.81) --
	(192.57,153.76) --
	(198.66,159.50) --
	(204.76,164.42) --
	(210.85,168.55) --
	(216.94,171.95) --
	(223.04,174.68) --
	(229.13,176.83) --
	(235.23,178.48) --
	(241.32,179.69) --
	(247.41,180.55) --
	(253.51,181.12) --
	(259.89,181.47) --
	(266.67,181.63) --
	(273.77,181.63) --
	(280.87,181.51) --
	(288.35,181.31) --
	(296.52,181.05) --
	(305.61,180.74) --
	(315.56,180.42) --
	(325.94,180.13) --
	(336.72,179.90) --
	(348.51,179.70) --
	(363.00,179.55) --
	(377.99,179.46) --
	(392.98,179.43) --
	(407.96,179.41) --
	(422.95,179.42);

\path[draw=drawColor,line width= 0.4pt,line join=round,line cap=round] ( 61.33,177.20) -- ( 65.83,181.70);

\path[draw=drawColor,line width= 0.4pt,line join=round,line cap=round] ( 61.33,181.70) -- ( 65.83,177.20);

\path[draw=drawColor,line width= 0.4pt,line join=round,line cap=round] ( 61.35,177.20) -- ( 65.85,181.70);

\path[draw=drawColor,line width= 0.4pt,line join=round,line cap=round] ( 61.35,181.70) -- ( 65.85,177.20);

\path[draw=drawColor,line width= 0.4pt,line join=round,line cap=round] ( 61.38,177.20) -- ( 65.88,181.70);

\path[draw=drawColor,line width= 0.4pt,line join=round,line cap=round] ( 61.38,181.70) -- ( 65.88,177.20);

\path[draw=drawColor,line width= 0.4pt,line join=round,line cap=round] ( 61.41,177.20) -- ( 65.91,181.70);

\path[draw=drawColor,line width= 0.4pt,line join=round,line cap=round] ( 61.41,181.70) -- ( 65.91,177.20);

\path[draw=drawColor,line width= 0.4pt,line join=round,line cap=round] ( 61.45,177.20) -- ( 65.95,181.70);

\path[draw=drawColor,line width= 0.4pt,line join=round,line cap=round] ( 61.45,181.70) -- ( 65.95,177.20);

\path[draw=drawColor,line width= 0.4pt,line join=round,line cap=round] ( 61.48,177.19) -- ( 65.98,181.69);

\path[draw=drawColor,line width= 0.4pt,line join=round,line cap=round] ( 61.48,181.69) -- ( 65.98,177.19);

\path[draw=drawColor,line width= 0.4pt,line join=round,line cap=round] ( 61.51,177.19) -- ( 66.01,181.69);

\path[draw=drawColor,line width= 0.4pt,line join=round,line cap=round] ( 61.51,181.69) -- ( 66.01,177.19);

\path[draw=drawColor,line width= 0.4pt,line join=round,line cap=round] ( 61.54,177.19) -- ( 66.04,181.69);

\path[draw=drawColor,line width= 0.4pt,line join=round,line cap=round] ( 61.54,181.69) -- ( 66.04,177.19);

\path[draw=drawColor,line width= 0.4pt,line join=round,line cap=round] ( 61.59,177.19) -- ( 66.09,181.69);

\path[draw=drawColor,line width= 0.4pt,line join=round,line cap=round] ( 61.59,181.69) -- ( 66.09,177.19);

\path[draw=drawColor,line width= 0.4pt,line join=round,line cap=round] ( 61.66,177.18) -- ( 66.16,181.68);

\path[draw=drawColor,line width= 0.4pt,line join=round,line cap=round] ( 61.66,181.68) -- ( 66.16,177.18);

\path[draw=drawColor,line width= 0.4pt,line join=round,line cap=round] ( 61.79,177.16) -- ( 66.29,181.66);

\path[draw=drawColor,line width= 0.4pt,line join=round,line cap=round] ( 61.79,181.66) -- ( 66.29,177.16);

\path[draw=drawColor,line width= 0.4pt,line join=round,line cap=round] ( 62.02,177.12) -- ( 66.52,181.62);

\path[draw=drawColor,line width= 0.4pt,line join=round,line cap=round] ( 62.02,181.62) -- ( 66.52,177.12);

\path[draw=drawColor,line width= 0.4pt,line join=round,line cap=round] ( 62.52,176.96) -- ( 67.02,181.46);

\path[draw=drawColor,line width= 0.4pt,line join=round,line cap=round] ( 62.52,181.46) -- ( 67.02,176.96);

\path[draw=drawColor,line width= 0.4pt,line join=round,line cap=round] ( 62.93,176.77) -- ( 67.43,181.27);

\path[draw=drawColor,line width= 0.4pt,line join=round,line cap=round] ( 62.93,181.27) -- ( 67.43,176.77);

\path[draw=drawColor,line width= 0.4pt,line join=round,line cap=round] ( 63.35,176.53) -- ( 67.85,181.03);

\path[draw=drawColor,line width= 0.4pt,line join=round,line cap=round] ( 63.35,181.03) -- ( 67.85,176.53);

\path[draw=drawColor,line width= 0.4pt,line join=round,line cap=round] ( 63.77,176.23) -- ( 68.27,180.73);

\path[draw=drawColor,line width= 0.4pt,line join=round,line cap=round] ( 63.77,180.73) -- ( 68.27,176.23);

\path[draw=drawColor,line width= 0.4pt,line join=round,line cap=round] ( 64.19,175.89) -- ( 68.69,180.39);

\path[draw=drawColor,line width= 0.4pt,line join=round,line cap=round] ( 64.19,180.39) -- ( 68.69,175.89);

\path[draw=drawColor,line width= 0.4pt,line join=round,line cap=round] ( 64.61,175.49) -- ( 69.11,179.99);

\path[draw=drawColor,line width= 0.4pt,line join=round,line cap=round] ( 64.61,179.99) -- ( 69.11,175.49);

\path[draw=drawColor,line width= 0.4pt,line join=round,line cap=round] ( 65.03,175.05) -- ( 69.53,179.55);

\path[draw=drawColor,line width= 0.4pt,line join=round,line cap=round] ( 65.03,179.55) -- ( 69.53,175.05);

\path[draw=drawColor,line width= 0.4pt,line join=round,line cap=round] ( 65.45,174.56) -- ( 69.95,179.06);

\path[draw=drawColor,line width= 0.4pt,line join=round,line cap=round] ( 65.45,179.06) -- ( 69.95,174.56);

\path[draw=drawColor,line width= 0.4pt,line join=round,line cap=round] ( 65.97,173.89) -- ( 70.47,178.39);

\path[draw=drawColor,line width= 0.4pt,line join=round,line cap=round] ( 65.97,178.39) -- ( 70.47,173.89);

\path[draw=drawColor,line width= 0.4pt,line join=round,line cap=round] ( 66.75,172.75) -- ( 71.25,177.25);

\path[draw=drawColor,line width= 0.4pt,line join=round,line cap=round] ( 66.75,177.25) -- ( 71.25,172.75);

\path[draw=drawColor,line width= 0.4pt,line join=round,line cap=round] ( 68.67,169.43) -- ( 73.17,173.93);

\path[draw=drawColor,line width= 0.4pt,line join=round,line cap=round] ( 68.67,173.93) -- ( 73.17,169.43);

\path[draw=drawColor,line width= 0.4pt,line join=round,line cap=round] ( 70.21,166.23) -- ( 74.71,170.73);

\path[draw=drawColor,line width= 0.4pt,line join=round,line cap=round] ( 70.21,170.73) -- ( 74.71,166.23);

\path[draw=drawColor,line width= 0.4pt,line join=round,line cap=round] ( 71.76,162.64) -- ( 76.26,167.14);

\path[draw=drawColor,line width= 0.4pt,line join=round,line cap=round] ( 71.76,167.14) -- ( 76.26,162.64);

\path[draw=drawColor,line width= 0.4pt,line join=round,line cap=round] ( 73.30,158.74) -- ( 77.80,163.24);

\path[draw=drawColor,line width= 0.4pt,line join=round,line cap=round] ( 73.30,163.24) -- ( 77.80,158.74);

\path[draw=drawColor,line width= 0.4pt,line join=round,line cap=round] ( 74.97,154.23) -- ( 79.47,158.73);

\path[draw=drawColor,line width= 0.4pt,line join=round,line cap=round] ( 74.97,158.73) -- ( 79.47,154.23);

\path[draw=drawColor,line width= 0.4pt,line join=round,line cap=round] ( 76.63,149.48) -- ( 81.13,153.98);

\path[draw=drawColor,line width= 0.4pt,line join=round,line cap=round] ( 76.63,153.98) -- ( 81.13,149.48);

\path[draw=drawColor,line width= 0.4pt,line join=round,line cap=round] ( 78.30,144.55) -- ( 82.80,149.05);

\path[draw=drawColor,line width= 0.4pt,line join=round,line cap=round] ( 78.30,149.05) -- ( 82.80,144.55);

\path[draw=drawColor,line width= 0.4pt,line join=round,line cap=round] ( 79.96,139.52) -- ( 84.46,144.02);

\path[draw=drawColor,line width= 0.4pt,line join=round,line cap=round] ( 79.96,144.02) -- ( 84.46,139.52);

\path[draw=drawColor,line width= 0.4pt,line join=round,line cap=round] ( 81.77,133.99) -- ( 86.27,138.49);

\path[draw=drawColor,line width= 0.4pt,line join=round,line cap=round] ( 81.77,138.49) -- ( 86.27,133.99);

\path[draw=drawColor,line width= 0.4pt,line join=round,line cap=round] ( 83.58,128.46) -- ( 88.08,132.96);

\path[draw=drawColor,line width= 0.4pt,line join=round,line cap=round] ( 83.58,132.96) -- ( 88.08,128.46);

\path[draw=drawColor,line width= 0.4pt,line join=round,line cap=round] ( 85.38,122.99) -- ( 89.88,127.49);

\path[draw=drawColor,line width= 0.4pt,line join=round,line cap=round] ( 85.38,127.49) -- ( 89.88,122.99);

\path[draw=drawColor,line width= 0.4pt,line join=round,line cap=round] ( 87.19,117.62) -- ( 91.69,122.12);

\path[draw=drawColor,line width= 0.4pt,line join=round,line cap=round] ( 87.19,122.12) -- ( 91.69,117.62);

\path[draw=drawColor,line width= 0.4pt,line join=round,line cap=round] ( 89.00,112.40) -- ( 93.50,116.90);

\path[draw=drawColor,line width= 0.4pt,line join=round,line cap=round] ( 89.00,116.90) -- ( 93.50,112.40);

\path[draw=drawColor,line width= 0.4pt,line join=round,line cap=round] ( 90.80,107.38) -- ( 95.30,111.88);

\path[draw=drawColor,line width= 0.4pt,line join=round,line cap=round] ( 90.80,111.88) -- ( 95.30,107.38);

\path[draw=drawColor,line width= 0.4pt,line join=round,line cap=round] ( 92.75,102.21) -- ( 97.25,106.71);

\path[draw=drawColor,line width= 0.4pt,line join=round,line cap=round] ( 92.75,106.71) -- ( 97.25,102.21);

\path[draw=drawColor,line width= 0.4pt,line join=round,line cap=round] ( 94.70, 97.34) -- ( 99.20,101.84);

\path[draw=drawColor,line width= 0.4pt,line join=round,line cap=round] ( 94.70,101.84) -- ( 99.20, 97.34);

\path[draw=drawColor,line width= 0.4pt,line join=round,line cap=round] ( 96.64, 92.79) -- (101.14, 97.29);

\path[draw=drawColor,line width= 0.4pt,line join=round,line cap=round] ( 96.64, 97.29) -- (101.14, 92.79);

\path[draw=drawColor,line width= 0.4pt,line join=round,line cap=round] ( 98.59, 88.60) -- (103.09, 93.10);

\path[draw=drawColor,line width= 0.4pt,line join=round,line cap=round] ( 98.59, 93.10) -- (103.09, 88.60);

\path[draw=drawColor,line width= 0.4pt,line join=round,line cap=round] (100.76, 84.36) -- (105.26, 88.86);

\path[draw=drawColor,line width= 0.4pt,line join=round,line cap=round] (100.76, 88.86) -- (105.26, 84.36);

\path[draw=drawColor,line width= 0.4pt,line join=round,line cap=round] (103.89, 79.09) -- (108.39, 83.59);

\path[draw=drawColor,line width= 0.4pt,line join=round,line cap=round] (103.89, 83.59) -- (108.39, 79.09);

\path[draw=drawColor,line width= 0.4pt,line join=round,line cap=round] (107.69, 74.06) -- (112.19, 78.56);

\path[draw=drawColor,line width= 0.4pt,line join=round,line cap=round] (107.69, 78.56) -- (112.19, 74.06);

\path[draw=drawColor,line width= 0.4pt,line join=round,line cap=round] (110.94, 70.94) -- (115.44, 75.44);

\path[draw=drawColor,line width= 0.4pt,line join=round,line cap=round] (110.94, 75.44) -- (115.44, 70.94);

\path[draw=drawColor,line width= 0.4pt,line join=round,line cap=round] (114.20, 68.90) -- (118.70, 73.40);

\path[draw=drawColor,line width= 0.4pt,line join=round,line cap=round] (114.20, 73.40) -- (118.70, 68.90);

\path[draw=drawColor,line width= 0.4pt,line join=round,line cap=round] (117.56, 67.85) -- (122.06, 72.35);

\path[draw=drawColor,line width= 0.4pt,line join=round,line cap=round] (117.56, 72.35) -- (122.06, 67.85);

\path[draw=drawColor,line width= 0.4pt,line join=round,line cap=round] (121.43, 67.91) -- (125.93, 72.41);

\path[draw=drawColor,line width= 0.4pt,line join=round,line cap=round] (121.43, 72.41) -- (125.93, 67.91);

\path[draw=drawColor,line width= 0.4pt,line join=round,line cap=round] (125.52, 69.30) -- (130.02, 73.80);

\path[draw=drawColor,line width= 0.4pt,line join=round,line cap=round] (125.52, 73.80) -- (130.02, 69.30);

\path[draw=drawColor,line width= 0.4pt,line join=round,line cap=round] (129.06, 71.47) -- (133.56, 75.97);

\path[draw=drawColor,line width= 0.4pt,line join=round,line cap=round] (129.06, 75.97) -- (133.56, 71.47);

\path[draw=drawColor,line width= 0.4pt,line join=round,line cap=round] (132.59, 74.42) -- (137.09, 78.92);

\path[draw=drawColor,line width= 0.4pt,line join=round,line cap=round] (132.59, 78.92) -- (137.09, 74.42);

\path[draw=drawColor,line width= 0.4pt,line join=round,line cap=round] (136.12, 78.03) -- (140.62, 82.53);

\path[draw=drawColor,line width= 0.4pt,line join=round,line cap=round] (136.12, 82.53) -- (140.62, 78.03);

\path[draw=drawColor,line width= 0.4pt,line join=round,line cap=round] (140.10, 82.75) -- (144.60, 87.25);

\path[draw=drawColor,line width= 0.4pt,line join=round,line cap=round] (140.10, 87.25) -- (144.60, 82.75);

\path[draw=drawColor,line width= 0.4pt,line join=round,line cap=round] (144.36, 88.39) -- (148.86, 92.89);

\path[draw=drawColor,line width= 0.4pt,line join=round,line cap=round] (144.36, 92.89) -- (148.86, 88.39);

\path[draw=drawColor,line width= 0.4pt,line join=round,line cap=round] (148.61, 94.48) -- (153.11, 98.98);

\path[draw=drawColor,line width= 0.4pt,line join=round,line cap=round] (148.61, 98.98) -- (153.11, 94.48);

\path[draw=drawColor,line width= 0.4pt,line join=round,line cap=round] (152.86,100.83) -- (157.36,105.33);

\path[draw=drawColor,line width= 0.4pt,line join=round,line cap=round] (152.86,105.33) -- (157.36,100.83);

\path[draw=drawColor,line width= 0.4pt,line join=round,line cap=round] (157.12,107.31) -- (161.62,111.81);

\path[draw=drawColor,line width= 0.4pt,line join=round,line cap=round] (157.12,111.81) -- (161.62,107.31);

\path[draw=drawColor,line width= 0.4pt,line join=round,line cap=round] (161.37,113.80) -- (165.87,118.30);

\path[draw=drawColor,line width= 0.4pt,line join=round,line cap=round] (161.37,118.30) -- (165.87,113.80);

\path[draw=drawColor,line width= 0.4pt,line join=round,line cap=round] (165.80,120.43) -- (170.30,124.93);

\path[draw=drawColor,line width= 0.4pt,line join=round,line cap=round] (165.80,124.93) -- (170.30,120.43);

\path[draw=drawColor,line width= 0.4pt,line join=round,line cap=round] (170.46,127.18) -- (174.96,131.68);

\path[draw=drawColor,line width= 0.4pt,line join=round,line cap=round] (170.46,131.68) -- (174.96,127.18);

\path[draw=drawColor,line width= 0.4pt,line join=round,line cap=round] (175.13,133.60) -- (179.63,138.10);

\path[draw=drawColor,line width= 0.4pt,line join=round,line cap=round] (175.13,138.10) -- (179.63,133.60);

\path[draw=drawColor,line width= 0.4pt,line join=round,line cap=round] (179.79,139.61) -- (184.29,144.11);

\path[draw=drawColor,line width= 0.4pt,line join=round,line cap=round] (179.79,144.11) -- (184.29,139.61);

\path[draw=drawColor,line width= 0.4pt,line join=round,line cap=round] (184.78,145.56) -- (189.28,150.06);

\path[draw=drawColor,line width= 0.4pt,line join=round,line cap=round] (184.78,150.06) -- (189.28,145.56);

\path[draw=drawColor,line width= 0.4pt,line join=round,line cap=round] (190.32,151.51) -- (194.82,156.01);

\path[draw=drawColor,line width= 0.4pt,line join=round,line cap=round] (190.32,156.01) -- (194.82,151.51);

\path[draw=drawColor,line width= 0.4pt,line join=round,line cap=round] (196.41,157.25) -- (200.91,161.75);

\path[draw=drawColor,line width= 0.4pt,line join=round,line cap=round] (196.41,161.75) -- (200.91,157.25);

\path[draw=drawColor,line width= 0.4pt,line join=round,line cap=round] (202.51,162.17) -- (207.01,166.67);

\path[draw=drawColor,line width= 0.4pt,line join=round,line cap=round] (202.51,166.67) -- (207.01,162.17);

\path[draw=drawColor,line width= 0.4pt,line join=round,line cap=round] (208.60,166.30) -- (213.10,170.80);

\path[draw=drawColor,line width= 0.4pt,line join=round,line cap=round] (208.60,170.80) -- (213.10,166.30);

\path[draw=drawColor,line width= 0.4pt,line join=round,line cap=round] (214.69,169.70) -- (219.19,174.20);

\path[draw=drawColor,line width= 0.4pt,line join=round,line cap=round] (214.69,174.20) -- (219.19,169.70);

\path[draw=drawColor,line width= 0.4pt,line join=round,line cap=round] (220.79,172.43) -- (225.29,176.93);

\path[draw=drawColor,line width= 0.4pt,line join=round,line cap=round] (220.79,176.93) -- (225.29,172.43);

\path[draw=drawColor,line width= 0.4pt,line join=round,line cap=round] (226.88,174.58) -- (231.38,179.08);

\path[draw=drawColor,line width= 0.4pt,line join=round,line cap=round] (226.88,179.08) -- (231.38,174.58);

\path[draw=drawColor,line width= 0.4pt,line join=round,line cap=round] (232.98,176.23) -- (237.48,180.73);

\path[draw=drawColor,line width= 0.4pt,line join=round,line cap=round] (232.98,180.73) -- (237.48,176.23);

\path[draw=drawColor,line width= 0.4pt,line join=round,line cap=round] (239.07,177.44) -- (243.57,181.94);

\path[draw=drawColor,line width= 0.4pt,line join=round,line cap=round] (239.07,181.94) -- (243.57,177.44);

\path[draw=drawColor,line width= 0.4pt,line join=round,line cap=round] (245.16,178.30) -- (249.66,182.80);

\path[draw=drawColor,line width= 0.4pt,line join=round,line cap=round] (245.16,182.80) -- (249.66,178.30);

\path[draw=drawColor,line width= 0.4pt,line join=round,line cap=round] (251.26,178.87) -- (255.76,183.37);

\path[draw=drawColor,line width= 0.4pt,line join=round,line cap=round] (251.26,183.37) -- (255.76,178.87);

\path[draw=drawColor,line width= 0.4pt,line join=round,line cap=round] (257.64,179.22) -- (262.14,183.72);

\path[draw=drawColor,line width= 0.4pt,line join=round,line cap=round] (257.64,183.72) -- (262.14,179.22);

\path[draw=drawColor,line width= 0.4pt,line join=round,line cap=round] (264.42,179.38) -- (268.92,183.88);

\path[draw=drawColor,line width= 0.4pt,line join=round,line cap=round] (264.42,183.88) -- (268.92,179.38);

\path[draw=drawColor,line width= 0.4pt,line join=round,line cap=round] (271.52,179.38) -- (276.02,183.88);

\path[draw=drawColor,line width= 0.4pt,line join=round,line cap=round] (271.52,183.88) -- (276.02,179.38);

\path[draw=drawColor,line width= 0.4pt,line join=round,line cap=round] (278.62,179.26) -- (283.12,183.76);

\path[draw=drawColor,line width= 0.4pt,line join=round,line cap=round] (278.62,183.76) -- (283.12,179.26);

\path[draw=drawColor,line width= 0.4pt,line join=round,line cap=round] (286.10,179.06) -- (290.60,183.56);

\path[draw=drawColor,line width= 0.4pt,line join=round,line cap=round] (286.10,183.56) -- (290.60,179.06);

\path[draw=drawColor,line width= 0.4pt,line join=round,line cap=round] (294.27,178.80) -- (298.77,183.30);

\path[draw=drawColor,line width= 0.4pt,line join=round,line cap=round] (294.27,183.30) -- (298.77,178.80);

\path[draw=drawColor,line width= 0.4pt,line join=round,line cap=round] (303.36,178.49) -- (307.86,182.99);

\path[draw=drawColor,line width= 0.4pt,line join=round,line cap=round] (303.36,182.99) -- (307.86,178.49);

\path[draw=drawColor,line width= 0.4pt,line join=round,line cap=round] (313.31,178.17) -- (317.81,182.67);

\path[draw=drawColor,line width= 0.4pt,line join=round,line cap=round] (313.31,182.67) -- (317.81,178.17);

\path[draw=drawColor,line width= 0.4pt,line join=round,line cap=round] (323.69,177.88) -- (328.19,182.38);

\path[draw=drawColor,line width= 0.4pt,line join=round,line cap=round] (323.69,182.38) -- (328.19,177.88);

\path[draw=drawColor,line width= 0.4pt,line join=round,line cap=round] (334.47,177.65) -- (338.97,182.15);

\path[draw=drawColor,line width= 0.4pt,line join=round,line cap=round] (334.47,182.15) -- (338.97,177.65);

\path[draw=drawColor,line width= 0.4pt,line join=round,line cap=round] (346.26,177.45) -- (350.76,181.95);

\path[draw=drawColor,line width= 0.4pt,line join=round,line cap=round] (346.26,181.95) -- (350.76,177.45);

\path[draw=drawColor,line width= 0.4pt,line join=round,line cap=round] (360.75,177.30) -- (365.25,181.80);

\path[draw=drawColor,line width= 0.4pt,line join=round,line cap=round] (360.75,181.80) -- (365.25,177.30);

\path[draw=drawColor,line width= 0.4pt,line join=round,line cap=round] (375.74,177.21) -- (380.24,181.71);

\path[draw=drawColor,line width= 0.4pt,line join=round,line cap=round] (375.74,181.71) -- (380.24,177.21);

\path[draw=drawColor,line width= 0.4pt,line join=round,line cap=round] (390.73,177.18) -- (395.23,181.68);

\path[draw=drawColor,line width= 0.4pt,line join=round,line cap=round] (390.73,181.68) -- (395.23,177.18);

\path[draw=drawColor,line width= 0.4pt,line join=round,line cap=round] (405.71,177.17) -- (410.21,181.67);

\path[draw=drawColor,line width= 0.4pt,line join=round,line cap=round] (405.71,181.67) -- (410.21,177.17);

\path[draw=drawColor,line width= 0.4pt,line join=round,line cap=round] (420.70,177.17) -- (425.20,181.67);

\path[draw=drawColor,line width= 0.4pt,line join=round,line cap=round] (420.70,181.67) -- (425.20,177.17);
\definecolor{fillColor}{RGB}{255,255,255}

\path[draw=drawColor,line width= 0.4pt,line join=round,line cap=round,fill=fillColor] ( 49.20,297.70) rectangle (175.18,237.70);

\path[draw=drawColor,line width= 0.4pt,line join=round,line cap=round] ( 51.90,273.70) -- ( 69.90,273.70);

\path[draw=drawColor,line width= 0.4pt,dash pattern=on 4pt off 4pt ,line join=round,line cap=round] ( 51.90,261.70) -- ( 69.90,261.70);

\path[draw=drawColor,line width= 0.4pt,line join=round,line cap=round] ( 60.90,285.70) circle (  2.25);

\path[draw=drawColor,line width= 0.4pt,line join=round,line cap=round] ( 58.65,247.45) -- ( 63.15,251.95);

\path[draw=drawColor,line width= 0.4pt,line join=round,line cap=round] ( 58.65,251.95) -- ( 63.15,247.45);

\node[text=drawColor,anchor=base west,inner sep=0pt, outer sep=0pt, scale=  1.00] at ( 78.90,282.25) {approximation};

\node[text=drawColor,anchor=base west,inner sep=0pt, outer sep=0pt, scale=  1.00] at ( 78.90,270.25) {exact};

\node[text=drawColor,anchor=base west,inner sep=0pt, outer sep=0pt, scale=  1.00] at ( 78.90,258.25) {finite differences};

\node[text=drawColor,anchor=base west,inner sep=0pt, outer sep=0pt, scale=  1.00] at ( 78.90,246.25) {Cauchy int. formula};
\end{scope}
\end{tikzpicture}

  \caption{Comparison between the numerically obtained sensitivity
    approximation and the analytical sensitivity solution, for the
    case of a damped harmonic
    oscillator.  \label{fig:dampedGSLvsASolSensitivity}}
\end{figure}

\subsection{Driving Force}
\label{sec:driven}

We can also add a constant force to the model:
\begin{equation}
  \label{eq:F}
  \ddot y = -ky - c\dot y + F
\end{equation}

Without following through all analytical calculations, we can restrict
ourselves to numerical methods only. This is what we would do for
large models. Since we are not using the analytical sensitivity
solution, we calculate and depict the linearization
error~\eqref{eq:linerr} only. To obtain a direct solution for the shifted trajectory, we use the solver from the \texttt{deSolve} package (in \textsc{r}).

Figure~\ref{fig:SmallForceRdeSolve} shows the effects of a small force, while Figure~\ref{fig:WithForceRdeSolve} depicts the effects of a larger force.

\begin{figure}\sffamily\firalining
  \centering
  % Created by tikzDevice version 0.12.3.1 on 2021-06-16 15:27:58
% !TEX encoding = UTF-8 Unicode
\begin{tikzpicture}[x=1pt,y=1pt]
\definecolor{fillColor}{RGB}{255,255,255}
\path[use as bounding box,fill=fillColor,fill opacity=0.00] (0,0) rectangle (462.53,578.16);
\begin{scope}
\path[clip] ( 49.20,350.28) rectangle (437.33,528.96);
\definecolor{drawColor}{RGB}{0,0,0}

\path[draw=drawColor,line width= 0.4pt,line join=round,line cap=round] ( 63.58,519.23) --
	( 63.60,518.80) --
	( 63.63,518.38) --
	( 63.66,517.96) --
	( 63.68,517.53) --
	( 63.73,516.86) --
	( 63.80,515.73) --
	( 63.93,513.79) --
	( 64.16,510.29) --
	( 64.66,502.79) --
	( 65.08,496.74) --
	( 65.50,490.86) --
	( 65.93,485.14) --
	( 66.35,479.59) --
	( 66.77,474.19) --
	( 67.19,468.96) --
	( 67.61,463.91) --
	( 68.13,457.79) --
	( 68.93,448.95) --
	( 70.86,429.67) --
	( 72.42,416.19) --
	( 73.98,404.40) --
	( 75.54,394.19) --
	( 77.22,384.83) --
	( 78.91,377.06) --
	( 80.59,370.74) --
	( 82.39,365.46) --
	( 84.19,361.56) --
	( 85.99,358.92) --
	( 87.79,357.40) --
	( 89.59,356.90) --
	( 91.39,357.30) --
	( 93.34,358.64) --
	( 95.28,360.79) --
	( 97.23,363.65) --
	( 99.17,367.12) --
	(101.12,371.11) --
	(103.06,375.53) --
	(105.01,380.29) --
	(107.11,385.75) --
	(109.20,391.45) --
	(111.30,397.33) --
	(113.40,403.32) --
	(115.50,409.37) --
	(117.60,415.42) --
	(119.69,421.44) --
	(121.97,427.87) --
	(124.24,434.17) --
	(126.51,440.30) --
	(128.79,446.25) --
	(131.06,451.98) --
	(133.50,457.88) --
	(135.94,463.50) --
	(138.38,468.82) --
	(140.82,473.84) --
	(143.45,478.91) --
	(146.08,483.63) --
	(148.71,487.99) --
	(151.55,492.32) --
	(154.40,496.25) --
	(157.46,500.08) --
	(160.52,503.48) --
	(163.86,506.77) --
	(167.20,509.63) --
	(170.54,512.09) --
	(173.88,514.21) --
	(177.22,515.99) --
	(180.56,517.49) --
	(183.90,518.72) --
	(187.24,519.72) --
	(190.58,520.52) --
	(193.92,521.14) --
	(197.26,521.60) --
	(200.60,521.94) --
	(203.94,522.16) --
	(207.28,522.29) --
	(210.62,522.34) --
	(213.96,522.33) --
	(217.31,522.27) --
	(220.65,522.17) --
	(224.26,522.02) --
	(227.87,521.85) --
	(231.49,521.67) --
	(235.43,521.45) --
	(239.37,521.23) --
	(243.64,521.00) --
	(247.91,520.77) --
	(252.56,520.54) --
	(257.57,520.31) --
	(262.57,520.11) --
	(267.97,519.91) --
	(273.76,519.73) --
	(279.54,519.59) --
	(285.86,519.46) --
	(292.82,519.35) --
	(300.42,519.27) --
	(308.73,519.20) --
	(318.01,519.17) --
	(328.47,519.15) --
	(338.93,519.16) --
	(349.39,519.17) --
	(359.85,519.19) --
	(370.32,519.20) --
	(380.78,519.21) --
	(392.75,519.22) --
	(406.83,519.23) --
	(422.95,519.23);
\end{scope}
\begin{scope}
\path[clip] (  0.00,  0.00) rectangle (462.53,578.16);
\definecolor{drawColor}{RGB}{0,0,0}

\path[draw=drawColor,line width= 0.4pt,line join=round,line cap=round] ( 63.58,350.28) -- (372.10,350.28);

\path[draw=drawColor,line width= 0.4pt,line join=round,line cap=round] ( 63.58,350.28) -- ( 63.58,344.28);

\path[draw=drawColor,line width= 0.4pt,line join=round,line cap=round] (140.71,350.28) -- (140.71,344.28);

\path[draw=drawColor,line width= 0.4pt,line join=round,line cap=round] (217.84,350.28) -- (217.84,344.28);

\path[draw=drawColor,line width= 0.4pt,line join=round,line cap=round] (294.97,350.28) -- (294.97,344.28);

\path[draw=drawColor,line width= 0.4pt,line join=round,line cap=round] (372.10,350.28) -- (372.10,344.28);

\node[text=drawColor,anchor=base,inner sep=0pt, outer sep=0pt, scale=  1.00] at ( 63.58,328.68) {0};

\node[text=drawColor,anchor=base,inner sep=0pt, outer sep=0pt, scale=  1.00] at (140.71,328.68) {5};

\node[text=drawColor,anchor=base,inner sep=0pt, outer sep=0pt, scale=  1.00] at (217.84,328.68) {10};

\node[text=drawColor,anchor=base,inner sep=0pt, outer sep=0pt, scale=  1.00] at (294.97,328.68) {15};

\node[text=drawColor,anchor=base,inner sep=0pt, outer sep=0pt, scale=  1.00] at (372.10,328.68) {20};

\path[draw=drawColor,line width= 0.4pt,line join=round,line cap=round] ( 49.20,367.55) -- ( 49.20,519.23);

\path[draw=drawColor,line width= 0.4pt,line join=round,line cap=round] ( 49.20,367.55) -- ( 43.20,367.55);

\path[draw=drawColor,line width= 0.4pt,line join=round,line cap=round] ( 49.20,397.88) -- ( 43.20,397.88);

\path[draw=drawColor,line width= 0.4pt,line join=round,line cap=round] ( 49.20,428.22) -- ( 43.20,428.22);

\path[draw=drawColor,line width= 0.4pt,line join=round,line cap=round] ( 49.20,458.56) -- ( 43.20,458.56);

\path[draw=drawColor,line width= 0.4pt,line join=round,line cap=round] ( 49.20,488.89) -- ( 43.20,488.89);

\path[draw=drawColor,line width= 0.4pt,line join=round,line cap=round] ( 49.20,519.23) -- ( 43.20,519.23);

\node[text=drawColor,rotate= 90.00,anchor=base,inner sep=0pt, outer sep=0pt, scale=  1.00] at ( 34.80,367.55) {-0.25};

\node[text=drawColor,rotate= 90.00,anchor=base,inner sep=0pt, outer sep=0pt, scale=  1.00] at ( 34.80,397.88) {-0.20};

\node[text=drawColor,rotate= 90.00,anchor=base,inner sep=0pt, outer sep=0pt, scale=  1.00] at ( 34.80,428.22) {-0.15};

\node[text=drawColor,rotate= 90.00,anchor=base,inner sep=0pt, outer sep=0pt, scale=  1.00] at ( 34.80,458.56) {-0.10};

\node[text=drawColor,rotate= 90.00,anchor=base,inner sep=0pt, outer sep=0pt, scale=  1.00] at ( 34.80,488.89) {-0.05};

\node[text=drawColor,rotate= 90.00,anchor=base,inner sep=0pt, outer sep=0pt, scale=  1.00] at ( 34.80,519.23) {0.00};

\path[draw=drawColor,line width= 0.4pt,line join=round,line cap=round] ( 49.20,350.28) --
	(437.33,350.28) --
	(437.33,528.96) --
	( 49.20,528.96) --
	( 49.20,350.28);
\end{scope}
\begin{scope}
\path[clip] (  0.00,289.08) rectangle (462.53,578.16);
\definecolor{drawColor}{RGB}{0,0,0}

\node[text=drawColor,anchor=base,inner sep=0pt, outer sep=0pt, scale=  1.20] at (243.26,549.42) {\bfseries  with damping and small driving force};

\node[text=drawColor,anchor=base,inner sep=0pt, outer sep=0pt, scale=  1.00] at (243.26,292.68) {deSolve solution with $\Delta_k=0.04$ at $F=0.01$ and $c=1$};

\node[text=drawColor,anchor=base,inner sep=0pt, outer sep=0pt, scale=  1.00] at (243.26,304.68) {$t$};

\node[text=drawColor,rotate= 90.00,anchor=base,inner sep=0pt, outer sep=0pt, scale=  1.00] at ( 10.80,439.62) {$v$};
\end{scope}
\begin{scope}
\path[clip] ( 49.20,350.28) rectangle (437.33,528.96);
\definecolor{drawColor}{RGB}{0,0,0}

\path[draw=drawColor,line width= 0.4pt,dash pattern=on 4pt off 4pt ,line join=round,line cap=round] ( 63.58,519.23) --
	( 63.60,518.80) --
	( 63.63,518.38) --
	( 63.66,517.96) --
	( 63.68,517.53) --
	( 63.73,516.86) --
	( 63.80,515.74) --
	( 63.93,513.79) --
	( 64.16,510.30) --
	( 64.66,502.81) --
	( 65.08,496.76) --
	( 65.50,490.88) --
	( 65.93,485.16) --
	( 66.35,479.61) --
	( 66.77,474.22) --
	( 67.19,468.99) --
	( 67.61,463.92) --
	( 68.13,457.82) --
	( 68.93,449.00) --
	( 70.86,429.69) --
	( 72.42,416.17) --
	( 73.98,404.36) --
	( 75.54,394.13) --
	( 77.22,384.75) --
	( 78.91,376.95) --
	( 80.59,370.60) --
	( 82.39,365.28) --
	( 84.19,361.34) --
	( 85.99,358.66) --
	( 87.79,357.11) --
	( 89.59,356.58) --
	( 91.39,356.94) --
	( 93.34,358.24) --
	( 95.28,360.36) --
	( 97.23,363.18) --
	( 99.17,366.62) --
	(101.12,370.57) --
	(103.06,374.96) --
	(105.01,379.70) --
	(107.11,385.12) --
	(109.20,390.80) --
	(111.30,396.66) --
	(113.40,402.63) --
	(115.50,408.66) --
	(117.60,414.70) --
	(119.69,420.71) --
	(121.97,427.14) --
	(124.24,433.45) --
	(126.51,439.60) --
	(128.79,445.57) --
	(131.06,451.32) --
	(133.50,457.25) --
	(135.94,462.89) --
	(138.38,468.24) --
	(140.82,473.30) --
	(143.45,478.41) --
	(146.08,483.18) --
	(148.71,487.59) --
	(151.55,491.98) --
	(154.40,495.98) --
	(157.46,499.87) --
	(160.52,503.36) --
	(163.86,506.72) --
	(167.20,509.66) --
	(170.54,512.19) --
	(173.88,514.37) --
	(177.22,516.22) --
	(180.56,517.76) --
	(183.90,519.04) --
	(187.24,520.09) --
	(190.58,520.92) --
	(193.92,521.57) --
	(197.26,522.07) --
	(200.60,522.42) --
	(203.94,522.66) --
	(207.28,522.79) --
	(210.62,522.85) --
	(213.96,522.83) --
	(217.31,522.77) --
	(220.65,522.65) --
	(224.26,522.50) --
	(227.87,522.31) --
	(231.49,522.10) --
	(235.43,521.86) --
	(239.37,521.62) --
	(243.64,521.36) --
	(247.91,521.10) --
	(252.56,520.84) --
	(257.57,520.57) --
	(262.57,520.33) --
	(267.97,520.10) --
	(273.76,519.89) --
	(279.54,519.72) --
	(285.86,519.56) --
	(292.82,519.42) --
	(300.42,519.32) --
	(308.73,519.24) --
	(318.01,519.19) --
	(328.47,519.16) --
	(338.93,519.15) --
	(349.39,519.16) --
	(359.85,519.17) --
	(370.32,519.18) --
	(380.78,519.19) --
	(392.75,519.21) --
	(406.83,519.22) --
	(422.95,519.22);
\definecolor{fillColor}{RGB}{255,255,255}

\path[draw=drawColor,line width= 0.4pt,line join=round,line cap=round,fill=fillColor] (305.33,528.96) rectangle (437.33,492.96);

\path[draw=drawColor,line width= 0.4pt,line join=round,line cap=round] (314.33,516.96) -- (332.33,516.96);

\path[draw=drawColor,line width= 0.4pt,dash pattern=on 4pt off 4pt ,line join=round,line cap=round] (314.33,504.96) -- (332.33,504.96);

\node[text=drawColor,anchor=base west,inner sep=0pt, outer sep=0pt, scale=  1.00] at (341.33,513.52) {deSolve (R)};

\node[text=drawColor,anchor=base west,inner sep=0pt, outer sep=0pt, scale=  1.00] at (341.33,501.52) {$v(t;k) + S_v(t;k)\cdot\Delta_k$};
\end{scope}
\begin{scope}
\path[clip] ( 49.20, 61.20) rectangle (437.33,239.88);
\definecolor{drawColor}{RGB}{0,0,0}

\path[draw=drawColor,line width= 0.4pt,line join=round,line cap=round] ( 63.58,233.26) --
	( 63.60,233.26) --
	( 63.63,233.26) --
	( 63.66,233.26) --
	( 63.68,233.26) --
	( 63.73,233.26) --
	( 63.80,233.26) --
	( 63.93,233.25) --
	( 64.16,233.22) --
	( 64.66,233.11) --
	( 65.08,232.96) --
	( 65.50,232.77) --
	( 65.93,232.54) --
	( 66.35,232.27) --
	( 66.77,231.95) --
	( 67.19,231.60) --
	( 67.61,231.19) --
	( 68.13,230.66) --
	( 68.93,229.74) --
	( 70.86,226.97) --
	( 72.42,224.28) --
	( 73.98,221.25) --
	( 75.54,217.92) --
	( 77.22,214.03) --
	( 78.91,209.89) --
	( 80.59,205.54) --
	( 82.39,200.69) --
	( 84.19,195.70) --
	( 85.99,190.61) --
	( 87.79,185.45) --
	( 89.59,180.26) --
	( 91.39,175.07) --
	( 93.34,169.49) --
	( 95.28,163.98) --
	( 97.23,158.55) --
	( 99.17,153.23) --
	(101.12,148.03) --
	(103.06,142.99) --
	(105.01,138.10) --
	(107.11,133.02) --
	(109.20,128.15) --
	(111.30,123.49) --
	(113.40,119.05) --
	(115.50,114.84) --
	(117.60,110.86) --
	(119.69,107.10) --
	(121.97,103.28) --
	(124.24, 99.71) --
	(126.51, 96.40) --
	(128.79, 93.33) --
	(131.06, 90.50) --
	(133.50, 87.71) --
	(135.94, 85.16) --
	(138.38, 82.86) --
	(140.82, 80.78) --
	(143.45, 78.78) --
	(146.08, 77.00) --
	(148.71, 75.44) --
	(151.55, 73.97) --
	(154.40, 72.71) --
	(157.46, 71.56) --
	(160.52, 70.62) --
	(163.86, 69.78) --
	(167.20, 69.13) --
	(170.54, 68.64) --
	(173.88, 68.28) --
	(177.22, 68.03) --
	(180.56, 67.88) --
	(183.90, 67.82) --
	(187.24, 67.82) --
	(190.58, 67.87) --
	(193.92, 67.97) --
	(197.26, 68.10) --
	(200.60, 68.25) --
	(203.94, 68.42) --
	(207.28, 68.60) --
	(210.62, 68.78) --
	(213.96, 68.96) --
	(217.31, 69.14) --
	(220.65, 69.32) --
	(224.26, 69.51) --
	(227.87, 69.68) --
	(231.49, 69.84) --
	(235.43, 70.01) --
	(239.37, 70.15) --
	(243.64, 70.30) --
	(247.91, 70.42) --
	(252.56, 70.54) --
	(257.57, 70.65) --
	(262.57, 70.73) --
	(267.97, 70.81) --
	(273.76, 70.87) --
	(279.54, 70.92) --
	(285.86, 70.95) --
	(292.82, 70.97) --
	(300.42, 70.99) --
	(308.73, 70.99) --
	(318.01, 70.99) --
	(328.47, 70.97) --
	(338.93, 70.96) --
	(349.39, 70.95) --
	(359.85, 70.94) --
	(370.32, 70.93) --
	(380.78, 70.93) --
	(392.75, 70.92) --
	(406.83, 70.92) --
	(422.95, 70.92);
\end{scope}
\begin{scope}
\path[clip] (  0.00,  0.00) rectangle (462.53,578.16);
\definecolor{drawColor}{RGB}{0,0,0}

\path[draw=drawColor,line width= 0.4pt,line join=round,line cap=round] ( 63.58, 61.20) -- (372.10, 61.20);

\path[draw=drawColor,line width= 0.4pt,line join=round,line cap=round] ( 63.58, 61.20) -- ( 63.58, 55.20);

\path[draw=drawColor,line width= 0.4pt,line join=round,line cap=round] (140.71, 61.20) -- (140.71, 55.20);

\path[draw=drawColor,line width= 0.4pt,line join=round,line cap=round] (217.84, 61.20) -- (217.84, 55.20);

\path[draw=drawColor,line width= 0.4pt,line join=round,line cap=round] (294.97, 61.20) -- (294.97, 55.20);

\path[draw=drawColor,line width= 0.4pt,line join=round,line cap=round] (372.10, 61.20) -- (372.10, 55.20);

\node[text=drawColor,anchor=base,inner sep=0pt, outer sep=0pt, scale=  1.00] at ( 63.58, 39.60) {0};

\node[text=drawColor,anchor=base,inner sep=0pt, outer sep=0pt, scale=  1.00] at (140.71, 39.60) {5};

\node[text=drawColor,anchor=base,inner sep=0pt, outer sep=0pt, scale=  1.00] at (217.84, 39.60) {10};

\node[text=drawColor,anchor=base,inner sep=0pt, outer sep=0pt, scale=  1.00] at (294.97, 39.60) {15};

\node[text=drawColor,anchor=base,inner sep=0pt, outer sep=0pt, scale=  1.00] at (372.10, 39.60) {20};

\path[draw=drawColor,line width= 0.4pt,line join=round,line cap=round] ( 49.20, 66.85) -- ( 49.20,233.26);

\path[draw=drawColor,line width= 0.4pt,line join=round,line cap=round] ( 49.20, 66.85) -- ( 43.20, 66.85);

\path[draw=drawColor,line width= 0.4pt,line join=round,line cap=round] ( 49.20,100.13) -- ( 43.20,100.13);

\path[draw=drawColor,line width= 0.4pt,line join=round,line cap=round] ( 49.20,133.41) -- ( 43.20,133.41);

\path[draw=drawColor,line width= 0.4pt,line join=round,line cap=round] ( 49.20,166.70) -- ( 43.20,166.70);

\path[draw=drawColor,line width= 0.4pt,line join=round,line cap=round] ( 49.20,199.98) -- ( 43.20,199.98);

\path[draw=drawColor,line width= 0.4pt,line join=round,line cap=round] ( 49.20,233.26) -- ( 43.20,233.26);

\node[text=drawColor,rotate= 90.00,anchor=base,inner sep=0pt, outer sep=0pt, scale=  1.00] at ( 34.80, 66.85) {0.0};

\node[text=drawColor,rotate= 90.00,anchor=base,inner sep=0pt, outer sep=0pt, scale=  1.00] at ( 34.80,100.13) {0.2};

\node[text=drawColor,rotate= 90.00,anchor=base,inner sep=0pt, outer sep=0pt, scale=  1.00] at ( 34.80,133.41) {0.4};

\node[text=drawColor,rotate= 90.00,anchor=base,inner sep=0pt, outer sep=0pt, scale=  1.00] at ( 34.80,166.70) {0.6};

\node[text=drawColor,rotate= 90.00,anchor=base,inner sep=0pt, outer sep=0pt, scale=  1.00] at ( 34.80,199.98) {0.8};

\node[text=drawColor,rotate= 90.00,anchor=base,inner sep=0pt, outer sep=0pt, scale=  1.00] at ( 34.80,233.26) {1.0};

\path[draw=drawColor,line width= 0.4pt,line join=round,line cap=round] ( 49.20, 61.20) --
	(437.33, 61.20) --
	(437.33,239.88) --
	( 49.20,239.88) --
	( 49.20, 61.20);
\end{scope}
\begin{scope}
\path[clip] (  0.00,  0.00) rectangle (462.53,289.08);
\definecolor{drawColor}{RGB}{0,0,0}

\node[text=drawColor,anchor=base,inner sep=0pt, outer sep=0pt, scale=  1.20] at (243.26,260.34) {\bfseries  with damping and small driving force};

\node[text=drawColor,anchor=base,inner sep=0pt, outer sep=0pt, scale=  1.00] at (243.26,  3.60) {average missmatch: $0.038 \Delta_k$};

\node[text=drawColor,anchor=base,inner sep=0pt, outer sep=0pt, scale=  1.00] at (243.26, 15.60) {$t$};

\node[text=drawColor,rotate= 90.00,anchor=base,inner sep=0pt, outer sep=0pt, scale=  1.00] at ( 10.80,150.54) {$y$};
\end{scope}
\begin{scope}
\path[clip] ( 49.20, 61.20) rectangle (437.33,239.88);
\definecolor{drawColor}{RGB}{0,0,0}

\path[draw=drawColor,line width= 0.4pt,dash pattern=on 4pt off 4pt ,line join=round,line cap=round] ( 63.58,233.26) --
	( 63.60,233.26) --
	( 63.63,233.26) --
	( 63.66,233.26) --
	( 63.68,233.26) --
	( 63.73,233.26) --
	( 63.80,233.25) --
	( 63.93,233.24) --
	( 64.16,233.22) --
	( 64.66,233.10) --
	( 65.08,232.96) --
	( 65.50,232.77) --
	( 65.93,232.53) --
	( 66.35,232.26) --
	( 66.77,231.94) --
	( 67.19,231.58) --
	( 67.61,231.19) --
	( 68.13,230.65) --
	( 68.93,229.71) --
	( 70.86,226.96) --
	( 72.42,224.28) --
	( 73.98,221.26) --
	( 75.54,217.93) --
	( 77.22,214.04) --
	( 78.91,209.90) --
	( 80.59,205.54) --
	( 82.39,200.69) --
	( 84.19,195.70) --
	( 85.99,190.60) --
	( 87.79,185.43) --
	( 89.59,180.23) --
	( 91.39,175.02) --
	( 93.34,169.43) --
	( 95.28,163.89) --
	( 97.23,158.44) --
	( 99.17,153.10) --
	(101.12,147.89) --
	(103.06,142.82) --
	(105.01,137.91) --
	(107.11,132.80) --
	(109.20,127.90) --
	(111.30,123.22) --
	(113.40,118.76) --
	(115.50,114.52) --
	(117.60,110.50) --
	(119.69,106.72) --
	(121.97,102.86) --
	(124.24, 99.27) --
	(126.51, 95.92) --
	(128.79, 92.83) --
	(131.06, 89.96) --
	(133.50, 87.15) --
	(135.94, 84.58) --
	(138.38, 82.25) --
	(140.82, 80.15) --
	(143.45, 78.12) --
	(146.08, 76.32) --
	(148.71, 74.74) --
	(151.55, 73.25) --
	(154.40, 71.98) --
	(157.46, 70.82) --
	(160.52, 69.87) --
	(163.86, 69.02) --
	(167.20, 68.37) --
	(170.54, 67.88) --
	(173.88, 67.53) --
	(177.22, 67.30) --
	(180.56, 67.16) --
	(183.90, 67.12) --
	(187.24, 67.14) --
	(190.58, 67.21) --
	(193.92, 67.34) --
	(197.26, 67.49) --
	(200.60, 67.67) --
	(203.94, 67.87) --
	(207.28, 68.08) --
	(210.62, 68.29) --
	(213.96, 68.51) --
	(217.31, 68.72) --
	(220.65, 68.93) --
	(224.26, 69.14) --
	(227.87, 69.35) --
	(231.49, 69.54) --
	(235.43, 69.73) --
	(239.37, 69.91) --
	(243.64, 70.08) --
	(247.91, 70.23) --
	(252.56, 70.38) --
	(257.57, 70.51) --
	(262.57, 70.62) --
	(267.97, 70.71) --
	(273.76, 70.79) --
	(279.54, 70.85) --
	(285.86, 70.90) --
	(292.82, 70.93) --
	(300.42, 70.95) --
	(308.73, 70.96) --
	(318.01, 70.95) --
	(328.47, 70.94) --
	(338.93, 70.93) --
	(349.39, 70.92) --
	(359.85, 70.91) --
	(370.32, 70.90) --
	(380.78, 70.89) --
	(392.75, 70.88) --
	(406.83, 70.88) --
	(422.95, 70.88);
\definecolor{fillColor}{RGB}{255,255,255}

\path[draw=drawColor,line width= 0.4pt,line join=round,line cap=round,fill=fillColor] (305.22,239.88) rectangle (437.33,203.88);

\path[draw=drawColor,line width= 0.4pt,line join=round,line cap=round] (314.22,227.88) -- (332.22,227.88);

\path[draw=drawColor,line width= 0.4pt,dash pattern=on 4pt off 4pt ,line join=round,line cap=round] (314.22,215.88) -- (332.22,215.88);

\node[text=drawColor,anchor=base west,inner sep=0pt, outer sep=0pt, scale=  1.00] at (341.22,224.44) {deSolve (R)};

\node[text=drawColor,anchor=base west,inner sep=0pt, outer sep=0pt, scale=  1.00] at (341.22,212.44) {$y(t;k)+S_y(t;k)\cdot\Delta_k$};
\end{scope}
\end{tikzpicture}

  \caption{Linearization error compared to a direct solution with
    \textsc{r}'s \texttt{deSolve}. The error is very small.\label{fig:SmallForceRdeSolve} }
\end{figure}


\begin{figure}\sffamily\firalining
  \centering
  % Created by tikzDevice version 0.12.3.1 on 2021-06-16 15:27:58
% !TEX encoding = UTF-8 Unicode
\begin{tikzpicture}[x=1pt,y=1pt]
\definecolor{fillColor}{RGB}{255,255,255}
\path[use as bounding box,fill=fillColor,fill opacity=0.00] (0,0) rectangle (462.53,578.16);
\begin{scope}
\path[clip] ( 49.20,350.28) rectangle (437.33,528.96);
\definecolor{drawColor}{RGB}{0,0,0}

\path[draw=drawColor,line width= 0.4pt,line join=round,line cap=round] ( 63.58,510.11) --
	( 63.63,509.38) --
	( 63.72,508.29) --
	( 63.80,507.20) --
	( 63.89,506.11) --
	( 63.97,505.03) --
	( 64.06,503.96) --
	( 64.15,502.89) --
	( 64.27,501.39) --
	( 64.46,498.98) --
	( 64.79,494.93) --
	( 65.41,487.65) --
	( 66.34,477.05) --
	( 67.27,466.99) --
	( 68.20,457.46) --
	( 69.14,448.43) --
	( 70.07,439.92) --
	( 71.00,431.95) --
	( 71.93,424.38) --
	( 72.87,417.30) --
	( 73.94,409.76) --
	( 75.47,399.98) --
	( 78.92,382.18) --
	( 82.74,368.48) --
	( 86.07,361.10) --
	( 89.40,357.41) --
	( 92.73,356.90) --
	( 96.06,359.06) --
	( 99.38,363.42) --
	(102.71,369.59) --
	(105.58,376.04) --
	(108.44,383.30) --
	(111.31,391.18) --
	(114.17,399.49) --
	(117.04,408.06) --
	(120.16,417.55) --
	(124.46,430.55) --
	(130.18,447.22) --
	(135.13,460.62) --
	(140.07,472.81) --
	(145.02,483.61) --
	(150.31,493.53) --
	(155.82,502.08) --
	(161.34,508.89) --
	(166.85,514.07) --
	(172.37,517.81) --
	(177.88,520.30) --
	(183.64,521.78) --
	(189.76,522.34) --
	(196.20,522.08) --
	(202.85,521.16) --
	(209.81,519.75) --
	(217.38,517.96) --
	(226.15,515.85) --
	(237.37,513.42) --
	(249.20,511.45) --
	(258.05,510.40) --
	(266.89,509.68) --
	(275.73,509.26) --
	(284.93,509.08) --
	(294.87,509.08) --
	(305.48,509.22) --
	(316.08,509.44) --
	(327.12,509.67) --
	(339.49,509.90) --
	(355.46,510.11) --
	(375.94,510.21) --
	(391.61,510.21) --
	(407.28,510.18) --
	(422.95,510.15);
\end{scope}
\begin{scope}
\path[clip] (  0.00,  0.00) rectangle (462.53,578.16);
\definecolor{drawColor}{RGB}{0,0,0}

\path[draw=drawColor,line width= 0.4pt,line join=round,line cap=round] ( 63.58,350.28) -- (376.82,350.28);

\path[draw=drawColor,line width= 0.4pt,line join=round,line cap=round] ( 63.58,350.28) -- ( 63.58,344.28);

\path[draw=drawColor,line width= 0.4pt,line join=round,line cap=round] (141.89,350.28) -- (141.89,344.28);

\path[draw=drawColor,line width= 0.4pt,line join=round,line cap=round] (220.20,350.28) -- (220.20,344.28);

\path[draw=drawColor,line width= 0.4pt,line join=round,line cap=round] (298.51,350.28) -- (298.51,344.28);

\path[draw=drawColor,line width= 0.4pt,line join=round,line cap=round] (376.82,350.28) -- (376.82,344.28);

\node[text=drawColor,anchor=base,inner sep=0pt, outer sep=0pt, scale=  1.00] at ( 63.58,328.68) {0};

\node[text=drawColor,anchor=base,inner sep=0pt, outer sep=0pt, scale=  1.00] at (141.89,328.68) {5};

\node[text=drawColor,anchor=base,inner sep=0pt, outer sep=0pt, scale=  1.00] at (220.20,328.68) {10};

\node[text=drawColor,anchor=base,inner sep=0pt, outer sep=0pt, scale=  1.00] at (298.51,328.68) {15};

\node[text=drawColor,anchor=base,inner sep=0pt, outer sep=0pt, scale=  1.00] at (376.82,328.68) {20};

\path[draw=drawColor,line width= 0.4pt,line join=round,line cap=round] ( 49.20,361.08) -- ( 49.20,510.11);

\path[draw=drawColor,line width= 0.4pt,line join=round,line cap=round] ( 49.20,361.08) -- ( 43.20,361.08);

\path[draw=drawColor,line width= 0.4pt,line join=round,line cap=round] ( 49.20,398.34) -- ( 43.20,398.34);

\path[draw=drawColor,line width= 0.4pt,line join=round,line cap=round] ( 49.20,435.59) -- ( 43.20,435.59);

\path[draw=drawColor,line width= 0.4pt,line join=round,line cap=round] ( 49.20,472.85) -- ( 43.20,472.85);

\path[draw=drawColor,line width= 0.4pt,line join=round,line cap=round] ( 49.20,510.11) -- ( 43.20,510.11);

\node[text=drawColor,rotate= 90.00,anchor=base,inner sep=0pt, outer sep=0pt, scale=  1.00] at ( 34.80,361.08) {-0.08};

\node[text=drawColor,rotate= 90.00,anchor=base,inner sep=0pt, outer sep=0pt, scale=  1.00] at ( 34.80,398.34) {-0.06};

\node[text=drawColor,rotate= 90.00,anchor=base,inner sep=0pt, outer sep=0pt, scale=  1.00] at ( 34.80,435.59) {-0.04};

\node[text=drawColor,rotate= 90.00,anchor=base,inner sep=0pt, outer sep=0pt, scale=  1.00] at ( 34.80,472.85) {-0.02};

\node[text=drawColor,rotate= 90.00,anchor=base,inner sep=0pt, outer sep=0pt, scale=  1.00] at ( 34.80,510.11) {0.00};

\path[draw=drawColor,line width= 0.4pt,line join=round,line cap=round] ( 49.20,350.28) --
	(437.33,350.28) --
	(437.33,528.96) --
	( 49.20,528.96) --
	( 49.20,350.28);
\end{scope}
\begin{scope}
\path[clip] (  0.00,289.08) rectangle (462.53,578.16);
\definecolor{drawColor}{RGB}{0,0,0}

\node[text=drawColor,anchor=base,inner sep=0pt, outer sep=0pt, scale=  1.20] at (243.26,549.42) {\bfseries  with damping and driving force};

\node[text=drawColor,anchor=base,inner sep=0pt, outer sep=0pt, scale=  1.00] at (243.26,292.68) {deSolve solution with $\Delta_k=0.04$ at $F=0.3$ and $c=0.8$};

\node[text=drawColor,anchor=base,inner sep=0pt, outer sep=0pt, scale=  1.00] at (243.26,304.68) {$t$};

\node[text=drawColor,rotate= 90.00,anchor=base,inner sep=0pt, outer sep=0pt, scale=  1.00] at ( 10.80,439.62) {$v$};
\end{scope}
\begin{scope}
\path[clip] ( 49.20,350.28) rectangle (437.33,528.96);
\definecolor{drawColor}{RGB}{0,0,0}

\path[draw=drawColor,line width= 0.4pt,dash pattern=on 4pt off 4pt ,line join=round,line cap=round] ( 63.58,510.11) --
	( 63.63,509.38) --
	( 63.72,508.29) --
	( 63.80,507.20) --
	( 63.89,506.12) --
	( 63.97,505.04) --
	( 64.06,503.97) --
	( 64.15,502.90) --
	( 64.27,501.40) --
	( 64.46,499.00) --
	( 64.79,494.95) --
	( 65.41,487.68) --
	( 66.34,477.09) --
	( 67.27,467.04) --
	( 68.20,457.51) --
	( 69.14,448.49) --
	( 70.07,439.98) --
	( 71.00,431.96) --
	( 71.93,424.42) --
	( 72.87,417.34) --
	( 73.94,409.79) --
	( 75.47,399.94) --
	( 78.92,381.92) --
	( 82.74,367.95) --
	( 86.07,360.29) --
	( 89.40,356.26) --
	( 92.73,355.38) --
	( 96.06,357.16) --
	( 99.38,361.17) --
	(102.71,366.98) --
	(105.58,373.13) --
	(108.44,380.13) --
	(111.31,387.77) --
	(114.17,395.87) --
	(117.04,404.26) --
	(120.16,413.60) --
	(124.46,426.46) --
	(130.18,443.04) --
	(135.13,456.48) --
	(140.07,468.79) --
	(145.02,479.80) --
	(150.31,490.03) --
	(155.82,498.98) --
	(161.34,506.23) --
	(166.85,511.89) --
	(172.37,516.10) --
	(177.88,519.03) --
	(183.64,520.94) --
	(189.76,521.92) --
	(196.20,522.02) --
	(202.85,521.41) --
	(209.81,520.24) --
	(217.38,518.64) --
	(226.15,516.64) --
	(237.37,514.30) --
	(249.20,512.31) --
	(258.05,511.10) --
	(266.89,510.23) --
	(275.73,509.66) --
	(284.93,509.34) --
	(294.87,509.25) --
	(305.48,509.33) --
	(316.08,509.50) --
	(327.12,509.73) --
	(339.49,510.03) --
	(355.46,510.77) --
	(375.94,512.85) --
	(391.61,512.20) --
	(407.28,511.57) --
	(422.95,511.11);
\definecolor{fillColor}{RGB}{255,255,255}

\path[draw=drawColor,line width= 0.4pt,line join=round,line cap=round,fill=fillColor] (305.33,528.96) rectangle (437.33,492.96);

\path[draw=drawColor,line width= 0.4pt,line join=round,line cap=round] (314.33,516.96) -- (332.33,516.96);

\path[draw=drawColor,line width= 0.4pt,dash pattern=on 4pt off 4pt ,line join=round,line cap=round] (314.33,504.96) -- (332.33,504.96);

\node[text=drawColor,anchor=base west,inner sep=0pt, outer sep=0pt, scale=  1.00] at (341.33,513.52) {deSolve (R)};

\node[text=drawColor,anchor=base west,inner sep=0pt, outer sep=0pt, scale=  1.00] at (341.33,501.52) {$v(t;k) + S_v(t;k)\cdot\Delta_k$};
\end{scope}
\begin{scope}
\path[clip] ( 49.20, 61.20) rectangle (437.33,239.88);
\definecolor{drawColor}{RGB}{0,0,0}

\path[draw=drawColor,line width= 0.4pt,line join=round,line cap=round] ( 63.58,233.26) --
	( 63.63,233.26) --
	( 63.72,233.26) --
	( 63.80,233.26) --
	( 63.89,233.25) --
	( 63.97,233.24) --
	( 64.06,233.23) --
	( 64.15,233.22) --
	( 64.27,233.21) --
	( 64.46,233.17) --
	( 64.79,233.09) --
	( 65.41,232.87) --
	( 66.34,232.37) --
	( 67.27,231.67) --
	( 68.20,230.79) --
	( 69.14,229.75) --
	( 70.07,228.54) --
	( 71.00,227.13) --
	( 71.93,225.65) --
	( 72.87,224.02) --
	( 73.94,221.98) --
	( 75.47,218.74) --
	( 78.92,210.49) --
	( 82.74,200.20) --
	( 86.07,190.55) --
	( 89.40,180.55) --
	( 92.73,170.40) --
	( 96.06,160.30) --
	( 99.38,150.45) --
	(102.71,140.94) --
	(105.58,133.13) --
	(108.44,125.72) --
	(111.31,118.73) --
	(114.17,112.21) --
	(117.04,106.17) --
	(120.16,100.14) --
	(124.46, 92.80) --
	(130.18, 84.73) --
	(135.13, 79.23) --
	(140.07, 74.98) --
	(145.02, 71.86) --
	(150.31, 69.62) --
	(155.82, 68.31) --
	(161.34, 67.82) --
	(166.85, 67.99) --
	(172.37, 68.64) --
	(177.88, 69.63) --
	(183.64, 70.88) --
	(189.76, 72.34) --
	(196.20, 73.89) --
	(202.85, 75.41) --
	(209.81, 76.84) --
	(217.38, 78.16) --
	(226.15, 79.34) --
	(237.37, 80.35) --
	(249.20, 80.88) --
	(258.05, 81.03) --
	(266.89, 81.03) --
	(275.73, 80.93) --
	(284.93, 80.76) --
	(294.87, 80.56) --
	(305.48, 80.36) --
	(316.08, 80.20) --
	(327.12, 80.07) --
	(339.49, 79.99) --
	(355.46, 79.95) --
	(375.94, 79.97) --
	(391.61, 80.00) --
	(407.28, 80.03) --
	(422.95, 80.05);
\end{scope}
\begin{scope}
\path[clip] (  0.00,  0.00) rectangle (462.53,578.16);
\definecolor{drawColor}{RGB}{0,0,0}

\path[draw=drawColor,line width= 0.4pt,line join=round,line cap=round] ( 63.58, 61.20) -- (376.82, 61.20);

\path[draw=drawColor,line width= 0.4pt,line join=round,line cap=round] ( 63.58, 61.20) -- ( 63.58, 55.20);

\path[draw=drawColor,line width= 0.4pt,line join=round,line cap=round] (141.89, 61.20) -- (141.89, 55.20);

\path[draw=drawColor,line width= 0.4pt,line join=round,line cap=round] (220.20, 61.20) -- (220.20, 55.20);

\path[draw=drawColor,line width= 0.4pt,line join=round,line cap=round] (298.51, 61.20) -- (298.51, 55.20);

\path[draw=drawColor,line width= 0.4pt,line join=round,line cap=round] (376.82, 61.20) -- (376.82, 55.20);

\node[text=drawColor,anchor=base,inner sep=0pt, outer sep=0pt, scale=  1.00] at ( 63.58, 39.60) {0};

\node[text=drawColor,anchor=base,inner sep=0pt, outer sep=0pt, scale=  1.00] at (141.89, 39.60) {5};

\node[text=drawColor,anchor=base,inner sep=0pt, outer sep=0pt, scale=  1.00] at (220.20, 39.60) {10};

\node[text=drawColor,anchor=base,inner sep=0pt, outer sep=0pt, scale=  1.00] at (298.51, 39.60) {15};

\node[text=drawColor,anchor=base,inner sep=0pt, outer sep=0pt, scale=  1.00] at (376.82, 39.60) {20};

\path[draw=drawColor,line width= 0.4pt,line join=round,line cap=round] ( 49.20, 88.45) -- ( 49.20,233.26);

\path[draw=drawColor,line width= 0.4pt,line join=round,line cap=round] ( 49.20, 88.45) -- ( 43.20, 88.45);

\path[draw=drawColor,line width= 0.4pt,line join=round,line cap=round] ( 49.20,117.41) -- ( 43.20,117.41);

\path[draw=drawColor,line width= 0.4pt,line join=round,line cap=round] ( 49.20,146.37) -- ( 43.20,146.37);

\path[draw=drawColor,line width= 0.4pt,line join=round,line cap=round] ( 49.20,175.34) -- ( 43.20,175.34);

\path[draw=drawColor,line width= 0.4pt,line join=round,line cap=round] ( 49.20,204.30) -- ( 43.20,204.30);

\path[draw=drawColor,line width= 0.4pt,line join=round,line cap=round] ( 49.20,233.26) -- ( 43.20,233.26);

\node[text=drawColor,rotate= 90.00,anchor=base,inner sep=0pt, outer sep=0pt, scale=  1.00] at ( 34.80, 88.45) {0.75};

\node[text=drawColor,rotate= 90.00,anchor=base,inner sep=0pt, outer sep=0pt, scale=  1.00] at ( 34.80,117.41) {0.80};

\node[text=drawColor,rotate= 90.00,anchor=base,inner sep=0pt, outer sep=0pt, scale=  1.00] at ( 34.80,146.37) {0.85};

\node[text=drawColor,rotate= 90.00,anchor=base,inner sep=0pt, outer sep=0pt, scale=  1.00] at ( 34.80,175.34) {0.90};

\node[text=drawColor,rotate= 90.00,anchor=base,inner sep=0pt, outer sep=0pt, scale=  1.00] at ( 34.80,204.30) {0.95};

\node[text=drawColor,rotate= 90.00,anchor=base,inner sep=0pt, outer sep=0pt, scale=  1.00] at ( 34.80,233.26) {1.00};

\path[draw=drawColor,line width= 0.4pt,line join=round,line cap=round] ( 49.20, 61.20) --
	(437.33, 61.20) --
	(437.33,239.88) --
	( 49.20,239.88) --
	( 49.20, 61.20);
\end{scope}
\begin{scope}
\path[clip] (  0.00,  0.00) rectangle (462.53,289.08);
\definecolor{drawColor}{RGB}{0,0,0}

\node[text=drawColor,anchor=base,inner sep=0pt, outer sep=0pt, scale=  1.20] at (243.26,260.34) {\bfseries  with damping and driving force};

\node[text=drawColor,anchor=base,inner sep=0pt, outer sep=0pt, scale=  1.00] at (243.26,  3.60) {average missmatch: $0.11 \Delta_k$};

\node[text=drawColor,anchor=base,inner sep=0pt, outer sep=0pt, scale=  1.00] at (243.26, 15.60) {$t$};

\node[text=drawColor,rotate= 90.00,anchor=base,inner sep=0pt, outer sep=0pt, scale=  1.00] at ( 10.80,150.54) {$y$};
\end{scope}
\begin{scope}
\path[clip] ( 49.20, 61.20) rectangle (437.33,239.88);
\definecolor{drawColor}{RGB}{0,0,0}

\path[draw=drawColor,line width= 0.4pt,dash pattern=on 4pt off 4pt ,line join=round,line cap=round] ( 63.58,233.26) --
	( 63.63,233.26) --
	( 63.72,233.26) --
	( 63.80,233.25) --
	( 63.89,233.25) --
	( 63.97,233.24) --
	( 64.06,233.23) --
	( 64.15,233.22) --
	( 64.27,233.20) --
	( 64.46,233.16) --
	( 64.79,233.07) --
	( 65.41,232.84) --
	( 66.34,232.33) --
	( 67.27,231.63) --
	( 68.20,230.75) --
	( 69.14,229.69) --
	( 70.07,228.47) --
	( 71.00,227.10) --
	( 71.93,225.59) --
	( 72.87,223.94) --
	( 73.94,221.88) --
	( 75.47,218.67) --
	( 78.92,210.48) --
	( 82.74,200.20) --
	( 86.07,190.53) --
	( 89.40,180.47) --
	( 92.73,170.26) --
	( 96.06,160.07) --
	( 99.38,150.08) --
	(102.71,140.42) --
	(105.58,132.45) --
	(108.44,124.85) --
	(111.31,117.67) --
	(114.17,110.94) --
	(117.04,104.68) --
	(120.16, 98.40) --
	(124.46, 90.70) --
	(130.18, 82.14) --
	(135.13, 76.22) --
	(140.07, 71.56) --
	(145.02, 68.05) --
	(150.31, 65.41) --
	(155.82, 63.71) --
	(161.34, 62.90) --
	(166.85, 62.79) --
	(172.37, 63.22) --
	(177.88, 64.04) --
	(183.64, 65.18) --
	(189.76, 66.56) --
	(196.20, 68.09) --
	(202.85, 69.62) --
	(209.81, 71.11) --
	(217.38, 72.51) --
	(226.15, 73.81) --
	(237.37, 74.97) --
	(249.20, 75.67) --
	(258.05, 75.93) --
	(266.89, 76.02) --
	(275.73, 75.98) --
	(284.93, 75.85) --
	(294.87, 75.67) --
	(305.48, 75.47) --
	(316.08, 75.30) --
	(327.12, 75.16) --
	(339.49, 75.04) --
	(355.46, 74.85) --
	(375.94, 74.36) --
	(391.61, 74.87) --
	(407.28, 75.18) --
	(422.95, 75.32);
\definecolor{fillColor}{RGB}{255,255,255}

\path[draw=drawColor,line width= 0.4pt,line join=round,line cap=round,fill=fillColor] (305.22,239.88) rectangle (437.33,203.88);

\path[draw=drawColor,line width= 0.4pt,line join=round,line cap=round] (314.22,227.88) -- (332.22,227.88);

\path[draw=drawColor,line width= 0.4pt,dash pattern=on 4pt off 4pt ,line join=round,line cap=round] (314.22,215.88) -- (332.22,215.88);

\node[text=drawColor,anchor=base west,inner sep=0pt, outer sep=0pt, scale=  1.00] at (341.22,224.44) {deSolve (R)};

\node[text=drawColor,anchor=base west,inner sep=0pt, outer sep=0pt, scale=  1.00] at (341.22,212.44) {$y(t;k)+S_y(t;k)\cdot\Delta_k$};
\end{scope}
\end{tikzpicture}

  \caption{The linearization error for a bigger constant force, but
    otherwise calculated precisely as in
    Figure~\ref{fig:SmallForceRdeSolve}. The error is still small but clearly visible.  \label{fig:WithForceRdeSolve} }
\end{figure}

\end{document}