\documentclass[utf8,english,DIV=12]{scrartcl}
\usepackage[osf]{mathpazo}
\usepackage{FiraSans}
\usepackage[ttdefault=true]{AnonymousPro}
\usepackage[T1]{fontenc}
\usepackage[utf8]{inputenc}
\usepackage[names,dvipsnames]{xcolor}
\usepackage{amsmath,amssymb}
\usepackage{booktabs}
\usepackage{microtype}
\usepackage{graphicx}
\usepackage{listings}
\lstset{
  basicstyle=\ttfamily\footnotesize,%
  captionpos=b,                    % sets the caption-position to bottom
  commentstyle=\color{green!50!black},    % comment style
  keepspaces=true,                 % keeps spaces in text, useful for keeping indentation of code (possibly needs columns=flexible)
  keywordstyle=\color{blue},       % keyword style
  language=R,                 % the language of the code
  numbers=left,                    % where to put the line-numbers; possible values are (none, left, right)
  numbersep=5pt,                   % how far the line-numbers are from the code
  numberstyle=\tiny\color{gray}, % the style that is used for the line-numbers
  showstringspaces=true,          % underline spaces within strings only
  showtabs=true,                  % show tabs within strings adding particular underscores
  stepnumber=3,                    % the step between two line-numbers. If it's 1, each line will be numbered
  stringstyle=\color{Bittersweet},     % string literal style
}

\newcommand{\diag}{\text{diag}}

\frenchspacing
\setcapindent{0em}
\addtokomafont{captionlabel}{\bfseries}
\addtokomafont{caption}{\sffamily}
\addtokomafont{pagenumber}{\sffamily}
\pagestyle{headings}

\author{Andrei Kramer <andreikr@kth.se>}
\date{\today}
\title{Solution to Harmonic Oscillator}
\subject{Sensitivity Approximation and Verification}

\begin{document}
\maketitle
\begin{abstract}
  \noindent This part of the documentation justifies the expressions used in
  \texttt{verify.R} where we check the results of the sensitivity
  approximation method against the expected analytical result. The
  solution to the initial value problem of the Harmonic Oscillator is
  of course well known. The important part is its
  parameter-sensitivity (parameter derivative). We treat the damping
  parameter $c$ as a system input and the coefficiant $k$ of the main
  \emph{restoring force} as the \emph{interesting} parameter for sensitivity
  calculations. This choice is arbitrary.
\end{abstract}
\section{Model Definition}
\label{sec:model}

We will use a dampened harmonic oscillator model with unit mass: 
\begin{equation}
  \label{eq:ivp}
  \ddot y = -ky -c \dot y \,,
\end{equation}
to compare numerical $k$-sensitivity approximation methods to the
analytical solution sensitivity. This model is well suited for the
purpose of illustration as the sensitivity is a scalar function.

To use ordinary differential equation solvers, we reformulate the
model as a system of equatuions, with order one:
\begin{align}
  \dot v&=-ky -c v\,,&v(0)&=v_0\label{eq:y}\\
  \dot y&=v\,,&y(0)&=y_0\label{eq:v}\\
\end{align}
This system has a known analytical solution:
\begin{equation}
  \label{eq:sol}
  \begin{split}
    \omega&=\sqrt{k}\,,\\
    r&=\frac{c}{2\omega}\,,\\
    \Rightarrow r\omega&=\frac{c}{2}\,,\\
    y(t;k)&=a\exp(-r\omega t)\cos(\sqrt{1-r^2}\omega t+\phi)\,,\\
    \Rightarrow v(t;k)&=-a\exp(-r\omega t)(r\omega)\cos(\sqrt{1-r^2}\omega t+\phi)\\
    &\quad~- a\exp(-r\omega t)\sin(\sqrt{1-r^2}\omega t+\phi) \sqrt{1-r^2}\omega\,,\\
    &=-y(t;k)\left(r\omega + \tan(\sqrt{1-r^2}\omega t+\phi) \sqrt{1-r^2}\omega\right)\,,
  \end{split}
\end{equation}
where $a$ and $\phi$ are to be determined from the initial conditions
$v_0$ and $y_0$:
\begin{equation}
  \label{eq:init}
  \begin{split}
    y(0;k)&=a\cos(\phi)=y_0\,,\\
    v(0;k)&=-y_0r\omega-y_0\tan(\phi)\sqrt{1-r^2}\omega=v_0 \,,
  \end{split}
\end{equation}
In Section~\ref{sec:proof} we show that \eqref{eq:sol} solves the
ordinary differential equation~\eqref{eq:ivp}.

All four constants $a$, $\omega$, $\phi$, and $r$ are functions of $(k,c)$, so we could instead write:
$a(k,c)$, $r(k,c)$, etc.. But, to ease notation and avoid too many
parentheses, we will instead note the dependence on $k$ via
subscripts. Since we are interested in the sensitivity of $y$ only
with repsect to $k$ we will drop the dependence on $c$ in notation.

\subsection{Phase and Amplitude}
\label{sec:aphi}

We solve the initial value equations~\eqref{eq:init} for $\phi$:
\begin{align}
    \frac{v_0 + y_0r_k\omega_k}{y_0\sqrt{1-r_k^2}\omega_k} &= - \tan(\phi_k)\,\,,\label{eq:tanphi}\\
    \arctan\left(-\frac{2v_0 + y_0{c}}{2y_0\sqrt{1-r_k^2}\omega_k}\right) &= \phi_k\,.\label{eq:phi}
\end{align}
Here, we note that this solution for $\phi$ is also a function of $k$ and append this note as a subscript. 
This result makes $a$ immediately available as
\begin{equation}
  \label{eq:a}
  a_k = \frac{y_0}{\cos(\phi_k)}\,,
\end{equation}
where we make the dependence on $k$ noted once again. We disregard the dependence on 

\subsection{Sensitivity}
\label{sec:sens}

Because we are calculating the derivative with respect to $k$ it is
useful to rewrite the solution and make it explicit when a term does
depend on $k$, and show where $k$ cancels:
\begin{equation}
  \label{eq:yk}
  \begin{split}
    \Gamma_k:&=\sqrt{1-r_k^2}\,,\\
    \Gamma_k\omega_k&=\sqrt{1-\left(\frac{c}{2\sqrt{k}}\right)^2}\sqrt{k}=\sqrt{k-\frac{c^2k}{4k}}=\sqrt{k-\frac{c^2}{4}}\,,\\
  \Rightarrow y(t;k)&=a\exp\left(-\frac{c}{2}t\right)\cos\left(t\sqrt{k-\frac{c^2}{4}}+\phi_k\right)\,,
\end{split}
\end{equation}


The sensitivity of the solution $y(t;k)$ with respect to the parameter $k$ can be
obtained by straight forward differentiation, albeit with many terms. We collect a list of derivatives:
\begin{align}
  a_k &= \frac{y_0}{\cos(\phi_k)} & \frac{da_k}{dk} &= \frac{y_0\tan(\phi_k)}{\cos(\phi_k)}\frac{d\phi_k}{dk}=a_k\tan(\phi_k)\frac{d\phi_k}{dk}\,,\\
  \omega_k&=\sqrt{k} & \frac{d\omega_k}{dk}&=\frac{1}{2\sqrt{k}}\,,\\
  r_k&=\frac{c}{2\omega_k} & \frac{dr_k}{dk}&=-\frac{c}{2\omega_k^2}\frac{d\omega_k}{dk}=-\frac{c}{2 k}\frac{1}{2\sqrt{k}}=-\frac{r_k}{2k}\,,\\
  r_k\omega_k&=\frac{c}{2} & \frac{d(r_k\omega_k )}{dk}&=0\,,\\
  \Gamma_k&=\sqrt{1-r_k^2} & \frac{d\Gamma_k}{dk}&=\frac{2 r_k}{2\sqrt{1-r_k^2}}\frac{dr_k}{dk}=-\frac{r_k}{k}\frac{r_k}{2\Gamma_k}\,,\\
  \Gamma_k\omega_k&=\sqrt{k-\frac{c^2}{4}} & \frac{d(\Gamma_k\omega_k)}{dk}&=\frac{1}{2\sqrt{k-\frac{c^2}{4}}}=\frac{1}{2\Gamma_k\omega_k}\,,
\end{align}
First we take the derivative of $\phi$ from~\eqref{eq:tanphi}:
\begin{align}
  -\frac{d}{dk}\left(\frac{2v_0 + y_0c}{2y_0\Gamma_k\omega_k}\right) &= \frac{2}{cos(2\phi+1)} \frac{d\phi}{dk}\,,\\
  -\left(-\frac{2v_0 + y_0c}{4y_0(\Gamma_k\omega_k)^3}\right) &= \frac{2}{cos(2\phi+1)} \frac{d\phi_k}{dk}\,,
\end{align}
We solve for the derivative of $\phi$:
\begin{equation}
  \frac{d\phi_k}{dk}=\cos(2\phi_k+1)\frac{2v_0 + y_0c}{8y_0(\Gamma_k\omega_k)^3}\,,
\end{equation}
Using these results, we assemble the full sensitivity:
\begin{equation}
  \label{eq:dydk}
  \begin{split}
    \frac{dy(t;k)}{dk}&=\frac{d}{dk}\left(a_k\exp\left(-\frac{c}{2}t\right)\cos(\Gamma_k\omega_k t+\phi_k)\right)\,,\\
    &=\exp\left(-\frac{c}{2}t\right)\frac{d}{dk}\left(a_k\cos(\Gamma_k\omega_k t+\phi_k)\right)\,,\\
    &=\exp\left(-\frac{c}{2}t\right)\left(\frac{da_k}{dk}\cos(\Gamma_k\omega_k t+\phi_k) - a_k \sin(\Gamma_k\omega_k t+\phi_k)\left(\frac{t}{2\Gamma_k\omega_k}+\frac{d\phi_k}{dk}\right)\right)\,,\\
    &=\exp\left(-\frac{c}{2}t\right)\left(a_k\tan(\phi_k)\frac{d\phi_k}{dk}\cos(\Gamma_k\omega_k t+\phi_k) - a_k \sin(\Gamma_k\omega_k t+\phi_k)\left(\frac{t}{2\Gamma_k\omega_k}+\frac{d\phi_k}{dk}\right)\right)\,,\\
    &=y(t;k)\left(\tan(\phi_k)\frac{d\phi_k}{dk} - \tan(\Gamma_k\omega_k t+\phi_k)\left(\frac{t}{2\Gamma_k\omega_k}+\frac{d\phi_k}{dk}\right)\right)\,,
  \end{split}
\end{equation}
The last line is suitable for direct evaluation, given the parameters $(k,c)$. In \texttt{R} we calculate this using the function defined in Listing~\ref{lst:sens}.  
\begin{lstlisting}[caption={Calculation of the sensitivity in \texttt{R}. The variables have slightly different names: \texttt{srr1(k)} is $\Gamma_k$ and \texttt{srr1w(k)} is $\Gamma_k\omega_k$. The independent variable $t$ is \texttt{x} in the code.},label={lst:sens},float]
  w <- function(k) sqrt(k)
  r <- function(k) c/(2*w(k))                          # damping ratio
  srr1 <- function(k) sqrt(1-r(k)^2)                   # convenience
  ## srr1*w
  srr1w <- function(k) sqrt(k-0.25*c^2)
  ## phase and amplitude
  f <- function(k) atan(-(2*v0+y0*c)/(2*y0*srr1w(k))) 
  a <- function(k) y0/cos(f(k))
  ## derivatives
  dwdk <- function(k) 1/(2*w(k))
  dfdk <- function(k) cos(2*f(k) + 1)*(2*v0 + y0*c)/(8*y0*srr1w(k)^3)
  dadk <- function(k) y0 * (tan(f(k))/cos(f(k))) * dfdk(k)
  ## y and S = dy/dk
  y <- function(x,k) a(k)*exp(-0.5*c*x)*cos(srr1w(k)*x + f(k))
  S <- function(x,k) {
    y(x,k)*(  tan(f(k))*dfdk(k)
            - tan(srr1w(k)*x+f(k))*(0.5*x/srr1w(k) + dfdk(k)))
  }
\end{lstlisting}


\section{Proof for the Solution}
\label{sec:proof}

We insert the proposed solution~\eqref{eq:sol} into~\eqref{eq:ivp}. First we perform al needed derivatives:
\begin{equation}
  \begin{split}
    y(t;k)&=a\exp(-r\omega t)\cos(\sqrt{1-r^2}\omega t+\phi)\,,\\
    \dot y(t;k)&=-y(t;k)r\omega - a\exp(-r\omega t)\sin(\sqrt{1-r^2}\omega t+\phi) \sqrt{1-r^2}\omega\,,
  \end{split}\label{eq:proof}
\end{equation}
The second derivative of $y$:
\begin{multline}
  \ddot y(t;k)=-\dot y(t;k) r\omega \\
  - \left(a\exp(-r\omega t)(-r\omega)\sin(\sqrt{1-r^2}\omega t+\phi) \sqrt{1-r^2}\omega\right.\\
  + \left.a\exp(-r\omega t)\cos(\sqrt{1-r^2}\omega t+\phi) \left(\sqrt{1-r^2}\omega\right)^2\right)\,,  
\end{multline}
which simplifies to:
\begin{equation}
  \label{eq:ddoty}
  \begin{split}
  \ddot y(t;k)&=-\dot y(t;k) r\omega - \left(\underbrace{-a\exp(-r\omega t)\sin(\sqrt{1-r^2}\omega t+\phi)\sqrt{1-r^2}\omega}_{\dot y + yr\omega}   r\omega
    + y(t;k) \left(\sqrt{1-r^2}\omega\right)^2\right)\,, \\
  &=-\dot y(t;k) r\omega - \left(\dot y(t;k)r\omega + y(t;k)(r\omega)^2 
    + y(t;k) \left(\sqrt{1-r^2}\omega\right)^2\right)\,,\\
  &=-2\dot y(t;k) r\omega - \left(y(t;k)(r\omega)^2 
    + y(t;k) (1-r^2)\omega^2\right)\,,\\
  &=-2\dot y(t;k) r\omega - \left(y(t;k)\omega^2\right)\,,\\
  &=-2\dot y(t;k) \frac{c}{2} - y(t;k)k = - k y(t;k) - c\dot y(t;k)\,,
\end{split}
\end{equation}
which reconstructs the original ODE in~\eqref{eq:ivp}.
\end{document}